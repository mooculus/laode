\documentclass{ximera}

\usepackage{epsfig}

\graphicspath{
  {./}
  {figures/}
}


\usepackage{morewrites}

%\newcounter{ccounter}
%\setcounter{ccounter}{1}
%\newcommand{\Chapter}[1]{\setcounter{chapter}{\arabic{ccounter}}\chapter{#1}\addtocounter{ccounter}{1}}

%\newcommand{\section}[1]{\section{#1}\setcounter{thm}{0}\setcounter{equation}{0}}

%\renewcommand{\theequation}{\arabic{chapter}.\arabic{section}.\arabic{equation}}
%\renewcommand{\thefigure}{\arabic{chapter}.\arabic{figure}}
%\renewcommand{\thetable}{\arabic{chapter}.\arabic{table}}

%\newcommand{\Sec}[2]{\section{#1}\markright{\arabic{ccounter}.\arabic{section}.#2}\setcounter{equation}{0}\setcounter{thm}{0}\setcounter{figure}{0}}

\newcommand{\Sec}[2]{\section{#1}}

\setcounter{secnumdepth}{2}
%\setcounter{secnumdepth}{1} 

%\newcounter{THM}
%\renewcommand{\theTHM}{\arabic{chapter}.\arabic{section}}

\newcommand{\trademark}{{R\!\!\!\!\!\bigcirc}}
%\newtheorem{exercise}{}

\newcommand{\dfield}{{\sf dfield9}}
\newcommand{\pplane}{{\sf pplane9}}

\newcommand{\EXER}{\section*{Exercises}}%\vspace*{0.2in}\hrule\small\setcounter{exercise}{0}}
\newcommand{\CEXER}{}%\vspace{0.08in}\begin{center}Computer Exercises\end{center}}
\newcommand{\TEXER}{} %\vspace{0.08in}\begin{center}Hand Exercises\end{center}}
\newcommand{\AEXER}{} %\vspace{0.08in}\begin{center}Hand Exercises\end{center}}

% BADBAD: \newcommand{\Bbb}{\bf}

\newcommand{\R}{\mbox{$\Bbb{R}$}}
\newcommand{\C}{\mbox{$\Bbb{C}$}}
\newcommand{\Z}{\mbox{$\Bbb{Z}$}}
\newcommand{\N}{\mbox{$\Bbb{N}$}}
\newcommand{\D}{\mbox{{\bf D}}}
\usepackage{amssymb}
%\newcommand{\qed}{\hfill\mbox{\raggedright$\square$} \vspace{1ex}}
%\newcommand{\proof}{\noindent {\bf Proof:} \hspace{0.1in}}

\newcommand{\setmin}{\;\mbox{--}\;}
\newcommand{\Matlab}{{M\small{AT\-LAB}} }
\newcommand{\Matlabp}{{M\small{AT\-LAB}}}
\newcommand{\computer}{\Matlab Instructions}
\newcommand{\half}{\mbox{$\frac{1}{2}$}}
\newcommand{\compose}{\raisebox{.15ex}{\mbox{{\scriptsize$\circ$}}}}
\newcommand{\AND}{\quad\mbox{and}\quad}
\newcommand{\vect}[2]{\left(\begin{array}{c} #1_1 \\ \vdots \\
 #1_{#2}\end{array}\right)}
\newcommand{\mattwo}[4]{\left(\begin{array}{rr} #1 & #2\\ #3
&#4\end{array}\right)}
\newcommand{\mattwoc}[4]{\left(\begin{array}{cc} #1 & #2\\ #3
&#4\end{array}\right)}
\newcommand{\vectwo}[2]{\left(\begin{array}{r} #1 \\ #2\end{array}\right)}
\newcommand{\vectwoc}[2]{\left(\begin{array}{c} #1 \\ #2\end{array}\right)}



\newcommand{\inv}{^{-1}}
\newcommand{\CC}{{\cal C}}
\newcommand{\CCone}{\CC^1}
\newcommand{\Span}{{\rm span}}
\newcommand{\rank}{{\rm rank}}
\newcommand{\trace}{{\rm tr}}
\newcommand{\RE}{{\rm Re}}
\newcommand{\IM}{{\rm Im}}
\newcommand{\nulls}{{\rm null\;space}}

\newcommand{\dps}{\displaystyle}
\newcommand{\arraystart}{\renewcommand{\arraystretch}{1.8}}
\newcommand{\arrayfinish}{\renewcommand{\arraystretch}{1.2}}
\newcommand{\Start}[1]{\vspace{0.08in}\noindent {\bf Section~\ref{#1}}}
\newcommand{\exer}[1]{\noindent {\bf \ref{#1}}}
\newcommand{\ans}{}
\newcommand{\matthree}[9]{\left(\begin{array}{rrr} #1 & #2 & #3 \\ #4 & #5 & #6
\\ #7 & #8 & #9\end{array}\right)}
\newcommand{\cvectwo}[2]{\left(\begin{array}{c} #1 \\ #2\end{array}\right)}
\newcommand{\cmatthree}[9]{\left(\begin{array}{ccc} #1 & #2 & #3 \\ #4 & #5 &
#6 \\ #7 & #8 & #9\end{array}\right)}
\newcommand{\vecthree}[3]{\left(\begin{array}{r} #1 \\ #2 \\
#3\end{array}\right)}
\newcommand{\cvecthree}[3]{\left(\begin{array}{c} #1 \\ #2 \\
#3\end{array}\right)}
\newcommand{\cmattwo}[4]{\left(\begin{array}{cc} #1 & #2\\ #3
&#4\end{array}\right)}

\newcommand{\Matrix}[1]{\ensuremath{\left(\begin{array}{rrrrrrrrrrrrrrrrrr} #1 \end{array}\right)}}

\newcommand{\Matrixc}[1]{\ensuremath{\left(\begin{array}{cccccccccccc} #1 \end{array}\right)}}



\renewcommand{\labelenumi}{\theenumi)}
\newenvironment{enumeratea}%
{\begingroup
 \renewcommand{\theenumi}{\alph{enumi}}
 \renewcommand{\labelenumi}{(\theenumi)}
 \begin{enumerate}}
 {\end{enumerate}\endgroup}



\newcounter{help}
\renewcommand{\thehelp}{\thesection.\arabic{equation}}

%\newenvironment{equation*}%
%{\renewcommand\endequation{\eqno (\theequation)* $$}%
%   \begin{equation}}%
%   {\end{equation}\renewcommand\endequation{\eqno \@eqnnum
%$$\global\@ignoretrue}}

%\input{psfig.tex}

\author{Martin Golubitsky and Michael Dellnitz}

%\newenvironment{matlabEquation}%
%{\renewcommand\endequation{\eqno (\theequation*) $$}%
%   \begin{equation}}%
%   {\end{equation}\renewcommand\endequation{\eqno \@eqnnum
% $$\global\@ignoretrue}}

\newcommand{\soln}{\textbf{Solution:} }
\newcommand{\exercap}[1]{\centerline{Figure~\ref{#1}}}
\newcommand{\exercaptwo}[1]{\centerline{Figure~\ref{#1}a\hspace{2.1in}
Figure~\ref{#1}b}}
\newcommand{\exercapthree}[1]{\centerline{Figure~\ref{#1}a\hspace{1.2in}
Figure~\ref{#1}b\hspace{1.2in}Figure~\ref{#1}c}}
\newcommand{\para}{\hspace{0.4in}}

\renewenvironment{solution}{\suppress}{\endsuppress}

\ifxake
\newenvironment{matlabEquation}{\begin{equation}}{\end{equation}}
\else
\newenvironment{matlabEquation}%
{\let\oldtheequation\theequation\renewcommand{\theequation}{\oldtheequation*}\begin{equation}}%
  {\end{equation}\let\theequation\oldtheequation}
\fi

\makeatother


\title{Bifurcation Theory}

\begin{document}
\begin{abstract}
\end{abstract}
\maketitle


\label{C:BT}

\normalsize

Many applications are modeled by autonomous systems of differential 
equations that contain parameters. As these parameters
change, the stylized phase portraits of the differential
equations may also change; parameter values where these
changes occur are called {\em bifurcation\/}\index{bifurcation} 
values.  In this chapter we discuss how bifurcations occur.  To frame  
the discussion we introduce two systems of differential equations --- the 
Volterra-Lotka equations modeling the population evolution of two 
species (Section~\ref{S:TSPM}) and the CSTR equations \index{CSTR}
modeling a single exothermic (or heat producing) chemical reaction (Section~\ref{S:CSTR}).  In these models, we use scaling to identify the 
essential parameters, and we illustrate changes that can take place in 
phase portraits as a parameter is varied.  The information concerning 
changes in phase portraits is summarized in {\em bifurcation diagrams}, 
which are introduced by simple examples in Section~\ref{S:bifurcation}. 
Bifurcation diagrams are used in Section~\ref{S:CSTR} to summarize the 
results of numerical explorations on the CSTR.

In Chapter~\ref{C:NPS} we showed that stylized phase portraits of planar 
Morse-Smale \index{Morse-Smale system} systems can often be drawn by a 
combination of analysis and computer.  In this chapter we observe that 
bifurcations occur at parameter values where phase portraits of systems of 
differential equations are not Morse-Smale.  There are three ways that an 
autonomous planar system of differential equations can fail to be 
Morse-Smale.
\begin{itemize}
\item  There is a nonhyperbolic equilibrium.\index{nonhyperbolic} 
\index{equilibrium!nonhyperbolic}
\item  There is a periodic solution that is not a limit cycle.
\item  There is a trajectory that connects a saddle to itself or
that connects two different saddles.  
\end{itemize}
Typically, in a system of differential equations that depends on a single
parameter $\rho$, there are isolated values in $\rho$ where the 
differential equation fails to be Morse-Smale and at these bifurcation
values the phase portraits do actually change.

Section~\ref{S:bifurcation} discusses the typical bifurcations associated 
with nonhyperbolic equilibria: saddle-node bifurcations (where two equilibria 
are created) and Hopf bifurcations (where limit cycles are created).  These 
bifurcations are called {\em local bifurcations\/} as the changes in the phase 
portraits occur in a small neighborhood of the nonhyperbolic equilibrium.  

We also describe typical ways in which the remaining two failures in 
Morse-Smale occur.  Typically, when periodic solutions are not limit cycles, 
two periodic solutions collide and disappear (Section~\ref{S:GlobalBif}).  
Other global bifurcations occur when there are saddle-saddle connections.  
Homoclinic bifurcations occur when there is a connection from a saddle point 
to itself --- typically such bifurcations occur when a limit cycle disappears
(Section~\ref{S:bifurcation}).   Heteroclinic bifurcations occur when a 
trajectory connects two different saddles (Section~\ref{S:GlobalBif}).

Additional details concerning saddle-node and Hopf bifurcations are given in 
Sections~\ref{S:SNB} and \ref{S:HopfBif}.




\end{document}
