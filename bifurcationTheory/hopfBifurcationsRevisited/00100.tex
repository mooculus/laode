\documentclass{ximera}
\usepackage{epsfig}

\graphicspath{
  {./}
  {figures/}
}


\usepackage{morewrites}

%\newcounter{ccounter}
%\setcounter{ccounter}{1}
%\newcommand{\Chapter}[1]{\setcounter{chapter}{\arabic{ccounter}}\chapter{#1}\addtocounter{ccounter}{1}}

%\newcommand{\section}[1]{\section{#1}\setcounter{thm}{0}\setcounter{equation}{0}}

%\renewcommand{\theequation}{\arabic{chapter}.\arabic{section}.\arabic{equation}}
%\renewcommand{\thefigure}{\arabic{chapter}.\arabic{figure}}
%\renewcommand{\thetable}{\arabic{chapter}.\arabic{table}}

%\newcommand{\Sec}[2]{\section{#1}\markright{\arabic{ccounter}.\arabic{section}.#2}\setcounter{equation}{0}\setcounter{thm}{0}\setcounter{figure}{0}}

\newcommand{\Sec}[2]{\section{#1}}

\setcounter{secnumdepth}{2}
%\setcounter{secnumdepth}{1} 

%\newcounter{THM}
%\renewcommand{\theTHM}{\arabic{chapter}.\arabic{section}}

\newcommand{\trademark}{{R\!\!\!\!\!\bigcirc}}
%\newtheorem{exercise}{}

\newcommand{\dfield}{{\sf dfield9}}
\newcommand{\pplane}{{\sf pplane9}}

\newcommand{\EXER}{\section*{Exercises}}%\vspace*{0.2in}\hrule\small\setcounter{exercise}{0}}
\newcommand{\CEXER}{}%\vspace{0.08in}\begin{center}Computer Exercises\end{center}}
\newcommand{\TEXER}{} %\vspace{0.08in}\begin{center}Hand Exercises\end{center}}
\newcommand{\AEXER}{} %\vspace{0.08in}\begin{center}Hand Exercises\end{center}}

% BADBAD: \newcommand{\Bbb}{\bf}

\newcommand{\R}{\mbox{$\Bbb{R}$}}
\newcommand{\C}{\mbox{$\Bbb{C}$}}
\newcommand{\Z}{\mbox{$\Bbb{Z}$}}
\newcommand{\N}{\mbox{$\Bbb{N}$}}
\newcommand{\D}{\mbox{{\bf D}}}
\usepackage{amssymb}
%\newcommand{\qed}{\hfill\mbox{\raggedright$\square$} \vspace{1ex}}
%\newcommand{\proof}{\noindent {\bf Proof:} \hspace{0.1in}}

\newcommand{\setmin}{\;\mbox{--}\;}
\newcommand{\Matlab}{{M\small{AT\-LAB}} }
\newcommand{\Matlabp}{{M\small{AT\-LAB}}}
\newcommand{\computer}{\Matlab Instructions}
\newcommand{\half}{\mbox{$\frac{1}{2}$}}
\newcommand{\compose}{\raisebox{.15ex}{\mbox{{\scriptsize$\circ$}}}}
\newcommand{\AND}{\quad\mbox{and}\quad}
\newcommand{\vect}[2]{\left(\begin{array}{c} #1_1 \\ \vdots \\
 #1_{#2}\end{array}\right)}
\newcommand{\mattwo}[4]{\left(\begin{array}{rr} #1 & #2\\ #3
&#4\end{array}\right)}
\newcommand{\mattwoc}[4]{\left(\begin{array}{cc} #1 & #2\\ #3
&#4\end{array}\right)}
\newcommand{\vectwo}[2]{\left(\begin{array}{r} #1 \\ #2\end{array}\right)}
\newcommand{\vectwoc}[2]{\left(\begin{array}{c} #1 \\ #2\end{array}\right)}



\newcommand{\inv}{^{-1}}
\newcommand{\CC}{{\cal C}}
\newcommand{\CCone}{\CC^1}
\newcommand{\Span}{{\rm span}}
\newcommand{\rank}{{\rm rank}}
\newcommand{\trace}{{\rm tr}}
\newcommand{\RE}{{\rm Re}}
\newcommand{\IM}{{\rm Im}}
\newcommand{\nulls}{{\rm null\;space}}

\newcommand{\dps}{\displaystyle}
\newcommand{\arraystart}{\renewcommand{\arraystretch}{1.8}}
\newcommand{\arrayfinish}{\renewcommand{\arraystretch}{1.2}}
\newcommand{\Start}[1]{\vspace{0.08in}\noindent {\bf Section~\ref{#1}}}
\newcommand{\exer}[1]{\noindent {\bf \ref{#1}}}
\newcommand{\ans}{}
\newcommand{\matthree}[9]{\left(\begin{array}{rrr} #1 & #2 & #3 \\ #4 & #5 & #6
\\ #7 & #8 & #9\end{array}\right)}
\newcommand{\cvectwo}[2]{\left(\begin{array}{c} #1 \\ #2\end{array}\right)}
\newcommand{\cmatthree}[9]{\left(\begin{array}{ccc} #1 & #2 & #3 \\ #4 & #5 &
#6 \\ #7 & #8 & #9\end{array}\right)}
\newcommand{\vecthree}[3]{\left(\begin{array}{r} #1 \\ #2 \\
#3\end{array}\right)}
\newcommand{\cvecthree}[3]{\left(\begin{array}{c} #1 \\ #2 \\
#3\end{array}\right)}
\newcommand{\cmattwo}[4]{\left(\begin{array}{cc} #1 & #2\\ #3
&#4\end{array}\right)}

\newcommand{\Matrix}[1]{\ensuremath{\left(\begin{array}{rrrrrrrrrrrrrrrrrr} #1 \end{array}\right)}}

\newcommand{\Matrixc}[1]{\ensuremath{\left(\begin{array}{cccccccccccc} #1 \end{array}\right)}}



\renewcommand{\labelenumi}{\theenumi)}
\newenvironment{enumeratea}%
{\begingroup
 \renewcommand{\theenumi}{\alph{enumi}}
 \renewcommand{\labelenumi}{(\theenumi)}
 \begin{enumerate}}
 {\end{enumerate}\endgroup}



\newcounter{help}
\renewcommand{\thehelp}{\thesection.\arabic{equation}}

%\newenvironment{equation*}%
%{\renewcommand\endequation{\eqno (\theequation)* $$}%
%   \begin{equation}}%
%   {\end{equation}\renewcommand\endequation{\eqno \@eqnnum
%$$\global\@ignoretrue}}

%\input{psfig.tex}

\author{Martin Golubitsky and Michael Dellnitz}

%\newenvironment{matlabEquation}%
%{\renewcommand\endequation{\eqno (\theequation*) $$}%
%   \begin{equation}}%
%   {\end{equation}\renewcommand\endequation{\eqno \@eqnnum
% $$\global\@ignoretrue}}

\newcommand{\soln}{\textbf{Solution:} }
\newcommand{\exercap}[1]{\centerline{Figure~\ref{#1}}}
\newcommand{\exercaptwo}[1]{\centerline{Figure~\ref{#1}a\hspace{2.1in}
Figure~\ref{#1}b}}
\newcommand{\exercapthree}[1]{\centerline{Figure~\ref{#1}a\hspace{1.2in}
Figure~\ref{#1}b\hspace{1.2in}Figure~\ref{#1}c}}
\newcommand{\para}{\hspace{0.4in}}

\renewenvironment{solution}{\suppress}{\endsuppress}

\ifxake
\newenvironment{matlabEquation}{\begin{equation}}{\end{equation}}
\else
\newenvironment{matlabEquation}%
{\let\oldtheequation\theequation\renewcommand{\theequation}{\oldtheequation*}\begin{equation}}%
  {\end{equation}\let\theequation\oldtheequation}
\fi

\makeatother

\begin{document}

\TEXER

\noindent In Exercises~\ref{c9.6.1a} -- \ref{c9.6.1b} find points of Hopf
bifurcation and check for these points whether the eigenvalue crossing
condition \eqref{e:2dhopf} is satisfied.
\begin{exercise} \label{c9.6.1a}
$\begin{array}{rcl}
\dot{x} & = & \rho x - y -x^3 \\
\dot{y} & = & x.
\end{array}$

\begin{solution}

\ans The only equilibria of this system occur at the origin and a point of
Hopf bifurcation occurs at $\rho=0$.  The eigenvalue crossing condition is
satisfied.

\soln  To find the equilibria solve $\dot{y}=0$ for $x=0$ and $\dot{x}=0$ for
$y=0$.  The Jacobian matrix at the origin is 
\[
J(\rho) = \mattwo{\rho}{-1}{1}{0}.
\]
Thus, $\trace(J(\rho))=0$ only when $\rho=0$.  So a Hopf bifurcation point
can only occur at $\rho=0$.  Finally, note that 
\[
\frac{d}{dt}\trace(J(\rho)) = 1 \neq 0;
\]
So the eigenvalue crossing condition is satisfied.


\end{solution}
\end{exercise}
\begin{exercise} \label{c9.6.1b}
$\begin{array}{rcl}
\dot{x} & = & y - \rho x - x^2   \\
\dot{y} & = & -x - \rho y + x^2.
\end{array}$

\begin{solution}

\ans  There are two possible points of Hopf bifurcation: one at $(0,0)$ 
when $\rho=0$ and $(1,0)$ at $\rho=-1$.  The eigenvalue crossing condition is
satisfied at both points.

\soln  The system of differential equations
\[
\begin{array}{rcl}
\dot{x} & = & y - \rho x - x^2   \\
\dot{y} & = & -x - \rho y + x^2.
\end{array}
\]
has Jacobian matrix 
\[
J = \cmattwo{-\rho-2x}{1}{-1+2x}{-\rho}.
\]
Since a point of Hopf bifurcation must be at an equilibrium where the trace
of the Jacobian matrix is zero, points of Hopf bifurcation must satisfy the
three equations:
\begin{eqnarray*}
y - \rho x - x^2 & = & 0\\
-x - \rho y + x^2 & = & 0\\
-2\rho-2x & = & 0.
\end{eqnarray*}
On substituting $\rho = -x$ into the first two equations, we obtain:
\begin{eqnarray*}
y  & = & 0\\
-x + xy + x^2 & = & 0
\end{eqnarray*}
From these equations we see that there are two possible points of Hopf
bifurcation $(0,0)$ at $\rho=0$ or $(1,0)$ at $\rho=-1$.  

To check the eigenvalue crossing condition we must compute the trace of the
Jacobian matrix along the branch of equilibria.  To find the branches we must
solve the equilibrium equations
\begin{eqnarray*}
y - \rho x - x^2 & = & 0\\
-x - \rho y + x^2 & = & 0
\end{eqnarray*}
for $x$ and $y$ as a function of $\rho$.  Substituting $y - \rho x+x^2$ into
the $2^{nd}$ equation yields:
\[
-(1+\rho^2)x + (1-\rho)x^2 = 0.
\]
Therefore either 
\[
x= 0 \quad \mbox{ or } \quad  x = \frac{1+\rho^2}{1-\rho}.
\]
If $x=0$, then $y=0$ and the trace of the Jacobian along this branch is
$-2\rho$.  Since the derivative with respect to $\rho$ equals $-2$, the
eigenvalue crossing condition is satisfied.

In the second case, the trace of the Jacobian along the branch of equilibria
equals
\[
-2\rho-2x = -2\rho -2\frac{1+\rho^2}{1-\rho}.
\]
The derivative of this expression with respect to $\rho$ is $3$ at $\rho=-1$.
Hence the eigenvalue crossing condition is satisfied.




\end{solution}
\end{exercise}
\end{document}
