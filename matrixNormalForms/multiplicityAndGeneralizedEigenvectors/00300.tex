\documentclass{ximera}
\usepackage{epsfig}

\graphicspath{
  {./}
  {figures/}
}


\usepackage{morewrites}

%\newcounter{ccounter}
%\setcounter{ccounter}{1}
%\newcommand{\Chapter}[1]{\setcounter{chapter}{\arabic{ccounter}}\chapter{#1}\addtocounter{ccounter}{1}}

%\newcommand{\section}[1]{\section{#1}\setcounter{thm}{0}\setcounter{equation}{0}}

%\renewcommand{\theequation}{\arabic{chapter}.\arabic{section}.\arabic{equation}}
%\renewcommand{\thefigure}{\arabic{chapter}.\arabic{figure}}
%\renewcommand{\thetable}{\arabic{chapter}.\arabic{table}}

%\newcommand{\Sec}[2]{\section{#1}\markright{\arabic{ccounter}.\arabic{section}.#2}\setcounter{equation}{0}\setcounter{thm}{0}\setcounter{figure}{0}}

\newcommand{\Sec}[2]{\section{#1}}

\setcounter{secnumdepth}{2}
%\setcounter{secnumdepth}{1} 

%\newcounter{THM}
%\renewcommand{\theTHM}{\arabic{chapter}.\arabic{section}}

\newcommand{\trademark}{{R\!\!\!\!\!\bigcirc}}
%\newtheorem{exercise}{}

\newcommand{\dfield}{{\sf dfield9}}
\newcommand{\pplane}{{\sf pplane9}}

\newcommand{\EXER}{\section*{Exercises}}%\vspace*{0.2in}\hrule\small\setcounter{exercise}{0}}
\newcommand{\CEXER}{}%\vspace{0.08in}\begin{center}Computer Exercises\end{center}}
\newcommand{\TEXER}{} %\vspace{0.08in}\begin{center}Hand Exercises\end{center}}
\newcommand{\AEXER}{} %\vspace{0.08in}\begin{center}Hand Exercises\end{center}}

% BADBAD: \newcommand{\Bbb}{\bf}

\newcommand{\R}{\mbox{$\Bbb{R}$}}
\newcommand{\C}{\mbox{$\Bbb{C}$}}
\newcommand{\Z}{\mbox{$\Bbb{Z}$}}
\newcommand{\N}{\mbox{$\Bbb{N}$}}
\newcommand{\D}{\mbox{{\bf D}}}
\usepackage{amssymb}
%\newcommand{\qed}{\hfill\mbox{\raggedright$\square$} \vspace{1ex}}
%\newcommand{\proof}{\noindent {\bf Proof:} \hspace{0.1in}}

\newcommand{\setmin}{\;\mbox{--}\;}
\newcommand{\Matlab}{{M\small{AT\-LAB}} }
\newcommand{\Matlabp}{{M\small{AT\-LAB}}}
\newcommand{\computer}{\Matlab Instructions}
\newcommand{\half}{\mbox{$\frac{1}{2}$}}
\newcommand{\compose}{\raisebox{.15ex}{\mbox{{\scriptsize$\circ$}}}}
\newcommand{\AND}{\quad\mbox{and}\quad}
\newcommand{\vect}[2]{\left(\begin{array}{c} #1_1 \\ \vdots \\
 #1_{#2}\end{array}\right)}
\newcommand{\mattwo}[4]{\left(\begin{array}{rr} #1 & #2\\ #3
&#4\end{array}\right)}
\newcommand{\mattwoc}[4]{\left(\begin{array}{cc} #1 & #2\\ #3
&#4\end{array}\right)}
\newcommand{\vectwo}[2]{\left(\begin{array}{r} #1 \\ #2\end{array}\right)}
\newcommand{\vectwoc}[2]{\left(\begin{array}{c} #1 \\ #2\end{array}\right)}



\newcommand{\inv}{^{-1}}
\newcommand{\CC}{{\cal C}}
\newcommand{\CCone}{\CC^1}
\newcommand{\Span}{{\rm span}}
\newcommand{\rank}{{\rm rank}}
\newcommand{\trace}{{\rm tr}}
\newcommand{\RE}{{\rm Re}}
\newcommand{\IM}{{\rm Im}}
\newcommand{\nulls}{{\rm null\;space}}

\newcommand{\dps}{\displaystyle}
\newcommand{\arraystart}{\renewcommand{\arraystretch}{1.8}}
\newcommand{\arrayfinish}{\renewcommand{\arraystretch}{1.2}}
\newcommand{\Start}[1]{\vspace{0.08in}\noindent {\bf Section~\ref{#1}}}
\newcommand{\exer}[1]{\noindent {\bf \ref{#1}}}
\newcommand{\ans}{}
\newcommand{\matthree}[9]{\left(\begin{array}{rrr} #1 & #2 & #3 \\ #4 & #5 & #6
\\ #7 & #8 & #9\end{array}\right)}
\newcommand{\cvectwo}[2]{\left(\begin{array}{c} #1 \\ #2\end{array}\right)}
\newcommand{\cmatthree}[9]{\left(\begin{array}{ccc} #1 & #2 & #3 \\ #4 & #5 &
#6 \\ #7 & #8 & #9\end{array}\right)}
\newcommand{\vecthree}[3]{\left(\begin{array}{r} #1 \\ #2 \\
#3\end{array}\right)}
\newcommand{\cvecthree}[3]{\left(\begin{array}{c} #1 \\ #2 \\
#3\end{array}\right)}
\newcommand{\cmattwo}[4]{\left(\begin{array}{cc} #1 & #2\\ #3
&#4\end{array}\right)}

\newcommand{\Matrix}[1]{\ensuremath{\left(\begin{array}{rrrrrrrrrrrrrrrrrr} #1 \end{array}\right)}}

\newcommand{\Matrixc}[1]{\ensuremath{\left(\begin{array}{cccccccccccc} #1 \end{array}\right)}}



\renewcommand{\labelenumi}{\theenumi)}
\newenvironment{enumeratea}%
{\begingroup
 \renewcommand{\theenumi}{\alph{enumi}}
 \renewcommand{\labelenumi}{(\theenumi)}
 \begin{enumerate}}
 {\end{enumerate}\endgroup}



\newcounter{help}
\renewcommand{\thehelp}{\thesection.\arabic{equation}}

%\newenvironment{equation*}%
%{\renewcommand\endequation{\eqno (\theequation)* $$}%
%   \begin{equation}}%
%   {\end{equation}\renewcommand\endequation{\eqno \@eqnnum
%$$\global\@ignoretrue}}

%\input{psfig.tex}

\author{Martin Golubitsky and Michael Dellnitz}

%\newenvironment{matlabEquation}%
%{\renewcommand\endequation{\eqno (\theequation*) $$}%
%   \begin{equation}}%
%   {\end{equation}\renewcommand\endequation{\eqno \@eqnnum
% $$\global\@ignoretrue}}

\newcommand{\soln}{\textbf{Solution:} }
\newcommand{\exercap}[1]{\centerline{Figure~\ref{#1}}}
\newcommand{\exercaptwo}[1]{\centerline{Figure~\ref{#1}a\hspace{2.1in}
Figure~\ref{#1}b}}
\newcommand{\exercapthree}[1]{\centerline{Figure~\ref{#1}a\hspace{1.2in}
Figure~\ref{#1}b\hspace{1.2in}Figure~\ref{#1}c}}
\newcommand{\para}{\hspace{0.4in}}

\renewenvironment{solution}{\suppress}{\endsuppress}

\ifxake
\newenvironment{matlabEquation}{\begin{equation}}{\end{equation}}
\else
\newenvironment{matlabEquation}%
{\let\oldtheequation\theequation\renewcommand{\theequation}{\oldtheequation*}\begin{equation}}%
  {\end{equation}\let\theequation\oldtheequation}
\fi

\makeatother

\begin{document}



\noindent In Exercises~\ref{c10.5.3A} -- \ref{c10.5.3B}, use \Matlab to find 
the eigenvalues and their algebraic and geometric multiplicities for the given 
matrix.
\begin{computerExercise} \label{c10.5.3A}
\begin{matlabEquation}\label{eigenvalue-exercise}
A=\left(\begin{array}{rrrr} 2 & 3 & -21 & -3 \\2 & 7 & -41 & -5 \\ 
0 & 1 & -5 & -1 \\ 0 & 0 & 4 & 4 \end{array}
\right).
\end{matlabEquation}

\begin{solution}
\ans The eigenvalue $2$ has algebraic multiplicity $4$ and geometric
multiplicity $1$.

\soln Using \Matlab to find eigenvalues of high algebraic multiplicity is numerically
dangerous. Type {\tt eig(A)} and obtain
\begin{verbatim}
   2.0000 + 0.0006i
   2.0000 - 0.0006i
   1.9994          
   2.0006          
\end{verbatim}
Since the coefficients of $A$ are all integers, you might be suspicious of the answer
and guess that all of the eigenvalues of $A$ equal $2$.  Type {\tt null(A-2*eye(4))}
and obtain
\begin{verbatim}
ans =
   -0.4804
   -0.8006
   -0.1601
    0.3203
\end{verbatim}
dividing by {\tt ans(3)} yields the eigenvector $v_1=(3,5,1,-2)$.  To check whether
the eigenvalue $2$ has algebraic multiplicity greater than $1$, type 
{\tt null((A-eye(4))\^{}2)} and obtain
\begin{verbatim}
ans =
    0.5071         0
    0.8452         0
    0.1690         0
         0    1.0000
\end{verbatim}
Thus $v_2=(0,0,0,1)$ is a generalized eigenvector of $A$ with index $2$ and eigenvalue
$2$.  To find generalized eigenvectors of index $3$ type {\tt null((A-eye(4))\^{}3)} 
and obtain
\begin{verbatim}
ans =
   -0.9487         0         0
         0    1.0000         0
   -0.3162         0         0
         0         0    1.0000
\end{verbatim}
Thus, $v_3=(0,1,0,0)$ is a generalized eigenvector of index $3$.  Type 
{\tt null((A-eye(4))\^{}4)} and obtain
\begin{verbatim}
ans =
     1     0     0     0
     0     1     0     0
     0     0     1     0
     0     0     0     1
\end{verbatim}
to see that $2$ is an eigenvalue of $A$ of algebraic multiplicity $4$ and geometric
multiplicity $1$.  


\end{solution}
\end{computerExercise}
\begin{computerExercise} \label{c10.5.3B}
\begin{matlabEquation}\label{eigenvalue-exercise-2}
B=\left(\begin{array}{rrrrr} 179 & -230 & 0 & 10 & -30 \\
144 & -185 & 0 & 8 & -24 \\ 30 & -39 & -1 & 3 & -9 \\ 192 & -245 & 0 & 9 & -30 
\\ 40 & -51 & 0 & 2 & -7\end{array}\right).
\end{matlabEquation}

\begin{solution}
\ans The eigenvalue $-1$ has algebraic multiplicity $5$ and geometric
multiplicity $3$.

\soln Type {\tt eig(A)} and obtain
\begin{verbatim}
ans =
  -1.0000          
  -1.0000 + 0.0000i
  -1.0000 - 0.0000i
  -1.0000          
  -1.0000         
\end{verbatim}
Type {\tt null(B+eye(5))}
and obtain
\begin{verbatim}
ans =
    0.7701   -0.1043    0.0000
    0.6160   -0.0835    0.0000
   -0.0000   -0.0000   -1.0000
   -0.0966   -0.9443    0.0000
   -0.1349   -0.3008    0.0000
\end{verbatim}
We see that $-1$ is an eigenvalue of $B$ with geometric multiplicity $5$. 





\end{solution}
\end{computerExercise}
\end{document}
