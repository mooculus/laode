\documentclass{ximera}

\usepackage{epsfig}

\graphicspath{
  {./}
  {figures/}
}

\usepackage{epstopdf}
%\usepackage{ulem}
\usepackage[normalem]{ulem}

\epstopdfsetup{outdir=./}

\usepackage{morewrites}
\makeatletter
\newcommand\subfile[1]{%
\renewcommand{\input}[1]{}%
\begingroup\skip@preamble\otherinput{#1}\endgroup\par\vspace{\topsep}
\let\input\otherinput}
\makeatother

\newcommand{\EXER}{}
\newcommand{\includeexercises}{\EXER\directlua{dofile(kpse.find_file("exercises","lua"))}}

\newenvironment{computerExercise}{\begin{exercise}}{\end{exercise}}

%\newcounter{ccounter}
%\setcounter{ccounter}{1}
%\newcommand{\Chapter}[1]{\setcounter{chapter}{\arabic{ccounter}}\chapter{#1}\addtocounter{ccounter}{1}}

%\newcommand{\section}[1]{\section{#1}\setcounter{thm}{0}\setcounter{equation}{0}}

%\renewcommand{\theequation}{\arabic{chapter}.\arabic{section}.\arabic{equation}}
%\renewcommand{\thefigure}{\arabic{chapter}.\arabic{figure}}
%\renewcommand{\thetable}{\arabic{chapter}.\arabic{table}}

%\newcommand{\Sec}[2]{\section{#1}\markright{\arabic{ccounter}.\arabic{section}.#2}\setcounter{equation}{0}\setcounter{thm}{0}\setcounter{figure}{0}}
  
\newcommand{\Sec}[2]{\section{#1}}

\setcounter{secnumdepth}{2}
%\setcounter{secnumdepth}{1} 

%\newcounter{THM}
%\renewcommand{\theTHM}{\arabic{chapter}.\arabic{section}}

\newcommand{\trademark}{{R\!\!\!\!\!\bigcirc}}
%\newtheorem{exercise}{}

\newcommand{\dfield}{{\sf SlopeField}}

\newcommand{\pplane}{{\sf PhasePlane}}

\newcommand{\PPLANE}{{\sf PHASEPLANE}}

% BADBAD: \newcommand{\Bbb}{\bf}. % Package amsfonts Warning: Obsolete command \Bbb; \mathbb should be used instead.

\newcommand{\R}{\mbox{$\mathbb{R}$}}
\let\C\relax
\newcommand{\C}{\mbox{$\mathbb{C}$}}
\newcommand{\Z}{\mbox{$\mathbb{Z}$}}
\newcommand{\N}{\mbox{$\mathbb{N}$}}
\newcommand{\D}{\mbox{{\bf D}}}

\newcommand{\WW}{\mathcal{W}}

\usepackage{amssymb}
%\newcommand{\qed}{\hfill\mbox{\raggedright$\square$} \vspace{1ex}}
%\newcommand{\proof}{\noindent {\bf Proof:} \hspace{0.1in}}

\newcommand{\setmin}{\;\mbox{--}\;}
\newcommand{\Matlab}{{M\small{AT\-LAB}} }
\newcommand{\Matlabp}{{M\small{AT\-LAB}}}
\newcommand{\computer}{\Matlab Instructions}
\renewcommand{\computer}{M\small{ATLAB} Instructions}
\newcommand{\half}{\mbox{$\frac{1}{2}$}}
\newcommand{\compose}{\raisebox{.15ex}{\mbox{{\scriptsize$\circ$}}}}
\newcommand{\AND}{\quad\mbox{and}\quad}
\newcommand{\vect}[2]{\left(\begin{array}{c} #1_1 \\ \vdots \\
 #1_{#2}\end{array}\right)}
\newcommand{\mattwo}[4]{\left(\begin{array}{rr} #1 & #2\\ #3
&#4\end{array}\right)}
\newcommand{\mattwoc}[4]{\left(\begin{array}{cc} #1 & #2\\ #3
&#4\end{array}\right)}
\newcommand{\vectwo}[2]{\left(\begin{array}{r} #1 \\ #2\end{array}\right)}
\newcommand{\vectwoc}[2]{\left(\begin{array}{c} #1 \\ #2\end{array}\right)}

\newcommand{\ignore}[1]{}


\newcommand{\inv}{^{-1}}
\newcommand{\CC}{{\cal C}}
\newcommand{\CCone}{\CC^1}
\newcommand{\Span}{{\rm span}}
\newcommand{\rank}{{\rm rank}}
\newcommand{\trace}{{\rm tr}}
\newcommand{\RE}{{\rm Re}}
\newcommand{\IM}{{\rm Im}}
\newcommand{\nulls}{{\rm null\;space}}

\newcommand{\dps}{\displaystyle}
\newcommand{\arraystart}{\renewcommand{\arraystretch}{1.8}}
\newcommand{\arrayfinish}{\renewcommand{\arraystretch}{1.2}}
\newcommand{\Start}[1]{\vspace{0.08in}\noindent {\bf Section~\ref{#1}}}
\newcommand{\exer}[1]{\noindent {\bf \ref{#1}}}
\newcommand{\ans}{\textbf{Answer:} }
\newcommand{\matthree}[9]{\left(\begin{array}{rrr} #1 & #2 & #3 \\ #4 & #5 & #6
\\ #7 & #8 & #9\end{array}\right)}
\newcommand{\cvectwo}[2]{\left(\begin{array}{c} #1 \\ #2\end{array}\right)}
\newcommand{\cmatthree}[9]{\left(\begin{array}{ccc} #1 & #2 & #3 \\ #4 & #5 &
#6 \\ #7 & #8 & #9\end{array}\right)}
\newcommand{\vecthree}[3]{\left(\begin{array}{r} #1 \\ #2 \\
#3\end{array}\right)}
\newcommand{\cvecthree}[3]{\left(\begin{array}{c} #1 \\ #2 \\
#3\end{array}\right)}
\newcommand{\cmattwo}[4]{\left(\begin{array}{cc} #1 & #2\\ #3
&#4\end{array}\right)}

\newcommand{\Matrix}[1]{\ensuremath{\left(\begin{array}{rrrrrrrrrrrrrrrrrr} #1 \end{array}\right)}}

\newcommand{\Matrixc}[1]{\ensuremath{\left(\begin{array}{cccccccccccc} #1 \end{array}\right)}}



\renewcommand{\labelenumi}{\theenumi}
\newenvironment{enumeratea}%
{\begingroup
 \renewcommand{\theenumi}{\alph{enumi}}
 \renewcommand{\labelenumi}{(\theenumi)}
 \begin{enumerate}}
 {\end{enumerate}
 \endgroup}

\newcounter{help}
\renewcommand{\thehelp}{\thesection.\arabic{equation}}

%\newenvironment{equation*}%
%{\renewcommand\endequation{\eqno (\theequation)* $$}%
%   \begin{equation}}%
%   {\end{equation}\renewcommand\endequation{\eqno \@eqnnum
%$$\global\@ignoretrue}}

%\input{psfig.tex}

\author{Martin Golubitsky and Michael Dellnitz}

%\newenvironment{matlabEquation}%
%{\renewcommand\endequation{\eqno (\theequation*) $$}%
%   \begin{equation}}%
%   {\end{equation}\renewcommand\endequation{\eqno \@eqnnum
% $$\global\@ignoretrue}}

\newcommand{\soln}{\textbf{Solution:} }
\newcommand{\exercap}[1]{\centerline{Figure~\ref{#1}}}
\newcommand{\exercaptwo}[1]{\centerline{Figure~\ref{#1}a\hspace{2.1in}
Figure~\ref{#1}b}}
\newcommand{\exercapthree}[1]{\centerline{Figure~\ref{#1}a\hspace{1.2in}
Figure~\ref{#1}b\hspace{1.2in}Figure~\ref{#1}c}}
\newcommand{\para}{\hspace{0.4in}}

\usepackage{ifluatex}
\ifluatex
\ifcsname displaysolutions\endcsname%
\else
\renewenvironment{solution}{\suppress}{\endsuppress}
\fi
\else
\renewenvironment{solution}{}{}
\fi

\ifcsname answer\endcsname
\renewcommand{\answer}{}
\fi

%\ifxake
%\newenvironment{matlabEquation}{\begin{equation}}{\end{equation}}
%\else
\newenvironment{matlabEquation}%
{\let\oldtheequation\theequation\renewcommand{\theequation}{\oldtheequation*}\begin{equation}}%
  {\end{equation}\let\theequation\oldtheequation}
%\fi

\makeatother

\newcommand{\RED}[1]{{\color{red}{#1}}} 


\title{Linear Dependence and Linear Independence}

\begin{document}
\begin{abstract}
\end{abstract}
\maketitle

 \label{S:5.4}

An important question in linear algebra concerns finding spanning
sets for subspaces having the smallest
number of vectors. Let $w_1,\ldots,w_k$ be vectors in a vector
space $V$ and let $W=\Span\{w_1,\ldots,w_k\}$.  \index{span}
Suppose that $W$ is generated by a subset of these $k$ vectors.
Indeed, suppose that the $k^{th}$ vector is redundant in the
sense that $W=\Span\{w_1,\ldots,w_{k-1}\}$.  Since $w_k\in W$,
this is possible only if $w_k$ is a linear combination of the
$k-1$ vectors $w_1,\ldots,w_{k-1}$; that is, only if
\begin{equation}  \label{e:depend}
w_k = r_1w_1 + \cdots + r_{k-1}w_{k-1}.
\end{equation}
\begin{definition}  \label{lineardependence}
Let $w_1,\ldots,w_k$ be vectors in the vector space $V$.  The set
$\{w_1,\ldots,w_k\}$ is {\em linearly dependent\/} if one of the vectors
$w_j$ can be written as a linear combination of the remaining $k-1$ vectors.
\end{definition} \index{linearly!dependent} \index{linear!combination}
Note that when $k=1$, the phrase `$\{w_1\}$ is linearly dependent'
means that $w_1=0$.

If we set $r_k=-1$, then we may rewrite \eqref{e:depend} as
\[
r_1w_1 + \cdots + r_{k-1}w_{k-1} + r_k w_k =0.
\]
It follows that:
\begin{lemma}  \label{L:lindep}
The set of vectors $\{w_1,\ldots,w_k\}$ is linearly dependent if and
only if there exist scalars $r_1,\ldots,r_k$ such that
\begin{itemize}
\item[(a)]   at least one of the $r_j$ is nonzero, and
\item[(b)]   $r_1w_1 + \cdots + r_k w_k =0.$
\end{itemize}
\end{lemma}

For example, the vectors $w_1=(2,4,7)$, $w_2=(5,1,-1)$, and
$w_3=(1,-7,-15)$ are linearly dependent since $2w_1-w_2+w_3=0$.

\begin{definition}  \label{linearindependence}
A set of $k$ vectors $\{w_1,\ldots,w_k\}$ is {\em linearly
independent\/} if none of the $k$ vectors can be written as a
linear combination of the other $k-1$ vectors.
\end{definition} \index{linearly!independent}

Since linear independence means {\em not\/} linearly dependent,
Lemma~\ref{L:lindep} can be rewritten as:
\begin{lemma}  \label{L:linindep}
The set of vectors $\{w_1,\ldots,w_k\}$ is linearly independent if and
only if whenever
\[
r_1w_1 + \cdots + r_kw_k = 0,
\]
it follows that
\[
r_1 = r_2 = \cdots = r_k = 0.
\]
\end{lemma}

Let $e_j$ be the vector in $\R^n$ whose $j^{th}$ component is $1$
and all of whose other components are $0$. The set of vectors
$e_1,\ldots,e_n$ is the simplest example of a set of linearly
independent vectors in $\R^n$.  We use Lemma~\ref{L:linindep} to
verify independence by supposing that
\[
r_1e_1 + \cdots + r_ne_n = 0.
\]
A calculation shows that
\[
0 = r_1e_1 + \cdots + r_ne_n = (r_1,\ldots,r_n).
\]
It follows that each $r_j$ equals $0$, and the vectors
$e_1,\ldots,e_n$ are linearly independent.


\subsection*{Deciding Linear Dependence and Linear Independence}
\index{linearly!dependent}

Deciding whether a set of $k$ vectors in $\R^n$ is linearly
dependent or linearly independent is equivalent to solving a
system of linear equations.  Let $w_1,\ldots,w_k$ be vectors
in $\R^n$, and view these vectors as column vectors. Let
\begin{equation}  \label{E:Ank}
A=(w_1|\cdots|w_k)
\end{equation}
be the $n\times k$ matrix whose columns are the vectors $w_j$.
Then a vector
\[
R = \vect{r}{k}
\]
is a solution to the system of equations $AR=0$ precisely when
\begin{equation}
r_1w_1 + \cdots + r_kw_k = 0.
\end{equation}
If there is a nonzero solution $R$ to $AR=0$, then the vectors
$\{w_1,\ldots,w_k\}$ are linearly dependent; if the only solution
to $AR=0$ is $R=0$, then the vectors are linearly independent.

The preceding discussion is summarized by:
\begin{lemma}
The vectors $w_1,\ldots,w_k$ in $\R^n$ are linearly dependent if the
null space of the $n\times k$ matrix $A$ defined in \eqref{E:Ank} is
nonzero and linearly independent if the null space of $A$ is zero.
\end{lemma} \index{linearly!dependent}\index{linearly!independent}
\index{null space}


\subsubsection*{A Simple Example of Linear Independence with Two Vectors}

The two vectors
\[
w_1 = \left(\begin{array}{r} 2 \\ -8\\ 1\\ 0 \end{array}\right)
\AND
w_2 = \left(\begin{array}{r} 1 \\ -2\\ 0\\ 1 \end{array}\right)
\]
are linearly independent.  To see this suppose that
$r_1 w_1 + r_2 w_2 = 0$.  Using the components of $w_1$ and $w_2$
this equality is equivalent to the system of four equations
\[
2r_1 + r_2 = 0,\quad -8r_1 - 2r_2 = 0,\quad r_1 = 0, \AND r_2 = 0.
\]
In particular, $r_1 = r_2 = 0$; hence $w_1$ and $w_2$ are
linearly independent.


\subsubsection*{Using \Matlab to Decide Linear Dependence}

Suppose that we want to determine whether or not the vectors
\begin{matlabEquation}\label{MATLAB:66}
w_1 = \left(\begin{array}{r} 1 \\ 2 \\ -1 \\ 3 \\ 5 \end{array}\right)
\quad
w_2 = \left(\begin{array}{r} -1 \\ 1 \\ 4 \\ -2 \\ 0 \end{array}\right)
\end{matlabEquation}%
\begin{equation*}
w_3 = \left(\begin{array}{r} 1 \\ 1 \\ -1 \\ 3 \\ 12 \end{array}\right)
\quad
w_4 = \left(\begin{array}{r} 0 \\ 4 \\ 3 \\ 1 \\ -2 \end{array}\right)
\end{equation*}
are linearly dependent.  After typing {\tt e5\_4\_4} in \Matlabp, form
the $5\times 4$ matrix $A$ by typing
\begin{verbatim}
A = [w1 w2 w3 w4]
\end{verbatim}
Determine whether there is a nonzero solution to $AR=0$ by typing
\begin{verbatim}
null(A)
\end{verbatim} \index{\computer!null}
The response from \Matlab is
\begin{verbatim}
ans =
   -0.7559
   -0.3780
    0.3780
    0.3780
\end{verbatim}
showing that there is a nonzero solution to $AR=0$ and the vectors
$w_j$ are linearly dependent.  Indeed, this solution for $R$ shows
that we can solve for $w_1$ in terms of $w_2,w_3,w_4$.
We can now ask whether or not $w_2,w_3,w_4$ are linearly dependent.
To answer this question form the matrix
\begin{verbatim}
B = [w2 w3 w4]
\end{verbatim}
and type {\tt null(B)} to obtain
\begin{verbatim}
ans =
   3×0 empty double matrix
\end{verbatim}
showing that the only solution to $BR=0$ is the zero solution $R=0$.
Thus, $w_2,w_3,w_4$ are linearly independent.  For these particular
vectors, any three of the four are linearly independent.



\includeexercises

\end{document}
