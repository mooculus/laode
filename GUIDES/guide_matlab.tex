\documentclass{article}
\usepackage{hyperref}
\usepackage{fullpage}
\usepackage{enumerate}
\usepackage{upquote}

\setlength{\parindent}{0in}
\setlength{\parskip}{1ex}


\title{Guide: MATLAB and Course \texttt{.m}-files}
\author{Jim Fowler, Marty Golubitsky, and Brad Findell}

\usepackage{xcolor,verbatimbox}
\catcode`>=\active %
\catcode`<=\active %
\def\openesc{\color{red}}
\def\closeesc{\color{black}}
\def\vbdelim{\catcode`<=\active\catcode`>=\active%
\def<{\openesc}
\def>{\closeesc}}
\catcode`>=12 %
\catcode`<=12 %

\begin{document}

\maketitle

\section{MATLAB}\label{for-ohio-state-faculty-and-students}

Computers not only enable the completion of complicated calculations,
but they also improve the conceptual understanding of
mathematics. This course relies on MATLAB, a popular software package
for matrix computations. MATLAB is available free-of-charge to Ohio
State students and faculty.

Choose at least one of the following \textbf{two options}:

\subsection{Install the MATLAB Application on Your Computer}

To download and activate your free copy of MATLAB, please follow the
steps below.  For support, faculty should email
\href{mailto:support@math.osu.edu}{\texttt{support@math.osu.edu}} and
students should email
\href{mailto:8help@osu.edu}{\texttt{8help@osu.edu}}.

\begin{enumerate}
\def\labelenumi{\arabic{enumi})}
\item Launch \href{https://osuitsm.service-now.com/selfservice/}{Ohio
    State's IT Self-Service} and click ``Sign into my account'' at the upper right.
\item From the menu on the left, click Order Services. (Despite the shopping cart icon,
  MATLAB will be free.)
\item Press the ''Software Services'' button.  
\item Click the link ``Site Licensed Software Request.''
\item Complete the Software Request Form, choosing MATLAB. Agree to
  terms and conditions.
\item Click Check Out.
\item Click Submit Order at the lower right.
\item Your order confirmation page lists your selections and their
  costs, as well as provides a Request Number (i.e., REQ12345) for
  tracking purposes.
\end{enumerate}

The remaining instructions for installing MATLAB will be emailed to
your Ohio State email account. Instructions are OS dependent.

\subsection{Register for the Cloud-Hosted Version of MATLAB}
Anyone with an OSU e-mail account may use a cloud-hosted version of MATLAB, either in a Web browser on a computer or via the MATLAB Mobile app, if using a mobile device. 

Here is how to get started: 

\begin{enumerate}
\def\labelenumi{\arabic{enumi})}
\item Go to  \verb|http://www.mathworks.com|.
\item Click on the person icon in the upper right corner, for ``Sign in''.
\item Click on ``Create Account''.
\item Enter your OSU e-mail address, fill out the form, and press ``Create''.
\item Respond to the e-mail to verify your e-mail address.
\item For the Web browser version (on a computer): 
\begin{enumerate}
\item Go to \verb|https://matlab.mathworks.com/| and sign in.
\item You will arrive at a webpage organized just like the MATLAB application.
\end{enumerate}
\item For MATLAB Mobile: 
\begin{enumerate}
\item Go to the App Store or Google Play, search for MATLAB Mobile, and install the app.  
\item Open the app, sign in, and your screen will be organized like a simplified version of the MATLAB application.
\end{enumerate}
\end{enumerate}

\section{Course \texttt{.m}-files}

The course files~\verb|*.m| can be downloaded and installed either on your computer or in your cloud-hosted drived.
  
The MATLAB commands below store these files in a directory called
  \verb|m_files| and then add that directory to those that MATLAB searches for commands.
  
  To achieve this, start MATLAB and enter the following
  commands.

\begin{verbatim}  
cd(userpath)
websave('mfiles.zip','http://go.osu.edu/mfiles')
mkdir('m_files')
unzip('mfiles.zip','m_files')
addpath('m_files')
delete('mfiles.zip')
\end{verbatim}

Enter \verb|pplane9| into the MATLAB command window and
click on ``Proceed'' in the \verb|pplane9| setup window.  You should
see a vector field.

\end{document}
