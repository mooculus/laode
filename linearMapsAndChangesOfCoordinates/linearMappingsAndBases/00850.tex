\documentclass{ximera}

\author{Matthew Carr \& Marty Golubitsky}

\usepackage{epsfig}

\graphicspath{
  {./}
  {figures/}
}


\usepackage{morewrites}

%\newcounter{ccounter}
%\setcounter{ccounter}{1}
%\newcommand{\Chapter}[1]{\setcounter{chapter}{\arabic{ccounter}}\chapter{#1}\addtocounter{ccounter}{1}}

%\newcommand{\section}[1]{\section{#1}\setcounter{thm}{0}\setcounter{equation}{0}}

%\renewcommand{\theequation}{\arabic{chapter}.\arabic{section}.\arabic{equation}}
%\renewcommand{\thefigure}{\arabic{chapter}.\arabic{figure}}
%\renewcommand{\thetable}{\arabic{chapter}.\arabic{table}}

%\newcommand{\Sec}[2]{\section{#1}\markright{\arabic{ccounter}.\arabic{section}.#2}\setcounter{equation}{0}\setcounter{thm}{0}\setcounter{figure}{0}}

\newcommand{\Sec}[2]{\section{#1}}

\setcounter{secnumdepth}{2}
%\setcounter{secnumdepth}{1} 

%\newcounter{THM}
%\renewcommand{\theTHM}{\arabic{chapter}.\arabic{section}}

\newcommand{\trademark}{{R\!\!\!\!\!\bigcirc}}
%\newtheorem{exercise}{}

\newcommand{\dfield}{{\sf dfield9}}
\newcommand{\pplane}{{\sf pplane9}}

\newcommand{\EXER}{\section*{Exercises}}%\vspace*{0.2in}\hrule\small\setcounter{exercise}{0}}
\newcommand{\CEXER}{}%\vspace{0.08in}\begin{center}Computer Exercises\end{center}}
\newcommand{\TEXER}{} %\vspace{0.08in}\begin{center}Hand Exercises\end{center}}
\newcommand{\AEXER}{} %\vspace{0.08in}\begin{center}Hand Exercises\end{center}}

% BADBAD: \newcommand{\Bbb}{\bf}

\newcommand{\R}{\mbox{$\Bbb{R}$}}
\newcommand{\C}{\mbox{$\Bbb{C}$}}
\newcommand{\Z}{\mbox{$\Bbb{Z}$}}
\newcommand{\N}{\mbox{$\Bbb{N}$}}
\newcommand{\D}{\mbox{{\bf D}}}
\usepackage{amssymb}
%\newcommand{\qed}{\hfill\mbox{\raggedright$\square$} \vspace{1ex}}
%\newcommand{\proof}{\noindent {\bf Proof:} \hspace{0.1in}}

\newcommand{\setmin}{\;\mbox{--}\;}
\newcommand{\Matlab}{{M\small{AT\-LAB}} }
\newcommand{\Matlabp}{{M\small{AT\-LAB}}}
\newcommand{\computer}{\Matlab Instructions}
\newcommand{\half}{\mbox{$\frac{1}{2}$}}
\newcommand{\compose}{\raisebox{.15ex}{\mbox{{\scriptsize$\circ$}}}}
\newcommand{\AND}{\quad\mbox{and}\quad}
\newcommand{\vect}[2]{\left(\begin{array}{c} #1_1 \\ \vdots \\
 #1_{#2}\end{array}\right)}
\newcommand{\mattwo}[4]{\left(\begin{array}{rr} #1 & #2\\ #3
&#4\end{array}\right)}
\newcommand{\mattwoc}[4]{\left(\begin{array}{cc} #1 & #2\\ #3
&#4\end{array}\right)}
\newcommand{\vectwo}[2]{\left(\begin{array}{r} #1 \\ #2\end{array}\right)}
\newcommand{\vectwoc}[2]{\left(\begin{array}{c} #1 \\ #2\end{array}\right)}



\newcommand{\inv}{^{-1}}
\newcommand{\CC}{{\cal C}}
\newcommand{\CCone}{\CC^1}
\newcommand{\Span}{{\rm span}}
\newcommand{\rank}{{\rm rank}}
\newcommand{\trace}{{\rm tr}}
\newcommand{\RE}{{\rm Re}}
\newcommand{\IM}{{\rm Im}}
\newcommand{\nulls}{{\rm null\;space}}

\newcommand{\dps}{\displaystyle}
\newcommand{\arraystart}{\renewcommand{\arraystretch}{1.8}}
\newcommand{\arrayfinish}{\renewcommand{\arraystretch}{1.2}}
\newcommand{\Start}[1]{\vspace{0.08in}\noindent {\bf Section~\ref{#1}}}
\newcommand{\exer}[1]{\noindent {\bf \ref{#1}}}
\newcommand{\ans}{}
\newcommand{\matthree}[9]{\left(\begin{array}{rrr} #1 & #2 & #3 \\ #4 & #5 & #6
\\ #7 & #8 & #9\end{array}\right)}
\newcommand{\cvectwo}[2]{\left(\begin{array}{c} #1 \\ #2\end{array}\right)}
\newcommand{\cmatthree}[9]{\left(\begin{array}{ccc} #1 & #2 & #3 \\ #4 & #5 &
#6 \\ #7 & #8 & #9\end{array}\right)}
\newcommand{\vecthree}[3]{\left(\begin{array}{r} #1 \\ #2 \\
#3\end{array}\right)}
\newcommand{\cvecthree}[3]{\left(\begin{array}{c} #1 \\ #2 \\
#3\end{array}\right)}
\newcommand{\cmattwo}[4]{\left(\begin{array}{cc} #1 & #2\\ #3
&#4\end{array}\right)}

\newcommand{\Matrix}[1]{\ensuremath{\left(\begin{array}{rrrrrrrrrrrrrrrrrr} #1 \end{array}\right)}}

\newcommand{\Matrixc}[1]{\ensuremath{\left(\begin{array}{cccccccccccc} #1 \end{array}\right)}}



\renewcommand{\labelenumi}{\theenumi)}
\newenvironment{enumeratea}%
{\begingroup
 \renewcommand{\theenumi}{\alph{enumi}}
 \renewcommand{\labelenumi}{(\theenumi)}
 \begin{enumerate}}
 {\end{enumerate}\endgroup}



\newcounter{help}
\renewcommand{\thehelp}{\thesection.\arabic{equation}}

%\newenvironment{equation*}%
%{\renewcommand\endequation{\eqno (\theequation)* $$}%
%   \begin{equation}}%
%   {\end{equation}\renewcommand\endequation{\eqno \@eqnnum
%$$\global\@ignoretrue}}

%\input{psfig.tex}

\author{Martin Golubitsky and Michael Dellnitz}

%\newenvironment{matlabEquation}%
%{\renewcommand\endequation{\eqno (\theequation*) $$}%
%   \begin{equation}}%
%   {\end{equation}\renewcommand\endequation{\eqno \@eqnnum
% $$\global\@ignoretrue}}

\newcommand{\soln}{\textbf{Solution:} }
\newcommand{\exercap}[1]{\centerline{Figure~\ref{#1}}}
\newcommand{\exercaptwo}[1]{\centerline{Figure~\ref{#1}a\hspace{2.1in}
Figure~\ref{#1}b}}
\newcommand{\exercapthree}[1]{\centerline{Figure~\ref{#1}a\hspace{1.2in}
Figure~\ref{#1}b\hspace{1.2in}Figure~\ref{#1}c}}
\newcommand{\para}{\hspace{0.4in}}

\renewenvironment{solution}{\suppress}{\endsuppress}

\ifxake
\newenvironment{matlabEquation}{\begin{equation}}{\end{equation}}
\else
\newenvironment{matlabEquation}%
{\let\oldtheequation\theequation\renewcommand{\theequation}{\oldtheequation*}\begin{equation}}%
  {\end{equation}\let\theequation\oldtheequation}
\fi

\makeatother


\begin{document}

% 8.1 Linear Mappings and Bases

\begin{exercise}\label{mc.exercise12}
Which of the following are True and which False.  Give reasons for your answer.
\begin{enumerate}
\item For any $n \times n$ matrix $A$, $\det(A)$ is the product of its $n$ eigenvalues.
\item Similar matrices always have the same eigenvectors.
\item For any $n \times n$ matrix $A$ and scalar $k\in\R$, $\det(kA) = k^n \det(A)$.
\item There is a linear map $L: \R^3 \to \R^2$ such that 
\[
L(1,2,3) = (0,1) \AND L(2,4,6) = (1,1).
\]
\item The only rank $0$ matrix is the zero matrix.
\end{enumerate}

\begin{solution}

\ans 
True statements are circled and False statements are crossed out.


\begin{enumerate}

\item \fbox{For any $n \times n$ matrix $A$, $\det(A)$ is the product of its $n$ eigenvalues.}
\item \sout{Similar matrices always have the same eigenvectors}.
\item \fbox{For any $n \times n$ matrix $A$ and scalar $k\in\R$, $\det(kA) = k^n \det(A)$.}
\item \sout{There is a linear map $L: \R^3 \to \R^2$ such that $L(1,2,3) = (0,1)$ and $L(2,4,6) = (1,1)$}.
\item \fbox{The only rank $0$ matrix is the zero matrix.}
\end{enumerate}

\soln \begin{enumerate}

\item This follows from Theorem~\ref{T:eigens}(a).

%The characteristic polynomial of $A$ is computed as the determinant $p(\lambda)=\det(A-\lambda I)$, so when $\lambda=0$, we recover $\det(A)$. It is always the case that $p(\lambda)$ has leading coefficient $(-1)^n$, so the fundamental theorem of algebra guarantees that the polynomial $\det(A-\lambda I)$ factors as $(-1)^n (\lambda-\lambda_1)\cdot \cdots\cdot (\lambda-\lambda_n)$ where the $\lambda_i$ are the roots of $c(\lambda)$. So when $\lambda=0$, we recover 
%\[
%\det(A)=(-1)^n (-\lambda_1)(-\lambda_2)\cdots(-\lambda_n)=(-1)^n\left(-\lambda_1\cdot-\lambda_2\cdots-\lambda_n\right) =(-1)^n (-1)^n(\lambda_1\c=\prod_{i=1}^{n}\lambda_i\ldotp
%\]
\item Suppose matrices $A$ and $B$ are similar. Then, by definition, there exists an invertible matrix $P$ such that $A=P^{-1}BP$. Let $B=\Matrix{ 0 & 1 \\ 1 & 0}$ and $P=\Matrix{ 1 & 1 \\ 0 & 1}$. Then $P^{-1}=\Matrix{ 1 & -1 \\ 0 & 1}$ and $A=P^{-1}BP=\Matrix{ -1 & 0 \\ 1 & 1}$. $B$ has eigenvectors $(1,1)$ and $(1,-1)$. $A$ has eigenvectors $(0,1)$ and $(2,-1)$. These are not the same. 

\item Use properties of the determinant to compute
\[
\det(kA) = \det (kI_n A) = \det (kI_n)\det (A)=k^n \det A.
\]

\item Since the vector $(2,4,6)=2(1,2,3)$, linearity forces $L(2,4,6)=2L(1,2,3)$. But $2L(1,2,3)=(0,2)\ne (1,1)$. So there exists no such function.

\item The rank of a matrix $A$ is the number of linearly independent rows. Hence, if a row of $A$ is nonzero, then the $\rank(A)\geq 1$.  Therefore, $\rank(A) = 0$ implies all rows of $A$ equal $0$ and $A=0$.
\end{enumerate}



  
\end{solution}
\end{exercise}

\end{document}
