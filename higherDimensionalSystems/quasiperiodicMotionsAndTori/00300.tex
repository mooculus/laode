\documentclass{ximera}
\usepackage{epsfig}

\graphicspath{
  {./}
  {figures/}
}


\usepackage{morewrites}

%\newcounter{ccounter}
%\setcounter{ccounter}{1}
%\newcommand{\Chapter}[1]{\setcounter{chapter}{\arabic{ccounter}}\chapter{#1}\addtocounter{ccounter}{1}}

%\newcommand{\section}[1]{\section{#1}\setcounter{thm}{0}\setcounter{equation}{0}}

%\renewcommand{\theequation}{\arabic{chapter}.\arabic{section}.\arabic{equation}}
%\renewcommand{\thefigure}{\arabic{chapter}.\arabic{figure}}
%\renewcommand{\thetable}{\arabic{chapter}.\arabic{table}}

%\newcommand{\Sec}[2]{\section{#1}\markright{\arabic{ccounter}.\arabic{section}.#2}\setcounter{equation}{0}\setcounter{thm}{0}\setcounter{figure}{0}}

\newcommand{\Sec}[2]{\section{#1}}

\setcounter{secnumdepth}{2}
%\setcounter{secnumdepth}{1} 

%\newcounter{THM}
%\renewcommand{\theTHM}{\arabic{chapter}.\arabic{section}}

\newcommand{\trademark}{{R\!\!\!\!\!\bigcirc}}
%\newtheorem{exercise}{}

\newcommand{\dfield}{{\sf dfield9}}
\newcommand{\pplane}{{\sf pplane9}}

\newcommand{\EXER}{\section*{Exercises}}%\vspace*{0.2in}\hrule\small\setcounter{exercise}{0}}
\newcommand{\CEXER}{}%\vspace{0.08in}\begin{center}Computer Exercises\end{center}}
\newcommand{\TEXER}{} %\vspace{0.08in}\begin{center}Hand Exercises\end{center}}
\newcommand{\AEXER}{} %\vspace{0.08in}\begin{center}Hand Exercises\end{center}}

% BADBAD: \newcommand{\Bbb}{\bf}

\newcommand{\R}{\mbox{$\Bbb{R}$}}
\newcommand{\C}{\mbox{$\Bbb{C}$}}
\newcommand{\Z}{\mbox{$\Bbb{Z}$}}
\newcommand{\N}{\mbox{$\Bbb{N}$}}
\newcommand{\D}{\mbox{{\bf D}}}
\usepackage{amssymb}
%\newcommand{\qed}{\hfill\mbox{\raggedright$\square$} \vspace{1ex}}
%\newcommand{\proof}{\noindent {\bf Proof:} \hspace{0.1in}}

\newcommand{\setmin}{\;\mbox{--}\;}
\newcommand{\Matlab}{{M\small{AT\-LAB}} }
\newcommand{\Matlabp}{{M\small{AT\-LAB}}}
\newcommand{\computer}{\Matlab Instructions}
\newcommand{\half}{\mbox{$\frac{1}{2}$}}
\newcommand{\compose}{\raisebox{.15ex}{\mbox{{\scriptsize$\circ$}}}}
\newcommand{\AND}{\quad\mbox{and}\quad}
\newcommand{\vect}[2]{\left(\begin{array}{c} #1_1 \\ \vdots \\
 #1_{#2}\end{array}\right)}
\newcommand{\mattwo}[4]{\left(\begin{array}{rr} #1 & #2\\ #3
&#4\end{array}\right)}
\newcommand{\mattwoc}[4]{\left(\begin{array}{cc} #1 & #2\\ #3
&#4\end{array}\right)}
\newcommand{\vectwo}[2]{\left(\begin{array}{r} #1 \\ #2\end{array}\right)}
\newcommand{\vectwoc}[2]{\left(\begin{array}{c} #1 \\ #2\end{array}\right)}



\newcommand{\inv}{^{-1}}
\newcommand{\CC}{{\cal C}}
\newcommand{\CCone}{\CC^1}
\newcommand{\Span}{{\rm span}}
\newcommand{\rank}{{\rm rank}}
\newcommand{\trace}{{\rm tr}}
\newcommand{\RE}{{\rm Re}}
\newcommand{\IM}{{\rm Im}}
\newcommand{\nulls}{{\rm null\;space}}

\newcommand{\dps}{\displaystyle}
\newcommand{\arraystart}{\renewcommand{\arraystretch}{1.8}}
\newcommand{\arrayfinish}{\renewcommand{\arraystretch}{1.2}}
\newcommand{\Start}[1]{\vspace{0.08in}\noindent {\bf Section~\ref{#1}}}
\newcommand{\exer}[1]{\noindent {\bf \ref{#1}}}
\newcommand{\ans}{}
\newcommand{\matthree}[9]{\left(\begin{array}{rrr} #1 & #2 & #3 \\ #4 & #5 & #6
\\ #7 & #8 & #9\end{array}\right)}
\newcommand{\cvectwo}[2]{\left(\begin{array}{c} #1 \\ #2\end{array}\right)}
\newcommand{\cmatthree}[9]{\left(\begin{array}{ccc} #1 & #2 & #3 \\ #4 & #5 &
#6 \\ #7 & #8 & #9\end{array}\right)}
\newcommand{\vecthree}[3]{\left(\begin{array}{r} #1 \\ #2 \\
#3\end{array}\right)}
\newcommand{\cvecthree}[3]{\left(\begin{array}{c} #1 \\ #2 \\
#3\end{array}\right)}
\newcommand{\cmattwo}[4]{\left(\begin{array}{cc} #1 & #2\\ #3
&#4\end{array}\right)}

\newcommand{\Matrix}[1]{\ensuremath{\left(\begin{array}{rrrrrrrrrrrrrrrrrr} #1 \end{array}\right)}}

\newcommand{\Matrixc}[1]{\ensuremath{\left(\begin{array}{cccccccccccc} #1 \end{array}\right)}}



\renewcommand{\labelenumi}{\theenumi)}
\newenvironment{enumeratea}%
{\begingroup
 \renewcommand{\theenumi}{\alph{enumi}}
 \renewcommand{\labelenumi}{(\theenumi)}
 \begin{enumerate}}
 {\end{enumerate}\endgroup}



\newcounter{help}
\renewcommand{\thehelp}{\thesection.\arabic{equation}}

%\newenvironment{equation*}%
%{\renewcommand\endequation{\eqno (\theequation)* $$}%
%   \begin{equation}}%
%   {\end{equation}\renewcommand\endequation{\eqno \@eqnnum
%$$\global\@ignoretrue}}

%\input{psfig.tex}

\author{Martin Golubitsky and Michael Dellnitz}

%\newenvironment{matlabEquation}%
%{\renewcommand\endequation{\eqno (\theequation*) $$}%
%   \begin{equation}}%
%   {\end{equation}\renewcommand\endequation{\eqno \@eqnnum
% $$\global\@ignoretrue}}

\newcommand{\soln}{\textbf{Solution:} }
\newcommand{\exercap}[1]{\centerline{Figure~\ref{#1}}}
\newcommand{\exercaptwo}[1]{\centerline{Figure~\ref{#1}a\hspace{2.1in}
Figure~\ref{#1}b}}
\newcommand{\exercapthree}[1]{\centerline{Figure~\ref{#1}a\hspace{1.2in}
Figure~\ref{#1}b\hspace{1.2in}Figure~\ref{#1}c}}
\newcommand{\para}{\hspace{0.4in}}

\renewenvironment{solution}{\suppress}{\endsuppress}

\ifxake
\newenvironment{matlabEquation}{\begin{equation}}{\end{equation}}
\else
\newenvironment{matlabEquation}%
{\let\oldtheequation\theequation\renewcommand{\theequation}{\oldtheequation*}\begin{equation}}%
  {\end{equation}\let\theequation\oldtheequation}
\fi

\makeatother

\begin{document}

\noindent In Exercises~\ref{c14.5.2a} -- \ref{c14.5.2d} use \Matlab to find
out whether the system of differential equations $\dot X= AX$ for
the given matrix $A$ has quasiperiodic solutions.
\begin{exercise} \label{c14.5.2a}
\begin{matlabEquation}\label{MATLAB:57}
A=\left(\begin{array}{rrrr}
    4.9666  &  2.2833  &  0.8000  &  5.3666\\
   -0.9889  & -0.0944  & -1.6000  & -1.7889\\
    2.9889  &  4.0944  & -0.4000  &  1.7889\\
   -8.9443  & -4.4721  &       0  & -4.4721
\end{array}\right).
\end{matlabEquation}

\begin{solution}
\ans Yes.

\soln The eigenvalues of $A$ are 
\begin{verbatim}
   0.0000 + 4.4722i
   0.0000 - 4.4722i
   0.0000 + 2.0000i
   0.0000 - 2.0000i
\end{verbatim}
Having two pairs of purely imaginary complex conjugate eigenvalues leads to
two-frequency motions.  These motions will be periodic if the quotient of the 
frequencies is irrational.  That question cannot be decided by numerical
computation.


\end{solution}
\end{exercise}

\begin{exercise} \label{c14.5.2b}
\begin{matlabEquation}\label{MATLAB:58}
A=\left(\begin{array}{rrrr}
    2.7666  &  0.2833  &  1.4000  &  4.5666\\
    2.4111  &  1.9056  & -1.8000  & -0.1889\\
   -0.4111  &  2.0944  & -0.2000  &  0.1889\\
   -7.9443  & -2.4721  & -1.0000  & -4.4721
\end{array}\right).
\end{matlabEquation}

\begin{solution}
\ans No.

\soln The eigenvalues of $A$ are 
\begin{verbatim}
  -0.5104 + 4.5851i
  -0.5104 - 4.5851i
   0.5104 + 1.8703i
   0.5104 - 1.8703i
\end{verbatim}
The eigenvalues are not purely imaginary.

\end{solution}
\end{exercise}

\begin{exercise} \label{c14.5.2c}
\begin{matlabEquation}\label{MATLAB:59}
A=\left(\begin{array}{rrrrr}
    0.7130  & 24.8184  & 32.0740  &  2.8959  & 15.3610\\
    0.0552  & 17.0732  & 21.1395  &  6.6120  & 11.0843\\
   -0.6168  &-12.0764  &-16.0165  & -1.6151  & -7.3997\\
    1.0410  & -3.3312  & -2.0820  & -3.3312  & -3.1230\\
   -0.5205  &  1.6656  &  1.0410  &  1.6656  &  1.5615
\end{array}\right).
\end{matlabEquation}

\begin{solution}
\ans No.

\soln The eigenvalues of $A$ are 
\begin{verbatim}
   0.0003 + 3.1233i
   0.0003 - 3.1233i
  -0.0003 + 1.0005i
  -0.0003 - 1.0005i
        0          
\end{verbatim}
The eigenvalues are not purely imaginary.

\end{solution}
\end{exercise}

\begin{exercise} \label{c14.5.2d}
\begin{matlabEquation}\label{MATLAB:60}
A=\left(\begin{array}{rrrrr}
  -10.2870 &  40.8184 &  30.0740 &  19.8959 &  38.3610\\
   -7.4448 &  25.0732 &  16.1395 &  15.8620 &  26.0843\\
    6.8833 & -20.0764 & -11.0165 & -10.3652 & -21.3998\\
    1.0410 &  -3.3312 &  -2.0820 &  -3.3312 &  -3.1230\\
   -0.5205 &   1.6656 &   1.0410 &   1.1656 &   0.5615
\end{array}\right).
\end{matlabEquation}

\begin{solution}
\ans No.

\soln The eigenvalues of $A$ are 
\begin{verbatim}
  -0.9999 + 2.9587i
  -0.9999 - 2.9587i
   0.2677          
   3.7321          
  -1.0000          
\end{verbatim}
The eigenvalues are not purely imaginary.




\end{solution}
\end{exercise}
\end{document}
