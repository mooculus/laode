\documentclass{ximera}
\usepackage{epsfig}

\graphicspath{
  {./}
  {figures/}
}


\usepackage{morewrites}

%\newcounter{ccounter}
%\setcounter{ccounter}{1}
%\newcommand{\Chapter}[1]{\setcounter{chapter}{\arabic{ccounter}}\chapter{#1}\addtocounter{ccounter}{1}}

%\newcommand{\section}[1]{\section{#1}\setcounter{thm}{0}\setcounter{equation}{0}}

%\renewcommand{\theequation}{\arabic{chapter}.\arabic{section}.\arabic{equation}}
%\renewcommand{\thefigure}{\arabic{chapter}.\arabic{figure}}
%\renewcommand{\thetable}{\arabic{chapter}.\arabic{table}}

%\newcommand{\Sec}[2]{\section{#1}\markright{\arabic{ccounter}.\arabic{section}.#2}\setcounter{equation}{0}\setcounter{thm}{0}\setcounter{figure}{0}}

\newcommand{\Sec}[2]{\section{#1}}

\setcounter{secnumdepth}{2}
%\setcounter{secnumdepth}{1} 

%\newcounter{THM}
%\renewcommand{\theTHM}{\arabic{chapter}.\arabic{section}}

\newcommand{\trademark}{{R\!\!\!\!\!\bigcirc}}
%\newtheorem{exercise}{}

\newcommand{\dfield}{{\sf dfield9}}
\newcommand{\pplane}{{\sf pplane9}}

\newcommand{\EXER}{\section*{Exercises}}%\vspace*{0.2in}\hrule\small\setcounter{exercise}{0}}
\newcommand{\CEXER}{}%\vspace{0.08in}\begin{center}Computer Exercises\end{center}}
\newcommand{\TEXER}{} %\vspace{0.08in}\begin{center}Hand Exercises\end{center}}
\newcommand{\AEXER}{} %\vspace{0.08in}\begin{center}Hand Exercises\end{center}}

% BADBAD: \newcommand{\Bbb}{\bf}

\newcommand{\R}{\mbox{$\Bbb{R}$}}
\newcommand{\C}{\mbox{$\Bbb{C}$}}
\newcommand{\Z}{\mbox{$\Bbb{Z}$}}
\newcommand{\N}{\mbox{$\Bbb{N}$}}
\newcommand{\D}{\mbox{{\bf D}}}
\usepackage{amssymb}
%\newcommand{\qed}{\hfill\mbox{\raggedright$\square$} \vspace{1ex}}
%\newcommand{\proof}{\noindent {\bf Proof:} \hspace{0.1in}}

\newcommand{\setmin}{\;\mbox{--}\;}
\newcommand{\Matlab}{{M\small{AT\-LAB}} }
\newcommand{\Matlabp}{{M\small{AT\-LAB}}}
\newcommand{\computer}{\Matlab Instructions}
\newcommand{\half}{\mbox{$\frac{1}{2}$}}
\newcommand{\compose}{\raisebox{.15ex}{\mbox{{\scriptsize$\circ$}}}}
\newcommand{\AND}{\quad\mbox{and}\quad}
\newcommand{\vect}[2]{\left(\begin{array}{c} #1_1 \\ \vdots \\
 #1_{#2}\end{array}\right)}
\newcommand{\mattwo}[4]{\left(\begin{array}{rr} #1 & #2\\ #3
&#4\end{array}\right)}
\newcommand{\mattwoc}[4]{\left(\begin{array}{cc} #1 & #2\\ #3
&#4\end{array}\right)}
\newcommand{\vectwo}[2]{\left(\begin{array}{r} #1 \\ #2\end{array}\right)}
\newcommand{\vectwoc}[2]{\left(\begin{array}{c} #1 \\ #2\end{array}\right)}



\newcommand{\inv}{^{-1}}
\newcommand{\CC}{{\cal C}}
\newcommand{\CCone}{\CC^1}
\newcommand{\Span}{{\rm span}}
\newcommand{\rank}{{\rm rank}}
\newcommand{\trace}{{\rm tr}}
\newcommand{\RE}{{\rm Re}}
\newcommand{\IM}{{\rm Im}}
\newcommand{\nulls}{{\rm null\;space}}

\newcommand{\dps}{\displaystyle}
\newcommand{\arraystart}{\renewcommand{\arraystretch}{1.8}}
\newcommand{\arrayfinish}{\renewcommand{\arraystretch}{1.2}}
\newcommand{\Start}[1]{\vspace{0.08in}\noindent {\bf Section~\ref{#1}}}
\newcommand{\exer}[1]{\noindent {\bf \ref{#1}}}
\newcommand{\ans}{}
\newcommand{\matthree}[9]{\left(\begin{array}{rrr} #1 & #2 & #3 \\ #4 & #5 & #6
\\ #7 & #8 & #9\end{array}\right)}
\newcommand{\cvectwo}[2]{\left(\begin{array}{c} #1 \\ #2\end{array}\right)}
\newcommand{\cmatthree}[9]{\left(\begin{array}{ccc} #1 & #2 & #3 \\ #4 & #5 &
#6 \\ #7 & #8 & #9\end{array}\right)}
\newcommand{\vecthree}[3]{\left(\begin{array}{r} #1 \\ #2 \\
#3\end{array}\right)}
\newcommand{\cvecthree}[3]{\left(\begin{array}{c} #1 \\ #2 \\
#3\end{array}\right)}
\newcommand{\cmattwo}[4]{\left(\begin{array}{cc} #1 & #2\\ #3
&#4\end{array}\right)}

\newcommand{\Matrix}[1]{\ensuremath{\left(\begin{array}{rrrrrrrrrrrrrrrrrr} #1 \end{array}\right)}}

\newcommand{\Matrixc}[1]{\ensuremath{\left(\begin{array}{cccccccccccc} #1 \end{array}\right)}}



\renewcommand{\labelenumi}{\theenumi)}
\newenvironment{enumeratea}%
{\begingroup
 \renewcommand{\theenumi}{\alph{enumi}}
 \renewcommand{\labelenumi}{(\theenumi)}
 \begin{enumerate}}
 {\end{enumerate}\endgroup}



\newcounter{help}
\renewcommand{\thehelp}{\thesection.\arabic{equation}}

%\newenvironment{equation*}%
%{\renewcommand\endequation{\eqno (\theequation)* $$}%
%   \begin{equation}}%
%   {\end{equation}\renewcommand\endequation{\eqno \@eqnnum
%$$\global\@ignoretrue}}

%\input{psfig.tex}

\author{Martin Golubitsky and Michael Dellnitz}

%\newenvironment{matlabEquation}%
%{\renewcommand\endequation{\eqno (\theequation*) $$}%
%   \begin{equation}}%
%   {\end{equation}\renewcommand\endequation{\eqno \@eqnnum
% $$\global\@ignoretrue}}

\newcommand{\soln}{\textbf{Solution:} }
\newcommand{\exercap}[1]{\centerline{Figure~\ref{#1}}}
\newcommand{\exercaptwo}[1]{\centerline{Figure~\ref{#1}a\hspace{2.1in}
Figure~\ref{#1}b}}
\newcommand{\exercapthree}[1]{\centerline{Figure~\ref{#1}a\hspace{1.2in}
Figure~\ref{#1}b\hspace{1.2in}Figure~\ref{#1}c}}
\newcommand{\para}{\hspace{0.4in}}

\renewenvironment{solution}{\suppress}{\endsuppress}

\ifxake
\newenvironment{matlabEquation}{\begin{equation}}{\end{equation}}
\else
\newenvironment{matlabEquation}%
{\let\oldtheequation\theequation\renewcommand{\theequation}{\oldtheequation*}\begin{equation}}%
  {\end{equation}\let\theequation\oldtheequation}
\fi

\makeatother

\begin{document}
\begin{exercise} \label{c13.4.2}
Reconfirm that the solution of the initial value problem
\[
\ddot x + 4 x=H_1(t),\quad x(0)=1,\quad \dot x(0)=0
\]
is given by
\[
x(t)=\frac{1}{4}H_1(t)(1-\cos(2(t-1))) + \cos(2t).
\]
Use the methods of Chapter \ref{C:LDE} and proceed as follows:
\begin{itemize}
\item[(a)] Find the solution $x_1(t)$ of the initial value problem
\[
\ddot x + 4 x=0,\quad x(0)=1,\quad \dot x(0)=0
\]
on the time interval $t\in[0,1]$.
\item[(b)] Find the general solution $x_2(t)$ of the second order
differential equation $\ddot x + 4 x=1$.
\item[(c)] Adjust the parameters in $x_2(t)$ such that the function
\[
x(t)=\left\{ \begin{array}{l} x_1(t)\quad t\in[0,1]\\
x_2(t) \quad t>1 \end{array}\right.
\]
is differentiable at $t=1$.
\end{itemize}

\begin{solution}

(a) Let $p(D) = D^2 + 4$.  Then the characteristic polynomial of
$p(D)$ has eigenvalues $\lambda = \pm 2i$, and the general solution to
$p(D)x = 0$ is
\[
x(t) = \alpha_1\cos(2t) + \alpha_2\sin(2t).
\]
Substitute the initial conditions into $x(t)$, obtaining
\[
1 = x(0) = \alpha_1 \AND
0 = \dot{x}(0) = \alpha_2.
\]
Thus,
\[
x_1(t) = \cos(2t).
\]

(b) Let $g(t) = 1$ and use the method of undetermined coefficients.
\paragraph{Step 1.} An annihilator for $g(t)$ is $q(D) = D$, since
the characteristic polynomial of $q(D)$ has eigenvalue $\lambda = 0$.

\paragraph{Step 2.} The eigenvalues of $p(D)$ are distinct from that
of $q(D)$, so the trial space is simply the solution space of $q(D)$,
which is $y(t) = c_1$.

\paragraph{Step 3.} Substitute $y(t)$ into $p(D)x = g(t)$, obtaining
\[
4c_1 = 1.
\]
So, the general solution to $\ddot{x} + 4x = 1$ is
\[
x_2(t) = \alpha_1\cos(2t) + \alpha_2\sin(2t) + \frac{1}{4}.
\]

(c) In order for $x(t)$ to be continuous at $t = 1$,
\[
\dps\lim_{t \rightarrow 1^{-}} x(t) = x_1(1) = \cos(2)
\]
must equal
\[
\dps\lim_{t \rightarrow 1^{+}} x(t) = x_2(1) =
\alpha_1\cos(2) + \alpha_2\sin(2) + \frac{1}{4}.
\]
So
\begin{equation} \label{ex:13.4.2a}
\cos(2)\alpha_1 + \sin(2)\alpha_2 = \cos(2) - \frac{1}{4}.
\end{equation}
In order for $x(t)$ to be differentiable at $t = 1$,
\[
\dps\lim_{h \rightarrow 0^{-}} \frac{x(1 + h) - x(1)}{h} =
\frac{dx_1}{dt}(1) = -2\sin(2)
\]
must equal
\[
\dps\lim_{h \rightarrow 0^{+}} \frac{x(1 + h) - x(1)}{h} =
\frac{dx_2}{dt}(1) = -2\alpha_1\sin(2) + 2\alpha_2\cos(2).
\]
So
\begin{equation} \label{ex:13.4.2b}
-2\sin(2)\alpha_1 + 2\cos(2)\alpha_2 = -2\sin(2).
\end{equation}
Solve \eqref{ex:13.4.2a} and \eqref{ex:13.4.2b} to find that $x(t)$ is
differentiable when $\alpha_1 = 1 - \frac{1}{4}\cos(2)$ and $\alpha_2
= -\frac{1}{4}\sin(2)$.  Substituting these values into $x_2(t)$ and
using trigonometric identities, we obtain
\[
\begin{array}{rcl}
x_1(t) & = & \cos(2t) \\
x_2(t) & = & \frac{1}{4}(1 - \cos(2(t - 1))) + \cos(2t)
\end{array}
\]
confirming the given solution to the initial value problem.











\end{solution}
\end{exercise}
\end{document}
