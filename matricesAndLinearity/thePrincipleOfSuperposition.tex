\documentclass{ximera}

\usepackage{epsfig}

\graphicspath{
  {./}
  {figures/}
}


\usepackage{morewrites}

%\newcounter{ccounter}
%\setcounter{ccounter}{1}
%\newcommand{\Chapter}[1]{\setcounter{chapter}{\arabic{ccounter}}\chapter{#1}\addtocounter{ccounter}{1}}

%\newcommand{\section}[1]{\section{#1}\setcounter{thm}{0}\setcounter{equation}{0}}

%\renewcommand{\theequation}{\arabic{chapter}.\arabic{section}.\arabic{equation}}
%\renewcommand{\thefigure}{\arabic{chapter}.\arabic{figure}}
%\renewcommand{\thetable}{\arabic{chapter}.\arabic{table}}

%\newcommand{\Sec}[2]{\section{#1}\markright{\arabic{ccounter}.\arabic{section}.#2}\setcounter{equation}{0}\setcounter{thm}{0}\setcounter{figure}{0}}

\newcommand{\Sec}[2]{\section{#1}}

\setcounter{secnumdepth}{2}
%\setcounter{secnumdepth}{1} 

%\newcounter{THM}
%\renewcommand{\theTHM}{\arabic{chapter}.\arabic{section}}

\newcommand{\trademark}{{R\!\!\!\!\!\bigcirc}}
%\newtheorem{exercise}{}

\newcommand{\dfield}{{\sf dfield9}}
\newcommand{\pplane}{{\sf pplane9}}

\newcommand{\EXER}{\section*{Exercises}}%\vspace*{0.2in}\hrule\small\setcounter{exercise}{0}}
\newcommand{\CEXER}{}%\vspace{0.08in}\begin{center}Computer Exercises\end{center}}
\newcommand{\TEXER}{} %\vspace{0.08in}\begin{center}Hand Exercises\end{center}}
\newcommand{\AEXER}{} %\vspace{0.08in}\begin{center}Hand Exercises\end{center}}

% BADBAD: \newcommand{\Bbb}{\bf}

\newcommand{\R}{\mbox{$\Bbb{R}$}}
\newcommand{\C}{\mbox{$\Bbb{C}$}}
\newcommand{\Z}{\mbox{$\Bbb{Z}$}}
\newcommand{\N}{\mbox{$\Bbb{N}$}}
\newcommand{\D}{\mbox{{\bf D}}}
\usepackage{amssymb}
%\newcommand{\qed}{\hfill\mbox{\raggedright$\square$} \vspace{1ex}}
%\newcommand{\proof}{\noindent {\bf Proof:} \hspace{0.1in}}

\newcommand{\setmin}{\;\mbox{--}\;}
\newcommand{\Matlab}{{M\small{AT\-LAB}} }
\newcommand{\Matlabp}{{M\small{AT\-LAB}}}
\newcommand{\computer}{\Matlab Instructions}
\newcommand{\half}{\mbox{$\frac{1}{2}$}}
\newcommand{\compose}{\raisebox{.15ex}{\mbox{{\scriptsize$\circ$}}}}
\newcommand{\AND}{\quad\mbox{and}\quad}
\newcommand{\vect}[2]{\left(\begin{array}{c} #1_1 \\ \vdots \\
 #1_{#2}\end{array}\right)}
\newcommand{\mattwo}[4]{\left(\begin{array}{rr} #1 & #2\\ #3
&#4\end{array}\right)}
\newcommand{\mattwoc}[4]{\left(\begin{array}{cc} #1 & #2\\ #3
&#4\end{array}\right)}
\newcommand{\vectwo}[2]{\left(\begin{array}{r} #1 \\ #2\end{array}\right)}
\newcommand{\vectwoc}[2]{\left(\begin{array}{c} #1 \\ #2\end{array}\right)}



\newcommand{\inv}{^{-1}}
\newcommand{\CC}{{\cal C}}
\newcommand{\CCone}{\CC^1}
\newcommand{\Span}{{\rm span}}
\newcommand{\rank}{{\rm rank}}
\newcommand{\trace}{{\rm tr}}
\newcommand{\RE}{{\rm Re}}
\newcommand{\IM}{{\rm Im}}
\newcommand{\nulls}{{\rm null\;space}}

\newcommand{\dps}{\displaystyle}
\newcommand{\arraystart}{\renewcommand{\arraystretch}{1.8}}
\newcommand{\arrayfinish}{\renewcommand{\arraystretch}{1.2}}
\newcommand{\Start}[1]{\vspace{0.08in}\noindent {\bf Section~\ref{#1}}}
\newcommand{\exer}[1]{\noindent {\bf \ref{#1}}}
\newcommand{\ans}{}
\newcommand{\matthree}[9]{\left(\begin{array}{rrr} #1 & #2 & #3 \\ #4 & #5 & #6
\\ #7 & #8 & #9\end{array}\right)}
\newcommand{\cvectwo}[2]{\left(\begin{array}{c} #1 \\ #2\end{array}\right)}
\newcommand{\cmatthree}[9]{\left(\begin{array}{ccc} #1 & #2 & #3 \\ #4 & #5 &
#6 \\ #7 & #8 & #9\end{array}\right)}
\newcommand{\vecthree}[3]{\left(\begin{array}{r} #1 \\ #2 \\
#3\end{array}\right)}
\newcommand{\cvecthree}[3]{\left(\begin{array}{c} #1 \\ #2 \\
#3\end{array}\right)}
\newcommand{\cmattwo}[4]{\left(\begin{array}{cc} #1 & #2\\ #3
&#4\end{array}\right)}

\newcommand{\Matrix}[1]{\ensuremath{\left(\begin{array}{rrrrrrrrrrrrrrrrrr} #1 \end{array}\right)}}

\newcommand{\Matrixc}[1]{\ensuremath{\left(\begin{array}{cccccccccccc} #1 \end{array}\right)}}



\renewcommand{\labelenumi}{\theenumi)}
\newenvironment{enumeratea}%
{\begingroup
 \renewcommand{\theenumi}{\alph{enumi}}
 \renewcommand{\labelenumi}{(\theenumi)}
 \begin{enumerate}}
 {\end{enumerate}\endgroup}



\newcounter{help}
\renewcommand{\thehelp}{\thesection.\arabic{equation}}

%\newenvironment{equation*}%
%{\renewcommand\endequation{\eqno (\theequation)* $$}%
%   \begin{equation}}%
%   {\end{equation}\renewcommand\endequation{\eqno \@eqnnum
%$$\global\@ignoretrue}}

%\input{psfig.tex}

\author{Martin Golubitsky and Michael Dellnitz}

%\newenvironment{matlabEquation}%
%{\renewcommand\endequation{\eqno (\theequation*) $$}%
%   \begin{equation}}%
%   {\end{equation}\renewcommand\endequation{\eqno \@eqnnum
% $$\global\@ignoretrue}}

\newcommand{\soln}{\textbf{Solution:} }
\newcommand{\exercap}[1]{\centerline{Figure~\ref{#1}}}
\newcommand{\exercaptwo}[1]{\centerline{Figure~\ref{#1}a\hspace{2.1in}
Figure~\ref{#1}b}}
\newcommand{\exercapthree}[1]{\centerline{Figure~\ref{#1}a\hspace{1.2in}
Figure~\ref{#1}b\hspace{1.2in}Figure~\ref{#1}c}}
\newcommand{\para}{\hspace{0.4in}}

\renewenvironment{solution}{\suppress}{\endsuppress}

\ifxake
\newenvironment{matlabEquation}{\begin{equation}}{\end{equation}}
\else
\newenvironment{matlabEquation}%
{\let\oldtheequation\theequation\renewcommand{\theequation}{\oldtheequation*}\begin{equation}}%
  {\end{equation}\let\theequation\oldtheequation}
\fi

\makeatother


\title{The Principle of Superposition}

\begin{document}
\begin{abstract}
\end{abstract}
\maketitle


\label{S:Superposition}

The principle of superposition is just a restatement of the fact
that matrix mappings are linear.  Nevertheless, this restatement
is helpful when trying to understand the structure of solutions
to systems of linear equations.

\subsection*{Homogeneous Equations}
\index{homogeneous}

A system of linear equations is {\em homogeneous\/} if it has
the form
\begin{equation} \label{homosys}
Ax=0,
\end{equation}
where $A$ is an $m\times n$ matrix and $x\in\R^n$.  Note that
homogeneous systems are consistent since $0\in\R^n$ is always a
solution, that is, $A(0)=0$.

The {\em principle of superposition\/} \index{principle of
superposition} \index{superposition} makes two assertions:
\begin{itemize}
\item  Suppose that $y$ and $z$ in $\R^n$ are solutions to \Ref{homosys}
(that is, suppose that $Ay=0$ and $Az=0$); then $y+z$ is a solution
to \Ref{homosys}.
\item Suppose that $c$ is a scalar; then $cy$ is a solution to
\Ref{homosys}.
\end{itemize}
The principle of superposition is proved using the linearity of matrix 
multiplication.  Calculate
\[
A(y+z) = Ay + Az = 0+0=0
\]
to verify that $y+z$ is a solution, and calculate
\[
A(cy) = c(Ay) = c\cdot 0 = 0
\]
to verify that $cy$ is a solution.

We see that solutions to homogeneous systems of linear equations
always satisfy the general property of superposition: sums of
solutions are solutions and scalar multiples of solutions are
solutions.

We illustrate this principle by explicitly solving the system of
equations
\[
\left(\begin{array}{rrrr} 1 & 2 & -1 & 1\\ 2 & 5 & -4 & -1
\end{array}\right)\left(\begin{array}{c} x_1\\x_2\\x_3\\x_4
\end{array}\right) = \left(\begin{array}{c} 0\\0
\end{array}\right).
\]
Use row reduction to show that the matrix
\[
\left(\begin{array}{rrrr} 1 & 2 & -1 & 1\\ 2 & 5 & -4 & -1
\end{array}\right)
\]
is row equivalent to
\[
\left(\begin{array}{rrrr} 1 & 0 & 3 & 7\\ 0 & 1 & -2 & -3
\end{array}\right)
\]
which is in reduced echelon form.  Recall, using the methods of
Section~\ref{S:Gauss}, that every
solution to this linear system has the form
\[
\left(\begin{array}{c} -3x_3-7x_4\\ 2x_3+3x_4\\ x_3\\
x_4\end{array}\right) =
x_3\left(\begin{array}{r}-3\\2\\1\\0\end{array}\right) +
x_4\left(\begin{array}{r}-7\\3\\0\\1\end{array}\right).
\]
Superposition is verified again by observing that the form of 
the solutions is preserved under vector addition and scalar
multiplication.  For instance, suppose that
\[
\alpha_1 \left(\begin{array}{r}-3\\2\\1\\0\end{array}\right) +
\alpha_2 \left(\begin{array}{r}-7\\3\\0\\1\end{array}\right)
\AND
\beta_1 \left(\begin{array}{r}-3\\2\\1\\0\end{array}\right) +
\beta_2 \left(\begin{array}{r}-7\\3\\0\\1\end{array}\right)
\]
are two solutions.  Then the sum has the form
\[
\gamma_1 \left(\begin{array}{r}-3\\2\\1\\0\end{array}\right) +
\gamma_2 \left(\begin{array}{r}-7\\3\\0\\1\end{array}\right)
\]
where $\gamma_j = \alpha_j + \beta_j$.


We have actually proved more than superposition.  We have shown
in this example that every solution is a superposition
of just two solutions
\[
\left(\begin{array}{r}-3\\2\\1\\0\end{array}\right) \AND
\left(\begin{array}{r}-7\\3\\0\\1\end{array}\right).
\]

\subsection*{Inhomogeneous Equations}
\index{inhomogeneous}

The linear system of $m$ equations in $n$ unknowns is written as
\[
Ax=b
\]
where $A$ is an $m\times n$ matrix, $x\in\R^n$, and $b\in\R^m$.
This system is {\em inhomogeneous\/} when the vector $b$ is nonzero.
Note that if $y,z\in\R^n$ are solutions to the inhomogeneous
equation (that is, $Ay=b$ and $Az=b$), then $y-z$ is a solution
to the homogeneous equation.  That is,
\[
A(y-z) = Ay - Az = b - b = 0.
\]
For example, let 
\[
A = \left(\begin{array}{rrr}  1 & 2 & 0 \\ -2 & 0 & 1 \end{array}\right)
\AND b = \vectwo{3}{-1}.
\]
Then 
\[
y = \left(\begin{array}{c} 1\\1\\1\end{array}\right) \AND 
z =  \left(\begin{array}{c} 3\\0\\5\end{array}\right)
\]
are both solutions to the linear system $Ax=b$.  It follows that 
\[
y-z = \left(\begin{array}{r} -2\\1\\-4\end{array}\right)
\]
is a solution to the homogeneous system $Ax=0$, which can be checked by 
direct calculation.

 
Thus we can completely solve the inhomogeneous equation by
finding one solution to the inhomogeneous equation and then
adding to that solution every solution of the homogeneous
equation. More precisely, suppose that we know all of the
solutions $w$ to the homogeneous equation $Ax=0$ and one
solution $y$ to the inhomogeneous equation $Ax=b$.  Then $y+w$
is another solution to the inhomogeneous equation and {\em
every\/} solution to the inhomogeneous equation has this form.

\subsubsection*{An Example of an Inhomogeneous Equation}

Suppose that we want to find all solutions of $Ax=b$ where
\[
A = \left(
\begin{array}{rrr}
 3 & 2 & 1  \\
 0 & 1 & -2  \\
 3 & 3 & -1
\end{array}
\right)\quad\mbox{and}\quad
b=\left(
\begin{array}{r}
 -2   \\
 4   \\
 2
\end{array}
\right).
\]
Suppose that you are told that $y=(-5,6,1)^t$ is a solution of the
inhomogeneous equation.  (This fact can be verified by a short calculation
--- just multiply $Ay$ and see that the result equals $b$.)  Next find
all solutions to the homogeneous equation $Ax=0$ by putting $A$ into reduced
echelon form.  The resulting row echelon form matrix is
\[
\left(
\begin{array}{rrr}
 1 & 0 & \frac{5}{3}  \\
 0 & 1 & -2  \\
 0 & 0 & 0
\end{array}
\right).
\]
Hence we see that the solutions of the homogeneous equation $Ax=0$ are
\[
\left(\begin{array}{r} -\frac{5}{3}s\\ 2s\\ s\end{array}\right) =
s\left(\begin{array}{r}-\frac{5}{3}\\2\\1\end{array}\right).
\]
Combining these results, we conclude that all the solutions
of $Ax=b$ are given by
\[
	\left(\begin{array}{r}-5\\6\\1\end{array}\right)+
	s\left(\begin{array}{r}-\frac{5}{3}\\2\\1\end{array}\right).
\]


\EXER

\TEXER

\begin{exercise} \label{c4.4.1}
Consider the homogeneous linear equation
\[
x+y+z = 0
\]
\begin{enumerate}
\item[(a)]  Write all solutions to this equation as a general
superposition of a pair of vectors $v_1$ and $v_2$.
\item[(b)]  Write all solutions as a general superposition of
a second pair of vectors $w_1$ and $w_2$.
\end{enumerate}
\end{exercise}

\begin{exercise} \label{c4.4.2}
Write all solutions to the homogeneous system of linear
equations
\begin{eqnarray*}
x_1+2x_2+x_4-x_5 = 0\\
x_3-2x_4+x_5 = 0
\end{eqnarray*}
as the general superposition of three vectors.
\end{exercise}

\begin{exercise} \label{c4.4.3}
\begin{itemize}
\item[(a)] Find all solutions to the homogeneous equation
$Ax=0$ where
\[
A = \left(\begin{array}{ccc} 2 & 3 & 1 \\ 1 & 1 & 4 \end{array}
\right).
\]
\item[(b)] Find a single solution to the inhomogeneous equation
\begin{equation}  \label{E:inhom}
Ax =\vectwo{6}{6}.
\end{equation}
\item[(c)] Use your answers in (a) and (b) to find all solutions
to \Ref{E:inhom}.
\end{itemize}
\end{exercise}







\end{document}
