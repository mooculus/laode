\documentclass{ximera}

\usepackage{epsfig}

\graphicspath{
  {./}
  {figures/}
}


\usepackage{morewrites}

%\newcounter{ccounter}
%\setcounter{ccounter}{1}
%\newcommand{\Chapter}[1]{\setcounter{chapter}{\arabic{ccounter}}\chapter{#1}\addtocounter{ccounter}{1}}

%\newcommand{\section}[1]{\section{#1}\setcounter{thm}{0}\setcounter{equation}{0}}

%\renewcommand{\theequation}{\arabic{chapter}.\arabic{section}.\arabic{equation}}
%\renewcommand{\thefigure}{\arabic{chapter}.\arabic{figure}}
%\renewcommand{\thetable}{\arabic{chapter}.\arabic{table}}

%\newcommand{\Sec}[2]{\section{#1}\markright{\arabic{ccounter}.\arabic{section}.#2}\setcounter{equation}{0}\setcounter{thm}{0}\setcounter{figure}{0}}

\newcommand{\Sec}[2]{\section{#1}}

\setcounter{secnumdepth}{2}
%\setcounter{secnumdepth}{1} 

%\newcounter{THM}
%\renewcommand{\theTHM}{\arabic{chapter}.\arabic{section}}

\newcommand{\trademark}{{R\!\!\!\!\!\bigcirc}}
%\newtheorem{exercise}{}

\newcommand{\dfield}{{\sf dfield9}}
\newcommand{\pplane}{{\sf pplane9}}

\newcommand{\EXER}{\section*{Exercises}}%\vspace*{0.2in}\hrule\small\setcounter{exercise}{0}}
\newcommand{\CEXER}{}%\vspace{0.08in}\begin{center}Computer Exercises\end{center}}
\newcommand{\TEXER}{} %\vspace{0.08in}\begin{center}Hand Exercises\end{center}}
\newcommand{\AEXER}{} %\vspace{0.08in}\begin{center}Hand Exercises\end{center}}

% BADBAD: \newcommand{\Bbb}{\bf}

\newcommand{\R}{\mbox{$\Bbb{R}$}}
\newcommand{\C}{\mbox{$\Bbb{C}$}}
\newcommand{\Z}{\mbox{$\Bbb{Z}$}}
\newcommand{\N}{\mbox{$\Bbb{N}$}}
\newcommand{\D}{\mbox{{\bf D}}}
\usepackage{amssymb}
%\newcommand{\qed}{\hfill\mbox{\raggedright$\square$} \vspace{1ex}}
%\newcommand{\proof}{\noindent {\bf Proof:} \hspace{0.1in}}

\newcommand{\setmin}{\;\mbox{--}\;}
\newcommand{\Matlab}{{M\small{AT\-LAB}} }
\newcommand{\Matlabp}{{M\small{AT\-LAB}}}
\newcommand{\computer}{\Matlab Instructions}
\newcommand{\half}{\mbox{$\frac{1}{2}$}}
\newcommand{\compose}{\raisebox{.15ex}{\mbox{{\scriptsize$\circ$}}}}
\newcommand{\AND}{\quad\mbox{and}\quad}
\newcommand{\vect}[2]{\left(\begin{array}{c} #1_1 \\ \vdots \\
 #1_{#2}\end{array}\right)}
\newcommand{\mattwo}[4]{\left(\begin{array}{rr} #1 & #2\\ #3
&#4\end{array}\right)}
\newcommand{\mattwoc}[4]{\left(\begin{array}{cc} #1 & #2\\ #3
&#4\end{array}\right)}
\newcommand{\vectwo}[2]{\left(\begin{array}{r} #1 \\ #2\end{array}\right)}
\newcommand{\vectwoc}[2]{\left(\begin{array}{c} #1 \\ #2\end{array}\right)}



\newcommand{\inv}{^{-1}}
\newcommand{\CC}{{\cal C}}
\newcommand{\CCone}{\CC^1}
\newcommand{\Span}{{\rm span}}
\newcommand{\rank}{{\rm rank}}
\newcommand{\trace}{{\rm tr}}
\newcommand{\RE}{{\rm Re}}
\newcommand{\IM}{{\rm Im}}
\newcommand{\nulls}{{\rm null\;space}}

\newcommand{\dps}{\displaystyle}
\newcommand{\arraystart}{\renewcommand{\arraystretch}{1.8}}
\newcommand{\arrayfinish}{\renewcommand{\arraystretch}{1.2}}
\newcommand{\Start}[1]{\vspace{0.08in}\noindent {\bf Section~\ref{#1}}}
\newcommand{\exer}[1]{\noindent {\bf \ref{#1}}}
\newcommand{\ans}{}
\newcommand{\matthree}[9]{\left(\begin{array}{rrr} #1 & #2 & #3 \\ #4 & #5 & #6
\\ #7 & #8 & #9\end{array}\right)}
\newcommand{\cvectwo}[2]{\left(\begin{array}{c} #1 \\ #2\end{array}\right)}
\newcommand{\cmatthree}[9]{\left(\begin{array}{ccc} #1 & #2 & #3 \\ #4 & #5 &
#6 \\ #7 & #8 & #9\end{array}\right)}
\newcommand{\vecthree}[3]{\left(\begin{array}{r} #1 \\ #2 \\
#3\end{array}\right)}
\newcommand{\cvecthree}[3]{\left(\begin{array}{c} #1 \\ #2 \\
#3\end{array}\right)}
\newcommand{\cmattwo}[4]{\left(\begin{array}{cc} #1 & #2\\ #3
&#4\end{array}\right)}

\newcommand{\Matrix}[1]{\ensuremath{\left(\begin{array}{rrrrrrrrrrrrrrrrrr} #1 \end{array}\right)}}

\newcommand{\Matrixc}[1]{\ensuremath{\left(\begin{array}{cccccccccccc} #1 \end{array}\right)}}



\renewcommand{\labelenumi}{\theenumi)}
\newenvironment{enumeratea}%
{\begingroup
 \renewcommand{\theenumi}{\alph{enumi}}
 \renewcommand{\labelenumi}{(\theenumi)}
 \begin{enumerate}}
 {\end{enumerate}\endgroup}



\newcounter{help}
\renewcommand{\thehelp}{\thesection.\arabic{equation}}

%\newenvironment{equation*}%
%{\renewcommand\endequation{\eqno (\theequation)* $$}%
%   \begin{equation}}%
%   {\end{equation}\renewcommand\endequation{\eqno \@eqnnum
%$$\global\@ignoretrue}}

%\input{psfig.tex}

\author{Martin Golubitsky and Michael Dellnitz}

%\newenvironment{matlabEquation}%
%{\renewcommand\endequation{\eqno (\theequation*) $$}%
%   \begin{equation}}%
%   {\end{equation}\renewcommand\endequation{\eqno \@eqnnum
% $$\global\@ignoretrue}}

\newcommand{\soln}{\textbf{Solution:} }
\newcommand{\exercap}[1]{\centerline{Figure~\ref{#1}}}
\newcommand{\exercaptwo}[1]{\centerline{Figure~\ref{#1}a\hspace{2.1in}
Figure~\ref{#1}b}}
\newcommand{\exercapthree}[1]{\centerline{Figure~\ref{#1}a\hspace{1.2in}
Figure~\ref{#1}b\hspace{1.2in}Figure~\ref{#1}c}}
\newcommand{\para}{\hspace{0.4in}}

\renewenvironment{solution}{\suppress}{\endsuppress}

\ifxake
\newenvironment{matlabEquation}{\begin{equation}}{\end{equation}}
\else
\newenvironment{matlabEquation}%
{\let\oldtheequation\theequation\renewcommand{\theequation}{\oldtheequation*}\begin{equation}}%
  {\end{equation}\let\theequation\oldtheequation}
\fi

\makeatother


\title{Vectors and Matrices}

\begin{document}
\begin{abstract}
\end{abstract}
\maketitle


\label{S:1.1}

In their elementary form, matrices and vectors are just lists of 
real numbers in different formats.  An $n$-vector is a list of $n$ 
numbers $(x_1,x_2,\ldots,x_n)$.  We may write this vector as a 
{\em row\/} vector\index{vector} as we have just done --- or as a 
{\em column\/} vector
\[
\vect{x}{n}.
\]
The set of all (real-valued) $n$-vectors is denoted by $\R^n$\index{$\R^n$}; 
so points in $\R^n$ are called vectors.  The sets $\R^n$ when $n$ is small
are very familiar sets.  The set $\R^1=\R$ is the real number
line, and the set $\R^2$ is the Cartesian plane\index{Cartesian plane}.
The set $\R^3$ consists of points or vectors in three dimensional space.

An $m\times n$ {\em matrix\/}\index{matrix} is a rectangular array
of numbers with $m$ rows and $n$ columns.  A general $2\times 3$
matrix has the form
\[
A=\left(\begin{array}{ccc} a_{11} & a_{12} & a_{13} \\
     a_{21} & a_{22} & a_{23} \end{array}\right).
\]
We use the convention that matrix entries $a_{ij}$ are indexed
so that the first subscript $i$ refers to the {\em row\/}
\index{row} while the second subscript $j$ refers to the {\em
column\/}\index{column}.  So the entry $a_{21}$ refers to the
matrix entry in the $2^{nd}$ row, $1^{st}$ column.

An $n\times m$ matrix $A$ and an $n'\times m'$ matrix $B$ are equal
precisely when the sizes of the matrices are equal ($n=n'$ and $m=m'$)
and when each of the corresponding entries are equal ($a_{ij}=b_{ij}$).

There is some redundancy in the use of the terms ``vector'' and
``matrix''.  For example, a row $n$-vector may be thought of as a 
$1\times n$ matrix, and a column $n$-vector may be thought of as a 
$n\times 1$ matrix.  There are situations where matrix notation is 
preferable to vector notation and vice-versa.


\subsection*{Addition and Scalar Multiplication of Vectors}

There are two basic operations on vectors: addition and scalar
multiplication.  Let $x=(x_1,\ldots,x_n)$ and
$y=(y_1,\ldots,y_n)$ be $n$-vectors.  Then
\[
x+y=(x_1+y_1,\ldots,x_n+y_n);
\]
that is, {\em vector addition\/}\index{vector!addition} is
defined as componentwise addition.

Similarly, {\em scalar multiplication\/}\index{scalar multiplication}
is defined as
componentwise multiplication.  A {\em scalar\/} is just a
number. Initially, we use the term scalar\index{scalar} to refer to a real
number --- but later on we sometimes use the term scalar
to refer to a {\em complex\/} number.  Suppose $r$ is a real
number; then the multiplication of a vector by the scalar $r$
is defined as
\[
rx = (rx_1,\ldots,rx_n).
\]

Subtraction of vectors\index{vector!subtraction} is defined simply as
\[
x-y = (x_1-y_1,\ldots,x_n-y_n).
\]
Formally, subtraction of vectors may also be defined as
\[
x-y = x+(-1)y.
\]
Division of a vector $x$ by a scalar $r$ is defined to be
\[
\frac{1}{r} x.
\]
The standard difficulties concerning division by zero still hold.

\subsection*{Addition and Scalar Multiplication of Matrices}

Similarly, we add two $m\times n$ matrices\index{matrix!addition}
by adding corresponding
entries, and we multiply a scalar times a matrix by multiplying each
entry of the matrix by that scalar\index{matrix!scalar multiplication}.
For example,
\[
\mattwo{0}{2}{4}{6} + \mattwo{1}{-3}{1}{4} = \mattwo{1}{-1}{5}{10}
\]
and
\[
4\mattwo{2}{-4}{3}{1} = \mattwo{8}{-16}{12}{4}.
\]
The main restriction on adding two matrices is that the matrices must
be of the same size.  So you cannot add a $4\times 3$ matrix to $6\times 2$
matrix --- even though they both have twelve entries.



\EXER

\TEXER

\noindent In Exercises~\ref{c1.1.1A} -- \ref{c1.1.1C}, let $x=(2,1,3)$ and 
$y=(1,1,-1)$ and compute the given expression.
\begin{exercise}  \label{c1.1.1A}
  $x+y\begin{prompt}
    = \left(\answer{3},\answer{2},\answer{2}\right).
  \end{prompt}$.
\end{exercise}
\begin{exercise}  \label{c1.1.1B}
  $2x-3y\begin{prompt}
    = \left(\answer{1},\answer{-1},\answer{9}\right)
  \end{prompt}$.
  \begin{hint}
    $2x - 3y = (4,2,6) - (3,3,-3)$.
  \end{hint}
  \begin{hint}
    $(4,2,6) - (3,3,-3) = (1,-1,9)$.
  \end{hint}  
\end{exercise}
\begin{exercise}  \label{c1.1.1C}
  $4x\begin{prompt}
    = \left(\answer{8},\answer{4},\answer{12}\right)
    \end{prompt}$.
\end{exercise}

\begin{exercise} \label{c1.1.2}
Let $A$ be the $3\times 4$ matrix
\[
A=\left(\begin{array}{rrrr} 2 & -1 & 0 & 1 \\ 3 & 4 & -7 & 10\\
        6 & -3 & 4 & 2 \end{array}\right).
\]
\begin{enumerate}
\item[(a)]  For which $n$ is a row of $A$ a vector in $\R^n$? \begin{prompt}$n = \answer{4}$\end{prompt}.
\item[(b)]  What is the $2^{nd}$ column of $A$?
  \begin{prompt}
    \[
      \left(\begin{array}{r} \answer{-1} \\ \answer{4} \\ \answer{-3} \end{array} \right)
    \]
  \end{prompt}
\item[(c)] Let $a_{ij}$ be the entry of $A$ in the $i^{th}$ row
  and the $j^{th}$ column.  What is $a_{23}-a_{31}$?
  \begin{prompt}
    \[
      a_{23}-a_{31} = \answer{-13}.
    \]
  \end{prompt}
\end{enumerate}
\end{exercise}

\noindent For each of the pairs of vectors or matrices in
Exercises~\ref{c1.1.3a} -- \ref{c1.1.3e}, decide whether addition
of the members of the pair is possible; and, if addition is possible,
perform the addition.
\begin{exercise}\label{c1.1.3a}
  $x=(2,1)$ and $y=(3,-1)$.
  
  \begin{multipleChoice}
    \choice[correct]{Addition is possible.}
    \choice{Addition is not possible.}
  \end{multipleChoice}
  \begin{exercise}
    $x + y = \left(\answer{5},\answer{0}\right)$.
  \end{exercise}
\end{exercise}

\begin{exercise}\label{c1.1.3b}
  $x=(1,2,2)$ and $y=(-2,1,4)$.
  
  \begin{multipleChoice}
    \choice[correct]{Addition is possible.}
    \choice{Addition is not possible.}
  \end{multipleChoice}
  \begin{exercise}
    $x + y = \left(\answer{-1},\answer{3},\answer{6}\right)$.
  \end{exercise}
\end{exercise}

\begin{exercise}\label{c1.1.3c}
  $x=(1,2,3)$ and $y=(-2,1)$.
  
  \begin{multipleChoice}
    \choice{Addition is possible.}
    \choice[correct]{Addition is not possible.}
  \end{multipleChoice}  
\end{exercise}

\begin{exercise}\label{c1.1.3d}
  $A=\mattwo{1}{3}{0}{4}$ and $B=\mattwo{2}{1}{1}{-2}$.
  
  \begin{multipleChoice}
    \choice[correct]{Addition is possible.}
    \choice{Addition is not possible.}
  \end{multipleChoice}
  \begin{exercise}
    $A + B = \mattwo{\answer{3}}{\answer{4}}{\answer{1}}{\answer{2}}$.
  \end{exercise}
\end{exercise}

\begin{exercise}\label{c1.1.3e}
  $A=\left(\begin{array}{rrr} 2 & 1 & 0\\ 4 & 1 & 0\\
             0 & 0 & 0\end{array}\right)$ and $B=\mattwo{2}{1}{1}{-2}$.
         
         \begin{multipleChoice}
           \choice{Addition is possible.}
           \choice[correct]{Addition is not possible.}
         \end{multipleChoice}         
\end{exercise}

\noindent In Exercises~\ref{c1.1.4A} -- \ref{c1.1.4B}, let
$A=\mattwo{2}{1}{-1}{4}$ and $B=\mattwo{0}{2}{3}{-1}$ and compute the given 
expression.
\begin{exercise}\label{c1.1.4A}
  $4A+B\begin{prompt}= \mattwo{\answer{8}}{\answer{6}}{\answer{-1}}{\answer{15}}\end{prompt}$.
\end{exercise}
\begin{exercise}\label{c1.1.4B}
  $2A-3B\begin{prompt}=\mattwo{\answer{4}}{\answer{-4}}{\answer{-11}}{\answer{11}}\end{prompt}$.
\end{exercise}

\end{document}
