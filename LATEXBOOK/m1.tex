\documentclass{ximera}

\usepackage{epsfig}

\graphicspath{
  {./}
  {figures/}
}


\usepackage{morewrites}

%\newcounter{ccounter}
%\setcounter{ccounter}{1}
%\newcommand{\Chapter}[1]{\setcounter{chapter}{\arabic{ccounter}}\chapter{#1}\addtocounter{ccounter}{1}}

%\newcommand{\section}[1]{\section{#1}\setcounter{thm}{0}\setcounter{equation}{0}}

%\renewcommand{\theequation}{\arabic{chapter}.\arabic{section}.\arabic{equation}}
%\renewcommand{\thefigure}{\arabic{chapter}.\arabic{figure}}
%\renewcommand{\thetable}{\arabic{chapter}.\arabic{table}}

%\newcommand{\Sec}[2]{\section{#1}\markright{\arabic{ccounter}.\arabic{section}.#2}\setcounter{equation}{0}\setcounter{thm}{0}\setcounter{figure}{0}}

\newcommand{\Sec}[2]{\section{#1}}

\setcounter{secnumdepth}{2}
%\setcounter{secnumdepth}{1} 

%\newcounter{THM}
%\renewcommand{\theTHM}{\arabic{chapter}.\arabic{section}}

\newcommand{\trademark}{{R\!\!\!\!\!\bigcirc}}
%\newtheorem{exercise}{}

\newcommand{\dfield}{{\sf dfield9}}
\newcommand{\pplane}{{\sf pplane9}}

\newcommand{\EXER}{\section*{Exercises}}%\vspace*{0.2in}\hrule\small\setcounter{exercise}{0}}
\newcommand{\CEXER}{}%\vspace{0.08in}\begin{center}Computer Exercises\end{center}}
\newcommand{\TEXER}{} %\vspace{0.08in}\begin{center}Hand Exercises\end{center}}
\newcommand{\AEXER}{} %\vspace{0.08in}\begin{center}Hand Exercises\end{center}}

% BADBAD: \newcommand{\Bbb}{\bf}

\newcommand{\R}{\mbox{$\Bbb{R}$}}
\newcommand{\C}{\mbox{$\Bbb{C}$}}
\newcommand{\Z}{\mbox{$\Bbb{Z}$}}
\newcommand{\N}{\mbox{$\Bbb{N}$}}
\newcommand{\D}{\mbox{{\bf D}}}
\usepackage{amssymb}
%\newcommand{\qed}{\hfill\mbox{\raggedright$\square$} \vspace{1ex}}
%\newcommand{\proof}{\noindent {\bf Proof:} \hspace{0.1in}}

\newcommand{\setmin}{\;\mbox{--}\;}
\newcommand{\Matlab}{{M\small{AT\-LAB}} }
\newcommand{\Matlabp}{{M\small{AT\-LAB}}}
\newcommand{\computer}{\Matlab Instructions}
\newcommand{\half}{\mbox{$\frac{1}{2}$}}
\newcommand{\compose}{\raisebox{.15ex}{\mbox{{\scriptsize$\circ$}}}}
\newcommand{\AND}{\quad\mbox{and}\quad}
\newcommand{\vect}[2]{\left(\begin{array}{c} #1_1 \\ \vdots \\
 #1_{#2}\end{array}\right)}
\newcommand{\mattwo}[4]{\left(\begin{array}{rr} #1 & #2\\ #3
&#4\end{array}\right)}
\newcommand{\mattwoc}[4]{\left(\begin{array}{cc} #1 & #2\\ #3
&#4\end{array}\right)}
\newcommand{\vectwo}[2]{\left(\begin{array}{r} #1 \\ #2\end{array}\right)}
\newcommand{\vectwoc}[2]{\left(\begin{array}{c} #1 \\ #2\end{array}\right)}



\newcommand{\inv}{^{-1}}
\newcommand{\CC}{{\cal C}}
\newcommand{\CCone}{\CC^1}
\newcommand{\Span}{{\rm span}}
\newcommand{\rank}{{\rm rank}}
\newcommand{\trace}{{\rm tr}}
\newcommand{\RE}{{\rm Re}}
\newcommand{\IM}{{\rm Im}}
\newcommand{\nulls}{{\rm null\;space}}

\newcommand{\dps}{\displaystyle}
\newcommand{\arraystart}{\renewcommand{\arraystretch}{1.8}}
\newcommand{\arrayfinish}{\renewcommand{\arraystretch}{1.2}}
\newcommand{\Start}[1]{\vspace{0.08in}\noindent {\bf Section~\ref{#1}}}
\newcommand{\exer}[1]{\noindent {\bf \ref{#1}}}
\newcommand{\ans}{}
\newcommand{\matthree}[9]{\left(\begin{array}{rrr} #1 & #2 & #3 \\ #4 & #5 & #6
\\ #7 & #8 & #9\end{array}\right)}
\newcommand{\cvectwo}[2]{\left(\begin{array}{c} #1 \\ #2\end{array}\right)}
\newcommand{\cmatthree}[9]{\left(\begin{array}{ccc} #1 & #2 & #3 \\ #4 & #5 &
#6 \\ #7 & #8 & #9\end{array}\right)}
\newcommand{\vecthree}[3]{\left(\begin{array}{r} #1 \\ #2 \\
#3\end{array}\right)}
\newcommand{\cvecthree}[3]{\left(\begin{array}{c} #1 \\ #2 \\
#3\end{array}\right)}
\newcommand{\cmattwo}[4]{\left(\begin{array}{cc} #1 & #2\\ #3
&#4\end{array}\right)}

\newcommand{\Matrix}[1]{\ensuremath{\left(\begin{array}{rrrrrrrrrrrrrrrrrr} #1 \end{array}\right)}}

\newcommand{\Matrixc}[1]{\ensuremath{\left(\begin{array}{cccccccccccc} #1 \end{array}\right)}}



\renewcommand{\labelenumi}{\theenumi)}
\newenvironment{enumeratea}%
{\begingroup
 \renewcommand{\theenumi}{\alph{enumi}}
 \renewcommand{\labelenumi}{(\theenumi)}
 \begin{enumerate}}
 {\end{enumerate}\endgroup}



\newcounter{help}
\renewcommand{\thehelp}{\thesection.\arabic{equation}}

%\newenvironment{equation*}%
%{\renewcommand\endequation{\eqno (\theequation)* $$}%
%   \begin{equation}}%
%   {\end{equation}\renewcommand\endequation{\eqno \@eqnnum
%$$\global\@ignoretrue}}

%\input{psfig.tex}

\author{Martin Golubitsky and Michael Dellnitz}

%\newenvironment{matlabEquation}%
%{\renewcommand\endequation{\eqno (\theequation*) $$}%
%   \begin{equation}}%
%   {\end{equation}\renewcommand\endequation{\eqno \@eqnnum
% $$\global\@ignoretrue}}

\newcommand{\soln}{\textbf{Solution:} }
\newcommand{\exercap}[1]{\centerline{Figure~\ref{#1}}}
\newcommand{\exercaptwo}[1]{\centerline{Figure~\ref{#1}a\hspace{2.1in}
Figure~\ref{#1}b}}
\newcommand{\exercapthree}[1]{\centerline{Figure~\ref{#1}a\hspace{1.2in}
Figure~\ref{#1}b\hspace{1.2in}Figure~\ref{#1}c}}
\newcommand{\para}{\hspace{0.4in}}

\renewenvironment{solution}{\suppress}{\endsuppress}

\ifxake
\newenvironment{matlabEquation}{\begin{equation}}{\end{equation}}
\else
\newenvironment{matlabEquation}%
{\let\oldtheequation\theequation\renewcommand{\theequation}{\oldtheequation*}\begin{equation}}%
  {\end{equation}\let\theequation\oldtheequation}
\fi

\makeatother


\title{m1.tex}

\begin{document}
\begin{abstract}
BADBAD
\end{abstract}
\maketitle

\setcounter{chapter}{0}

\chapter{Preliminaries}

\subsection*{Section~\protect{\ref{S:1.1}} Vectors and Matrices}
\rhead{S:1.1}{VECTORS AND MATRICES}

\exer{c1.1.1A} $ x + y = (3,2,2)$.

\exer{c1.1.1B} $4x = (8,4,12)$.

\exer{c1.1.1C} $2x - 3y = (4,2,6) - (3,3,-3) = (1,-1,9)$.

\exer{c1.1.2}
(a) $n = 4$;
(b) $\left(\begin{array}{r} -1 \\ 4 \\ -3 \end{array} \right)$;
(c) $a_{23}-a_{31} =  -7 - 6 = -13$.

\exer{c1.1.3a} $x + y = (5,0)$.

\exer{c1.1.3b} $x + y = (-1,3,6)$.

\exer{c1.1.3c} Addition is not possible.

\exer{c1.1.3d} $A + B = \mattwo{3}{4}{1}{2}$.

\exer{c1.1.3e} Addition is not possible.

\exer{c1.1.4A} $4A + B = \mattwo{8}{6}{-1}{15}$.

\newpage
\exer{c1.1.4B} $2A - 3B = \mattwo{4}{-4}{-11}{11}$.



\subsection*{Section~\protect{\ref{S:1.2}} \protect{\Matlab}}
\rhead{S:1.2}{MATLAB}

\exer{c1.2.1}
(a) $11$;
(b) $\left(\begin{array}{r} 15\\ 3 \\24\end{array} \right)$;
(c) $2(-1,2, 1,-2)- (4, 6, 8, 0)=(-6,-2,-6,-4)$;
(d) $\left(\begin{array}{r} 15 \\ 0 \\ 18 \end{array} \right)$.

\exer{c1.2.2}
Typing {\tt x + y} should generate a \Matlab error in both cases.

\exer{c1.2.3a}  $3.27x - 7.4y = (23.1640, -3.5620, -12.8215)$.

\exer{c1.2.3b}  $1.65x + 2.46y = (-4.4160, 5.0160, -2.4435)$.

\exer{c1.2.4a}  $-4.2A + 3.1B = \left(\begin{array}{rrr} 
-14.0300 & -5.8470 &    7.0600 \\
 -9.7600 & 11.0570 &   -9.6600\end{array}\right)$.

\exer{c1.2.4b}  $2.67A - 1.1B = \left(\begin{array}{rrr} 
    6.3940  &  4.7880 &  -3.0950\\
    4.2890  & -5.5750 &   6.1410 \end{array}\right)$.



\subsection*{Section~\protect{\ref{S:1.3}} Special Kinds of Matrices}
\rhead{S:1.3}{SPECIAL KINDS OF MATRICES}

\exer{c1.1.01a} The matrix is symmetric.

\exer{c1.1.01b} The matrix is not symmetric.

\exer{c1.1.01c} The matrix is symmetric.

\exer{c1.1.01d} The matrix is not symmetric.

\exer{c1.1.01e} The matrix is symmetric.

\exer{c1.1.02a} The matrix is lower triangular.

\exer{c1.1.02b} The matrix is strictly upper triangular.

\exer{c1.1.02c} The matrix is upper triangular.

\exer{c1.1.02d} The matrix is not upper triangular since a triangular
matrix must be square.

\newpage
\exer{c1.1.02e} The matrix is upper triangular.

\exer{c1.3.1a} $3$.

\exer{c1.3.1b} $3$.

\exer{c1.3.2} $mn$.

\exer{c1.3.3a} $n$.

\exer{c1.3.3b} $1 + 2 + \cdots + (n-1) + n = \frac{n(n + 1)}{2}$.

\exer{c1.3.3c} $\frac{n(n + 1)}{2}$.

\exer{c1.3.4a} $True$.

\exer{c1.3.4b} $False$ --- for example:
$\left(\begin{array}{ccc}
2 & 0 & 0 \\
0 & 1 & 0 \\
0 & 0 & 3 \end{array}\right)$.

\exer{c1.3.4c} $False$ --- for example:
$\left(\begin{array}{ccc}
1 & 2 & 0 \\
3 & 1 & 0 \\
0 & 0 & 4
\end{array}\right)$.

\exer{c1.3.5a} \ans $A^t =
\left(\begin{array}{rrr}
1 & 2 & 4 \\
2 & 1 & 6 \\
4 & 5 & 2 \\
7 & 6 & 1
\end{array}\right).$

\exer{c1.3.5b} \ans $A^t = (3).$



\subsection*{Section~\protect{\ref{S:1.4}} The Geometry of Vector Operations}
\rhead{S:1.4}{THE GEOMETRY OF VECTOR OPERATIONS}

\exer{c1.4.8a} \ans The length of $x$ is $\sqrt{3^2 + 0^2} = 3$.

\exer{c1.4.8b} \ans The length of $x$ is $\sqrt{2^2 + (-1)^2} = \sqrt{5}$.

\exer{c1.4.8c} \ans The length of $x$ is $\sqrt{(-1)^2 + 1^2 + 1^2} =
\sqrt{3}$.

\exer{c1.4.8d} \ans The length of $x$ is $\sqrt{(-1)^2 + 0^2 + 2^2 +
(-1)^2 + 3^2} = \sqrt{15}$.

\newpage
\exer{c1.4.1a} \ans The vectors are perpendicular.

\soln Vectors $x$ and $y$ are perpendicular if and only if $x \cdot y = 0$.
In this case, $(1,3) \cdot (3,-1) = 0$.

\exer{c1.4.1b} \ans The vectors are not perpendicular.

\soln Compute: $(2,-1) \cdot (-2,1) = -5$.

\exer{c1.4.1bb} \ans The vectors are not perpendicular.

\soln Compute: $(1,1,3,5) \cdot (1,-4,3,0) = 6$.

\exer{c1.4.1c} \ans The vectors are perpendicular.

\soln Compute: $(2,1,4,5) \cdot (1,-4,3,-2) = 0$.

\exer{c1.4.2}
When $a = \frac{10}{3}$, $x$ and $y$ are perpendicular, since
$(1,3,2) \cdot (2,a,-6) = 3a - 10 = 0$.

\exer{c1.4.3}
$||u||  =  \sqrt{2^2 + 1^2 + (-2)^2}  =  3$;

$||v||  =  \sqrt{0^2 + 1^2 + (-1)^2}  =  \sqrt{2}$;

$
\cos \theta = \frac{u \cdot v}{||u||\;||v||}  =  \frac{3}{3\sqrt{2}} =
\frac{1}{\sqrt{2}} = \frac{\pi}{4} = 45^\circ.
$

\exer{c1.4.9a} \ans The dot product $x \cdot y = 4$, and the cosine
of the angle $\theta$ between $x$ and $y$ is $\frac{2}{\sqrt{5}}$.

\soln Compute $x \cdot y = 4$, $||x|| = 2$, and $||y|| = \sqrt{5}$.
Then by Theorem~\ref{T:dotangle},
\[
\cos\theta = \frac{x \cdot y}{||x||\;||y||} = \frac{2}{\sqrt{5}}.
\]

\exer{c1.4.9b} \ans The dot product $x \cdot y = 0$, and the cosine
of the angle $\theta$ between $x$ and $y$ is $0$.

\soln Compute $x \cdot y = 0$, $||x|| = \sqrt{5}$, and $||y|| = \sqrt{5}$.
Then by Theorem~\ref{T:dotangle},
\[
\cos\theta = \frac{x \cdot y}{||x||\;||y||} = \frac{0}{5}
= 0.
\]

\exer{c1.4.9c} \ans The dot product $x \cdot y = 13$, and the cosine
of the angle $\theta$ between $x$ and $y$ is $\frac{13}{6\sqrt{5}}$.

\soln Compute $x \cdot y = 13$, $||x|| = 3\sqrt{2}$, and $||y|| = \sqrt{10}$.
Then by Theorem~\ref{T:dotangle},
\[
\cos\theta = \frac{x \cdot y}{||x||\;||y||} = \frac{13}{3\sqrt{20}}
= \frac{13}{6\sqrt{5}}.
\]

\exer{c1.4.9d} \ans The dot product $x \cdot y = 1$, and the cosine
of the angle $\theta$ between $x$ and $y$ is
$\frac{1}{\sqrt{40501}} \approx 0.0050$.

\soln Compute $x \cdot y = 1$, $||x|| = \sqrt{101}$, and $||y|| = \sqrt{401}$.
Then by Theorem~\ref{T:dotangle},
\[
\cos\theta = \frac{x \cdot y}{||x||\;||y||} = \frac{1}{\sqrt{101}\sqrt{401}}
= \frac{1}{\sqrt{40501}} \approx 0.0050.
\]

\exer{c1.4.9e} \ans The dot product $x \cdot y = 31$, and the cosine
of the angle $\theta$ between $x$ and $y$ is
$\frac{31}{\sqrt{1410}} \approx 0.8256$.

\soln Compute $x \cdot y = 31$, $||x|| = \sqrt{15}$, and $||y|| = \sqrt{94}$.
Then by Theorem~\ref{T:dotangle},
\[
\cos\theta = \frac{x \cdot y}{||x||\;||y||} = \frac{31}{\sqrt{15}\sqrt{94}}
= \frac{31}{\sqrt{1410}} \approx 0.8256.
\]

\exer{c1.4.9f} \ans The dot product $x \cdot y = -14$, and the cosine
of the angle $\theta$ between $x$ and $y$ is
$\frac{-14}{\sqrt{5805}} \approx -0.1837$.

\soln Compute $x \cdot y = -14$, $||x|| = \sqrt{43}$, and $||y|| = \sqrt{135}$.
Then by Theorem~\ref{T:dotangle},
\[
\cos\theta = \frac{x \cdot y}{||x||\;||y||} = -\frac{14}{\sqrt{43}\sqrt{135}}
= -\frac{14}{\sqrt{5805}} \approx -0.1837.
\]

\exer{c1.4.9A} Using the definition $||x|| = \sqrt{x_1^2 + \cdots + x_n^2}$,
we can compute
\[
\begin{array}{rcl}
||rx|| & = & ||r(x_1,\dots,x_n)|| \\
& = & ||(rx_1,\dots,rx_n)|| \\
& = & \sqrt{(rx_1)^2 + \cdots + (rx_n)^2} \\
& = & \sqrt{r^2(x_1^2 + \cdots + x_n^2)} \\
& = & |r|\sqrt{x_1^2 + \cdots + x_n^2} \\
& = & |r| ||x||.
\end{array}
\]


%\exer{c1.4.4} no written solution necessary

\exer{c1.4.5}
$\frac{x}{||x||} = (0.1244, 0.8397, -0.4167, 0.3253)$.

\exer{c1.4.5b}
$\frac{x}{||x||} = (0.3043, -0.5071, 0.2173, 0.1883, 0.7534)$.

\exer{c1.4.6a} $\theta =
\arccos \left(\frac{x \cdot y}{||x||\;||y||}\right) =
0.2715 = 15.5570^\circ$.

\exer{c1.4.6b} $\theta =
\arccos \left(\frac{x \cdot y}{||x||\;||y||}\right) =
2.0701 = 118.6076^\circ$.

\exer{c1.4.6c} $\theta =
\arccos \left(\frac{x \cdot y}{||x||\;||y||}\right) =
2.1769 = 124.7286^\circ$.

\exer{c1.4.7a} \ans The area of $P$ is $\sqrt{2294} \approx 47.8957$.

\soln Using \Ref{e:areaP},
\[
|P|^2 = ||v||^2||w||^2 - (v \cdot w)^2 = 75(189) - 109^2 = 2294.
\]

\exer{c1.4.7b} \ans The area of $P$ is $\sqrt{147} \approx 12.1244$.
\end{document}
