\chapter{Determinants and Eigenvalues}

\subsection*{Section~\protect{\ref{S:det}} Determinants}
\rhead{S:det}{DETERMINANTS}

\exer{c10.1.1a}
\ans The determinant of the matrix is $-28$.

\soln Expand along the third column, obtaining:
\[
\det\matthree{-2}{1}{0}{4}{5}{0}{1}{0}{2} = 2\det\mattwo{-2}{1}{4}{5}
= 2(-14) = -28.
\]

\exer{c10.1.1c}
\ans The determinant of the matrix is $14$.

\soln Using Lemma~\ref{L:detblockdiag}, compute
\[ 
\begin{array}{rcl}
\det(A) & = & 
\det\matthree{2}{1}{-1}{1}{-2}{3}{-3}{2}{-2}
\det\mattwo{2}{4}{-1}{-3} \\
& = & \big(2(-2)(-2) + 1 \cdot 3(-3) + (-1)1 \cdot 2
- (-1)(-2)(-3) - 1 \cdot 1(-2) \\
& & - 2 \cdot 3 \cdot 2\big)
(-2) \\
& = & (-7)(-2) \\
& = & 14.
\end{array}
\]

\exer{c10.1.3}
Two $n \times n$ matrices $B$ and $C$ are similar if there exists
an $n \times n$ matrix $P$ such that $B = P^{-1}CP$.  Therefore, by
Definition~\ref{D:determinants}(c),
\[
\det(B) = \det(P^{-1}CP)
= \det(P^{-1})\det(C)\det(P)
= \det(P)^{-1}\det(P)\det(C)
= \det(C).
\]

\exer{c10.1.5a}
\ans The determinant is $\det(B) = -4$.

\soln Compute by row reduction:
\[
\det\left(\begin{array}{rrrr}
1 & 0 & 1 & 0 \\
0 & 1 & 0 & -1 \\
1 & 0 & -1 & 0 \\
0 & 1 & 0 & 1 \end{array}\right)
= \det\left(\begin{array}{rrrr}
1 & 0 & 1 & 0 \\
0 & 1 & 0 & -1 \\
0 & 0 & -2 & 0 \\
0 & 0 & 0 & 2 \end{array}\right)
= 1(1)(-2)(2).
\]

\exer{c10.1.6}
(a) \ans When $\lambda = 1$ or $\lambda = -\frac{1}{3}$,
$\det(\lambda A - B) = 0$.

\soln Compute
\[
\begin{array}{rcl}
\det(\lambda A - B) & = & \det\cmatthree{2\lambda - 2}{-\lambda}{0}
{0}{3\lambda + 1}{0}{\lambda}{5\lambda}{3\lambda - 3} \\
& = & 3(\lambda - 1)\det\cmattwo{2(\lambda - 1)}{-\lambda}{0}
{3\lambda + 1} \\
& = & 6(\lambda - 1)^2(3\lambda - 1).
\end{array}
\]

(b) \ans Yes, there exists a vector $x \in \R^3$ such that $Ax = Bx$.

\soln Let $\lambda = 1$.  Then, part (a) of this exercise implies that
$\det(A - B) = 0$.  Therefore, there exists a vector $x$ such that
$(A - B)x = 0$, that is, $Ax = Bx$.  You can solve this equation for $x$
to obtain $x = (0,0,1)^t$.

\exer{c10.1.a7b}
By swapping rows $1$ and $3$ of matrix $B$, we find that:
\[
\det\matthree{0}{0}{1}{0}{1}{0}{1}{0}{0} =
-\det\matthree{1}{0}{0}{0}{1}{0}{0}{0}{1} = -1.
\]

\exer{c10.1.b7b}
$B_{11} = \matthree{7}{-2}{10}{0}{0}{-1}{4}{2}{-10}$;
$B_{23} = \matthree{0}{2}{5}{0}{0}{-1}{3}{4}{-10}$;
$B_{43} = \matthree{0}{2}{5}{-1}{7}{10}{0}{0}{-1}$.

\exer{c10.1.c8}
By Proposition~\ref{P:ERO}, every
elementary row operation on $A$ can be represented by an invertible $n
\times n$ matrix $E$.  That is, the matrix $EA$ is row equivalent to
$A$.  If $A$ and $B$ are row equivalent, then there exist matrices
$E_j$ such that $B = E_k\ldots E_1A$.  The product of invertible $n
\times n$ matrices is an invertible $n \times n$ matrix.  Thus $P =
E_k\ldots E_1$ is an invertible $n \times n$ matrix such that $B =
PA$.



\subsection*{Section~\protect{\ref{S:eig}} Eigenvalues}
\rhead{S:eig}{EIGENVALUES}

\exer{c10.2.1a}
\ans The characteristic polynomial of $A$ is $p_A(\lambda) =
-\lambda^3 + 2\lambda^2 + \lambda - 2$, and the eigenvalues are
$\lambda_1 = 1$, $\lambda_2 = -1$, and $\lambda_3 = 2$.

\soln Compute:
\[
\begin{array}{rcl}
p_A(\lambda) & = & \det(A - \lambda I_3) \\
& = & \cmatthree{-9 - \lambda}{-2}{-10}{3}{2 - \lambda}{3}
{8}{2}{9 - \lambda} \\
& = & (-9 - \lambda)\det\cmattwo{2 - \lambda}{3}{2}{9 - \lambda}
- 3\det\cmattwo{-2}{-10}{2}{9 - \lambda} + \\
& & 8\det\cmattwo{-2}{-10}{2 - \lambda}{3} \\
& = & (-9 - \lambda)(\lambda^2 - 11\lambda + 12)
- 3(2 + 2\lambda) + 8(14 - 10\lambda) \\
& = & -\lambda^3 + 2\lambda^2 + \lambda - 2 \\
& = & (\lambda - 1)(\lambda + 1)(\lambda - 2). \end{array}
\]
The eigenvalues of $A$ are the roots of the characteristic polynomial.

\exer{c10.2.2}
\ans A basis for the eigenspace of $A$ corresponding to the eigenvalue
$\lambda = 2$ is:
\[
\left\{\vecthree{-1}{1}{0}, \vecthree{1}{0}{1}\right\}.
\]

\soln First, find all eigenvectors of $A$ with eigenvalue
$\lambda = 2$, that is, all vectors $v = (v_1,v_2,v_3)$ such
that $(A - 2I_3)v = 0$.  Solve the system
\[
\matthree{1}{1}{-1}{-1}{-1}{1}{2}{2}{-2}\vecthree{v_1}{v_2}{v_3} =
\vecthree{0}{0}{0}.
\]
All solutions $v$ to this system satisfy $v_1 = v_3 - v_2$.  Thus:
\[
v = \cvecthree{v_3 - v_2}{v_2}{v_3} = v_2\vecthree{-1}{1}{0} +
v_3\vecthree{1}{0}{1}.
\]
Therefore, the vectors $(-1,1,0)^t$ and $(1,0,1)^t$ form a basis
for this eigenspace.

\exer{c10.2.4}
(a) \ans The eigenvalues of $A$ are $\lambda_1 = 3$ and $\lambda_2 = -2$,
with corresponding eigenvectors $v_1 = (1,-1)^t$ and
$v_2 = (1,-2)^t$, respectively.

\soln The characteristic polynomial is $p_A(\lambda) =
\lambda^2 - \lambda - 6 = (\lambda - 3)(\lambda + 2)$.  Then, solve
$Av = \lambda v$ for each eigenvalue to find the corresponding eigenvectors.

(b) Two linearly independent vectors in $\R^2$ form a basis for $\R^2$.
Note that $v_1 \neq \alpha v_2$ for any scalar $\alpha$.  Therefore,
$v_1$ and $v_2$ form a basis for $\R^2$.

(c) \ans The coordinates of $(x_1,x_2)$ in the basis $\{v_1,v_2\}$ are
$(2x_1 + x_2, -x_1 - x_2)$. 

\soln Find $\alpha_1$ and $\alpha_2$ such that $\alpha_1v_1 +
\alpha_2v_2 = (x_1,x_2)^t$.  That is, solve:
\[
\mattwo{1}{1}{-1}{-2}\vectwo{\alpha_1}{\alpha_2} = \vectwo{x_1}{x_2}
\]
to obtain $\alpha_1 = 2x_1 + x_2$ and $\alpha_2 = -x_1 - x_2$.

\exer{c10.2.6}
We are given $A^2 + A + I_n = 0$.  Therefore, $I_n = -A^2 - A =
A(-A - I_n)$.  Thus, $A^{-1} = -A - I_n$ exists.

\exer{c10.2.7b} \ans The statement is false.

\soln For example, let
\[
A = \mattwo{1}{-1}{0}{1} \AND B = \mattwo{1}{-1}{2}{0}.
\]
Then $\trace(A)\trace(B) = 2(1) = 2$, and $\trace(AB) = -1$.

\exer{c10.2.9a}
(a) By calculation in \Matlab using the {\tt eig}, {\tt trace}, and
{\tt poly} commands, the eigenvalues of $A$ are 
\[
\lambda = -0.5861 \pm 20.2517, \quad
\lambda = -12.9416, \quad
\lambda = -9.1033, \AND
\lambda = 5.2171.
\]
The trace of $A$ is $-18$.  The characteristic polynomial of $A$ is
\[
p_A = \lambda^5 + 18\lambda^4 + 433\lambda^3 + 6296\lambda^2 +
429\lambda - 252292.
\]
Note that in order to obtain an accurate value for the characteristic
polynomial, it may be necessary to use the {\tt format} command.

(b) Theorem~\ref{T:inveig} states that the eigenvalues of $A^{-1}$ are
the inverses of the eigenvalues of $A$.  In \Matlab, compute
\begin{verbatim}
eig(inv(A)) =
  -0.1098    
  -0.0773    
  -0.0014 + 0.0493i
  -0.0014 - 0.0493i
   0.1917
\end{verbatim}
Then, compute the inverse of each eigenvalue of $A$ to find that if
$\lambda$ is an eigenvalue of $A$, then $\lambda^{-1}$ is indeed an
eigenvalue of $A^{-1}$. 

\exer{c10.2.10}
\ans The matrix $B$ is the zero matrix.

\soln First use \Matlab to compute
$p_A(\lambda) = \lambda^3 - 9\lambda^2 + 9\lambda - 348$.  Then
$B = p_A(A) = A^3 - 9A^2 + 9A - 348I_3$ is the zero matrix.  To
see why this is true, see the Cayley-Hamilton Theorem
(Theorem~\ref{T:CH}).

