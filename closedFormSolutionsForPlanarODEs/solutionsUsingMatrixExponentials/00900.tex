\documentclass{ximera}
\usepackage{epsfig}

\graphicspath{
  {./}
  {figures/}
}


\usepackage{morewrites}

%\newcounter{ccounter}
%\setcounter{ccounter}{1}
%\newcommand{\Chapter}[1]{\setcounter{chapter}{\arabic{ccounter}}\chapter{#1}\addtocounter{ccounter}{1}}

%\newcommand{\section}[1]{\section{#1}\setcounter{thm}{0}\setcounter{equation}{0}}

%\renewcommand{\theequation}{\arabic{chapter}.\arabic{section}.\arabic{equation}}
%\renewcommand{\thefigure}{\arabic{chapter}.\arabic{figure}}
%\renewcommand{\thetable}{\arabic{chapter}.\arabic{table}}

%\newcommand{\Sec}[2]{\section{#1}\markright{\arabic{ccounter}.\arabic{section}.#2}\setcounter{equation}{0}\setcounter{thm}{0}\setcounter{figure}{0}}

\newcommand{\Sec}[2]{\section{#1}}

\setcounter{secnumdepth}{2}
%\setcounter{secnumdepth}{1} 

%\newcounter{THM}
%\renewcommand{\theTHM}{\arabic{chapter}.\arabic{section}}

\newcommand{\trademark}{{R\!\!\!\!\!\bigcirc}}
%\newtheorem{exercise}{}

\newcommand{\dfield}{{\sf dfield9}}
\newcommand{\pplane}{{\sf pplane9}}

\newcommand{\EXER}{\section*{Exercises}}%\vspace*{0.2in}\hrule\small\setcounter{exercise}{0}}
\newcommand{\CEXER}{}%\vspace{0.08in}\begin{center}Computer Exercises\end{center}}
\newcommand{\TEXER}{} %\vspace{0.08in}\begin{center}Hand Exercises\end{center}}
\newcommand{\AEXER}{} %\vspace{0.08in}\begin{center}Hand Exercises\end{center}}

% BADBAD: \newcommand{\Bbb}{\bf}

\newcommand{\R}{\mbox{$\Bbb{R}$}}
\newcommand{\C}{\mbox{$\Bbb{C}$}}
\newcommand{\Z}{\mbox{$\Bbb{Z}$}}
\newcommand{\N}{\mbox{$\Bbb{N}$}}
\newcommand{\D}{\mbox{{\bf D}}}
\usepackage{amssymb}
%\newcommand{\qed}{\hfill\mbox{\raggedright$\square$} \vspace{1ex}}
%\newcommand{\proof}{\noindent {\bf Proof:} \hspace{0.1in}}

\newcommand{\setmin}{\;\mbox{--}\;}
\newcommand{\Matlab}{{M\small{AT\-LAB}} }
\newcommand{\Matlabp}{{M\small{AT\-LAB}}}
\newcommand{\computer}{\Matlab Instructions}
\newcommand{\half}{\mbox{$\frac{1}{2}$}}
\newcommand{\compose}{\raisebox{.15ex}{\mbox{{\scriptsize$\circ$}}}}
\newcommand{\AND}{\quad\mbox{and}\quad}
\newcommand{\vect}[2]{\left(\begin{array}{c} #1_1 \\ \vdots \\
 #1_{#2}\end{array}\right)}
\newcommand{\mattwo}[4]{\left(\begin{array}{rr} #1 & #2\\ #3
&#4\end{array}\right)}
\newcommand{\mattwoc}[4]{\left(\begin{array}{cc} #1 & #2\\ #3
&#4\end{array}\right)}
\newcommand{\vectwo}[2]{\left(\begin{array}{r} #1 \\ #2\end{array}\right)}
\newcommand{\vectwoc}[2]{\left(\begin{array}{c} #1 \\ #2\end{array}\right)}



\newcommand{\inv}{^{-1}}
\newcommand{\CC}{{\cal C}}
\newcommand{\CCone}{\CC^1}
\newcommand{\Span}{{\rm span}}
\newcommand{\rank}{{\rm rank}}
\newcommand{\trace}{{\rm tr}}
\newcommand{\RE}{{\rm Re}}
\newcommand{\IM}{{\rm Im}}
\newcommand{\nulls}{{\rm null\;space}}

\newcommand{\dps}{\displaystyle}
\newcommand{\arraystart}{\renewcommand{\arraystretch}{1.8}}
\newcommand{\arrayfinish}{\renewcommand{\arraystretch}{1.2}}
\newcommand{\Start}[1]{\vspace{0.08in}\noindent {\bf Section~\ref{#1}}}
\newcommand{\exer}[1]{\noindent {\bf \ref{#1}}}
\newcommand{\ans}{}
\newcommand{\matthree}[9]{\left(\begin{array}{rrr} #1 & #2 & #3 \\ #4 & #5 & #6
\\ #7 & #8 & #9\end{array}\right)}
\newcommand{\cvectwo}[2]{\left(\begin{array}{c} #1 \\ #2\end{array}\right)}
\newcommand{\cmatthree}[9]{\left(\begin{array}{ccc} #1 & #2 & #3 \\ #4 & #5 &
#6 \\ #7 & #8 & #9\end{array}\right)}
\newcommand{\vecthree}[3]{\left(\begin{array}{r} #1 \\ #2 \\
#3\end{array}\right)}
\newcommand{\cvecthree}[3]{\left(\begin{array}{c} #1 \\ #2 \\
#3\end{array}\right)}
\newcommand{\cmattwo}[4]{\left(\begin{array}{cc} #1 & #2\\ #3
&#4\end{array}\right)}

\newcommand{\Matrix}[1]{\ensuremath{\left(\begin{array}{rrrrrrrrrrrrrrrrrr} #1 \end{array}\right)}}

\newcommand{\Matrixc}[1]{\ensuremath{\left(\begin{array}{cccccccccccc} #1 \end{array}\right)}}



\renewcommand{\labelenumi}{\theenumi)}
\newenvironment{enumeratea}%
{\begingroup
 \renewcommand{\theenumi}{\alph{enumi}}
 \renewcommand{\labelenumi}{(\theenumi)}
 \begin{enumerate}}
 {\end{enumerate}\endgroup}



\newcounter{help}
\renewcommand{\thehelp}{\thesection.\arabic{equation}}

%\newenvironment{equation*}%
%{\renewcommand\endequation{\eqno (\theequation)* $$}%
%   \begin{equation}}%
%   {\end{equation}\renewcommand\endequation{\eqno \@eqnnum
%$$\global\@ignoretrue}}

%\input{psfig.tex}

\author{Martin Golubitsky and Michael Dellnitz}

%\newenvironment{matlabEquation}%
%{\renewcommand\endequation{\eqno (\theequation*) $$}%
%   \begin{equation}}%
%   {\end{equation}\renewcommand\endequation{\eqno \@eqnnum
% $$\global\@ignoretrue}}

\newcommand{\soln}{\textbf{Solution:} }
\newcommand{\exercap}[1]{\centerline{Figure~\ref{#1}}}
\newcommand{\exercaptwo}[1]{\centerline{Figure~\ref{#1}a\hspace{2.1in}
Figure~\ref{#1}b}}
\newcommand{\exercapthree}[1]{\centerline{Figure~\ref{#1}a\hspace{1.2in}
Figure~\ref{#1}b\hspace{1.2in}Figure~\ref{#1}c}}
\newcommand{\para}{\hspace{0.4in}}

\renewenvironment{solution}{\suppress}{\endsuppress}

\ifxake
\newenvironment{matlabEquation}{\begin{equation}}{\end{equation}}
\else
\newenvironment{matlabEquation}%
{\let\oldtheequation\theequation\renewcommand{\theequation}{\oldtheequation*}\begin{equation}}%
  {\end{equation}\let\theequation\oldtheequation}
\fi

\makeatother

\begin{document}

\noindent {\bf Remark:}  The completion of Exercises~\ref{c6.2.7} and
\ref{c6.2.8} constitutes a proof that the infinite series definition of
the matrix exponential is a convergent series for all $n\times n$ matrices.

\begin{exercise}  \label{c6.2.7}
Let $A=(a_{ij})$ be an $n\times n$ matrix.  Define
\[
||A||_m = \max_{1\leq i\leq n} (|a_{i1}|+\cdots+|a_{in}|)
= \max_{1\leq i\leq n} \left(\sum_{j=1}^n|a_{ij}|\right).
\]
That is, to compute $||A||_m$, first sum the absolute values of the entries
in each row of $A$, and then take the maximum of these sums.  Prove that:
\[
||AB||_m \leq ||A||_m ||B||_m.
\]
{\bf Hint:} Begin by noting that
\begin{align*}
||AB||_m &=
\max_{1\leq i\leq n}\left(\sum_{j=1}^n\left|\sum_{k=1}^na_{ik}b_{kj}\right|
\right)\leq \max_{1\leq i\leq n}\left(\sum_{j=1}^n\sum_{k=1}^n\left|a_{ik}b_{kj}
           \right|\right) \\
  &= \max_{1\leq i\leq n}\left(\sum_{k=1}^n\sum_{j=1}^n
\left|a_{ik}b_{kj}\right|\right).
\end{align*}

\begin{solution}
Using the hint from the text:
\[
||AB||_m \leq \max_{1\leq i\leq n}
\left(\sum_k\sum_j\left|a_{ik}b_{kj}\right|\right) \leq
\max_{1\leq i\leq n} \left(\sum_k|a_{ik}|\sum_j|b_{kj}|\right).
\]
Then, because $\sum_j|b_{kj}|$ is the sum of the entries in the $k^{th}$
row of $B$:
\[
\max_{1\leq i\leq n}
\left(\sum_k|a_{ik}|\sum_j|b_{kj}|\right) \leq
\max_{1 \leq i \leq n}
\left(||B||_m \sum_k|a_{ik}|\right) =
||A||_m ||B||_m.
\]
Thus $||AB||_m \leq ||A||_m ||B||_m$.

\end{solution}
\end{exercise}

\begin{exercise} \label{c6.2.8}
Recall that an infinite series of real numbers
\[
c_1+c_2 +\cdots+c_N + \cdots
\]
converges absolutely if there is a constant $K$ such that for every $N$
the partial sum satisfies:
\[
|c_1| + |c_2| + \cdots + |c_N| \leq K.
\]

Let $A$ be an $n\times n$ matrix.  To prove that the matrix exponential $e^A$
is an absolutely convergent infinite series use Exercise~\ref{c6.2.7} and the
following steps.  Let $a_N$ be the $(i,j)^{th}$ entry in the matrix $A^N$
where $A^0=I_n$.
\begin{itemize}
\item[(a)]  $|a_N| \leq ||A^N||_m$.
\item[(b)]  $||A^N||_m \leq ||A||_m^N$.
\item[(c)]  $|a_0| + |a_1| + \cdots + \frac{1}{N!}|a_N| \leq e^{||A||_m}$.
\end{itemize}

\begin{solution}

(a) Let $a_{ij}$ be the $(i,j)^{th}$ entry in $n \times n$ matrix $A$. 
Then, since $|a_{ij}| \geq 0$ for all $(i,j)$,
\[ |a_{ij}| \leq |a_{i1}| + \cdots + |a_{in}| \leq
\max_i(|a_{i1}| + \cdots + |a_{in}|) = ||A||_m. \]
This statement is valid for the matrix $A^N$, so $|a_N| \leq ||A^N||_m$.

(b) To show this, we use the fact that $||A^N||_m = ||A^{N - 1}A||_m$.
According to Exercise~\ref{c6.2.7},
\[
||A^{N - 1}A||_m \leq ||A^{N - 1}||_m ||A||_m.
\]
Expanding $||A^N||_m$ in this way, we find that
\[
||A^N||_m \leq ||A||_m \cdots ||A||_m = ||A||^N_m.
\]

(c) According to our result in (a)
\[ |a_0| + |a_1| + \cdots + \frac{1}{N!}|a_N| = \sum_N
\frac{1}{N!}|a_N| \leq \sum_N \frac{1}{N!}||A^N||_m. \]
According to the result in (b)
\[ \sum_N \frac{1}{N!}||A^N||_m \leq \sum_N \frac{1}{N!}||A||^N_m. \]
We know that
\[ \sum_{N = 0}^{\infty} \frac{1}{N!}||A||^N_m = e^{||A||_m}.  \]
Therefore,
\[ \sum_N \frac{1}{N!}|a_N| \leq e^{||A||_m}, \]
so $e^A$ is absolutely convergent.





\end{solution}
\end{exercise}
\end{document}
