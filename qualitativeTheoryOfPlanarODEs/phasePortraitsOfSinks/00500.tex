\documentclass{ximera}
\usepackage{epsfig}

\graphicspath{
  {./}
  {figures/}
}


\usepackage{morewrites}

%\newcounter{ccounter}
%\setcounter{ccounter}{1}
%\newcommand{\Chapter}[1]{\setcounter{chapter}{\arabic{ccounter}}\chapter{#1}\addtocounter{ccounter}{1}}

%\newcommand{\section}[1]{\section{#1}\setcounter{thm}{0}\setcounter{equation}{0}}

%\renewcommand{\theequation}{\arabic{chapter}.\arabic{section}.\arabic{equation}}
%\renewcommand{\thefigure}{\arabic{chapter}.\arabic{figure}}
%\renewcommand{\thetable}{\arabic{chapter}.\arabic{table}}

%\newcommand{\Sec}[2]{\section{#1}\markright{\arabic{ccounter}.\arabic{section}.#2}\setcounter{equation}{0}\setcounter{thm}{0}\setcounter{figure}{0}}

\newcommand{\Sec}[2]{\section{#1}}

\setcounter{secnumdepth}{2}
%\setcounter{secnumdepth}{1} 

%\newcounter{THM}
%\renewcommand{\theTHM}{\arabic{chapter}.\arabic{section}}

\newcommand{\trademark}{{R\!\!\!\!\!\bigcirc}}
%\newtheorem{exercise}{}

\newcommand{\dfield}{{\sf dfield9}}
\newcommand{\pplane}{{\sf pplane9}}

\newcommand{\EXER}{\section*{Exercises}}%\vspace*{0.2in}\hrule\small\setcounter{exercise}{0}}
\newcommand{\CEXER}{}%\vspace{0.08in}\begin{center}Computer Exercises\end{center}}
\newcommand{\TEXER}{} %\vspace{0.08in}\begin{center}Hand Exercises\end{center}}
\newcommand{\AEXER}{} %\vspace{0.08in}\begin{center}Hand Exercises\end{center}}

% BADBAD: \newcommand{\Bbb}{\bf}

\newcommand{\R}{\mbox{$\Bbb{R}$}}
\newcommand{\C}{\mbox{$\Bbb{C}$}}
\newcommand{\Z}{\mbox{$\Bbb{Z}$}}
\newcommand{\N}{\mbox{$\Bbb{N}$}}
\newcommand{\D}{\mbox{{\bf D}}}
\usepackage{amssymb}
%\newcommand{\qed}{\hfill\mbox{\raggedright$\square$} \vspace{1ex}}
%\newcommand{\proof}{\noindent {\bf Proof:} \hspace{0.1in}}

\newcommand{\setmin}{\;\mbox{--}\;}
\newcommand{\Matlab}{{M\small{AT\-LAB}} }
\newcommand{\Matlabp}{{M\small{AT\-LAB}}}
\newcommand{\computer}{\Matlab Instructions}
\newcommand{\half}{\mbox{$\frac{1}{2}$}}
\newcommand{\compose}{\raisebox{.15ex}{\mbox{{\scriptsize$\circ$}}}}
\newcommand{\AND}{\quad\mbox{and}\quad}
\newcommand{\vect}[2]{\left(\begin{array}{c} #1_1 \\ \vdots \\
 #1_{#2}\end{array}\right)}
\newcommand{\mattwo}[4]{\left(\begin{array}{rr} #1 & #2\\ #3
&#4\end{array}\right)}
\newcommand{\mattwoc}[4]{\left(\begin{array}{cc} #1 & #2\\ #3
&#4\end{array}\right)}
\newcommand{\vectwo}[2]{\left(\begin{array}{r} #1 \\ #2\end{array}\right)}
\newcommand{\vectwoc}[2]{\left(\begin{array}{c} #1 \\ #2\end{array}\right)}



\newcommand{\inv}{^{-1}}
\newcommand{\CC}{{\cal C}}
\newcommand{\CCone}{\CC^1}
\newcommand{\Span}{{\rm span}}
\newcommand{\rank}{{\rm rank}}
\newcommand{\trace}{{\rm tr}}
\newcommand{\RE}{{\rm Re}}
\newcommand{\IM}{{\rm Im}}
\newcommand{\nulls}{{\rm null\;space}}

\newcommand{\dps}{\displaystyle}
\newcommand{\arraystart}{\renewcommand{\arraystretch}{1.8}}
\newcommand{\arrayfinish}{\renewcommand{\arraystretch}{1.2}}
\newcommand{\Start}[1]{\vspace{0.08in}\noindent {\bf Section~\ref{#1}}}
\newcommand{\exer}[1]{\noindent {\bf \ref{#1}}}
\newcommand{\ans}{}
\newcommand{\matthree}[9]{\left(\begin{array}{rrr} #1 & #2 & #3 \\ #4 & #5 & #6
\\ #7 & #8 & #9\end{array}\right)}
\newcommand{\cvectwo}[2]{\left(\begin{array}{c} #1 \\ #2\end{array}\right)}
\newcommand{\cmatthree}[9]{\left(\begin{array}{ccc} #1 & #2 & #3 \\ #4 & #5 &
#6 \\ #7 & #8 & #9\end{array}\right)}
\newcommand{\vecthree}[3]{\left(\begin{array}{r} #1 \\ #2 \\
#3\end{array}\right)}
\newcommand{\cvecthree}[3]{\left(\begin{array}{c} #1 \\ #2 \\
#3\end{array}\right)}
\newcommand{\cmattwo}[4]{\left(\begin{array}{cc} #1 & #2\\ #3
&#4\end{array}\right)}

\newcommand{\Matrix}[1]{\ensuremath{\left(\begin{array}{rrrrrrrrrrrrrrrrrr} #1 \end{array}\right)}}

\newcommand{\Matrixc}[1]{\ensuremath{\left(\begin{array}{cccccccccccc} #1 \end{array}\right)}}



\renewcommand{\labelenumi}{\theenumi)}
\newenvironment{enumeratea}%
{\begingroup
 \renewcommand{\theenumi}{\alph{enumi}}
 \renewcommand{\labelenumi}{(\theenumi)}
 \begin{enumerate}}
 {\end{enumerate}\endgroup}



\newcounter{help}
\renewcommand{\thehelp}{\thesection.\arabic{equation}}

%\newenvironment{equation*}%
%{\renewcommand\endequation{\eqno (\theequation)* $$}%
%   \begin{equation}}%
%   {\end{equation}\renewcommand\endequation{\eqno \@eqnnum
%$$\global\@ignoretrue}}

%\input{psfig.tex}

\author{Martin Golubitsky and Michael Dellnitz}

%\newenvironment{matlabEquation}%
%{\renewcommand\endequation{\eqno (\theequation*) $$}%
%   \begin{equation}}%
%   {\end{equation}\renewcommand\endequation{\eqno \@eqnnum
% $$\global\@ignoretrue}}

\newcommand{\soln}{\textbf{Solution:} }
\newcommand{\exercap}[1]{\centerline{Figure~\ref{#1}}}
\newcommand{\exercaptwo}[1]{\centerline{Figure~\ref{#1}a\hspace{2.1in}
Figure~\ref{#1}b}}
\newcommand{\exercapthree}[1]{\centerline{Figure~\ref{#1}a\hspace{1.2in}
Figure~\ref{#1}b\hspace{1.2in}Figure~\ref{#1}c}}
\newcommand{\para}{\hspace{0.4in}}

\renewenvironment{solution}{\suppress}{\endsuppress}

\ifxake
\newenvironment{matlabEquation}{\begin{equation}}{\end{equation}}
\else
\newenvironment{matlabEquation}%
{\let\oldtheequation\theequation\renewcommand{\theequation}{\oldtheequation*}\begin{equation}}%
  {\end{equation}\let\theequation\oldtheequation}
\fi

\makeatother

\begin{document}

\noindent In Exercises~\ref{c6.8.2a} -- \ref{c6.8.2d}, find a $2\times 2$
matrix $C$ so that the given statement is satisfied.
\begin{exercise} \label{c6.8.2a}
The differential equation $\dot{X}=CX$ has a saddle at the origin with
unstable orbit in the direction $(2,3)$.

\begin{solution}

\ans One possible solution is $C = \mattwo{1}{0}{3}{-1}$.

\soln Since the system has an invariant manifold in the direction $(2,3)$,
$v_1 = (2,3)$.  Let $\lambda_1$ be the eigenvalue associated to $v_1$.
Since the manifold is unstable, $\lambda_1 > 0$.  The requirement
that the system has a saddle at the origin implies that $C$ has a second
eigenvalue $\lambda_2 < 0$.  Choose a second eigenvector $v_2$ independent
of $v_1$, such as $v_2 = (0,1)$ and choose eigenvalues, for example:
$\lambda_1 = 1$ and $\lambda_2 = -1$.

\para The matrix $C$ has two real eigenvalues.  Thus, by
Theorem~\ref{T:putinform}, $C$ is similar
to the diagonal matrix $D$ which has $\lambda_1$ and $\lambda_2$ as
entries along the main diagonal.  Therefore, $C = PDP^{-1}$, where $P$
is the matrix whose columns are the eigenvectors of $C$.  Thus,
\[
C = \mattwo{2}{0}{3}{1}\mattwo{1}{0}{0}{-1}\mattwo{\frac{1}{2}}{0}
{-\frac{3}{2}}{1} = \mattwo{1}{0}{3}{-1}. \]

\end{solution}
\end{exercise}
\begin{exercise} \label{c6.8.2b}
The differential equation $\dot{X}=CX$ has a spiral sink at the origin
where solutions decay to the origin at rate $\sigma=-0.5$.

\begin{solution}

\ans One possible matrix is $C = \mattwo{-0.5}{-1}{1}{-0.5}$.

\soln Since the origin is a spiral, the eigenvalues of $C$ are a complex
conjugate pair, $lambda_1 = \sigma + \tau i$ and $lambda_2 = \sigma -
\tau i$.  Since the origin is a sink, $\sigma$ is negative, and determines
the rate at which solutions decay to the origin, so $\sigma = -0.5$.
From Section~\ref{S:evchp}, we know that the matrix
\[ \mattwo{\sigma}{-\tau}{\tau}{\sigma} \]
has complex conjugate eigenvalues $\sigma \pm \tau$ if $\tau \neq 0$.
Thus, this matrix is a solution for any nonzero $\tau$.  For example,
let $\tau = 1$.

\end{solution}
\end{exercise}
\begin{exercise} \label{c6.8.2c}
The differential equation $\dot{X}=CX$ has an improper nodal source at the
origin with trajectories approaching the origin tangent to the $y$ axis.

\begin{solution}

\ans One possible matrix is $C = \mattwo{1}{0}{1}{1}$.

\soln Since the system has an improper nodal source at the origin, there
are two real equal positive eigenvalues, $\lambda_1 = \lambda_2 > 0$,
associated to a single eigenvector, $v$.  Let $\lambda = 1$ be an
eigenvalue.  The trajectories of an improper nodal system approach the
origin tangent to the eigenvector, so, since trajectories approach
tangent to the $y$-axis, the eigenvector is $v = (0,1)$.  By
Theorem~\ref{T:putinform}, if $C$ has one
real eigenvector, then $C = PDP^1$, where
\[ D = \mattwo{\lambda}{1}{0}{\lambda} \AND P = (v|w) \]
such that $Cv = v + \lambda w$.  We can assign an arbitrary value to
$w$.  For example, let $w = (1,0)$.  Then,
\[ C = \mattwo{0}{1}{1}{0}\mattwo{1}{1}{0}{1}\mattwo{0}{1}{1}{0} =
\mattwo{1}{0}{1}{1}. \]

\end{solution}
\end{exercise}
\begin{exercise} \label{c6.8.2d}
The differential equation $\dot{X}=CX$ has a nodal sink at the origin with
trajectories approaching the origin tangent to the line $y=x$.

\begin{solution}

\ans One possible matrix is $C = \mattwo{-1}{0}{1}{-2}$.

\soln Since the system has a nodal sink, the eigenvalues are real, unequal
and negative.  That is, $\lambda_2 < \lambda_1 < 0$.  The trajectories
of a nodal system approach the origin tangent to the eigenvector
associated to the eigenvalue with smaller absolute value.  Thus, for
this system, $v_1 = (1,1)$ is an eigenvalue associated to $\lambda_1$.
Arbitrarily assign values to $\lambda_1$, $\lambda_2$ and $v_2$, with
the given restrictions.  For example, let $\lambda_1 = -1$, $\lambda_2
= -2$ and $v_2 = (0,1)$.  Therefore, by
Theorem~\ref{T:putinform}, $C = PDP^{-1}$, where
\[ D = \mattwo{-1}{0}{0}{-2} \AND P = (v_1|v_2). \]
Thus,
\[ C = \mattwo{1}{0}{1}{1}\mattwo{-1}{0}{0}{-2}\mattwo{1}{0}{-1}{1} =
\mattwo{-1}{0}{1}{-2}. \]

\end{solution}
\end{exercise}
\end{document}
