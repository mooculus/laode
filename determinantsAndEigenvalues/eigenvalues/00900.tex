\documentclass{ximera}
\usepackage{epsfig}

\graphicspath{
  {./}
  {figures/}
}


\usepackage{morewrites}

%\newcounter{ccounter}
%\setcounter{ccounter}{1}
%\newcommand{\Chapter}[1]{\setcounter{chapter}{\arabic{ccounter}}\chapter{#1}\addtocounter{ccounter}{1}}

%\newcommand{\section}[1]{\section{#1}\setcounter{thm}{0}\setcounter{equation}{0}}

%\renewcommand{\theequation}{\arabic{chapter}.\arabic{section}.\arabic{equation}}
%\renewcommand{\thefigure}{\arabic{chapter}.\arabic{figure}}
%\renewcommand{\thetable}{\arabic{chapter}.\arabic{table}}

%\newcommand{\Sec}[2]{\section{#1}\markright{\arabic{ccounter}.\arabic{section}.#2}\setcounter{equation}{0}\setcounter{thm}{0}\setcounter{figure}{0}}

\newcommand{\Sec}[2]{\section{#1}}

\setcounter{secnumdepth}{2}
%\setcounter{secnumdepth}{1} 

%\newcounter{THM}
%\renewcommand{\theTHM}{\arabic{chapter}.\arabic{section}}

\newcommand{\trademark}{{R\!\!\!\!\!\bigcirc}}
%\newtheorem{exercise}{}

\newcommand{\dfield}{{\sf dfield9}}
\newcommand{\pplane}{{\sf pplane9}}

\newcommand{\EXER}{\section*{Exercises}}%\vspace*{0.2in}\hrule\small\setcounter{exercise}{0}}
\newcommand{\CEXER}{}%\vspace{0.08in}\begin{center}Computer Exercises\end{center}}
\newcommand{\TEXER}{} %\vspace{0.08in}\begin{center}Hand Exercises\end{center}}
\newcommand{\AEXER}{} %\vspace{0.08in}\begin{center}Hand Exercises\end{center}}

% BADBAD: \newcommand{\Bbb}{\bf}

\newcommand{\R}{\mbox{$\Bbb{R}$}}
\newcommand{\C}{\mbox{$\Bbb{C}$}}
\newcommand{\Z}{\mbox{$\Bbb{Z}$}}
\newcommand{\N}{\mbox{$\Bbb{N}$}}
\newcommand{\D}{\mbox{{\bf D}}}
\usepackage{amssymb}
%\newcommand{\qed}{\hfill\mbox{\raggedright$\square$} \vspace{1ex}}
%\newcommand{\proof}{\noindent {\bf Proof:} \hspace{0.1in}}

\newcommand{\setmin}{\;\mbox{--}\;}
\newcommand{\Matlab}{{M\small{AT\-LAB}} }
\newcommand{\Matlabp}{{M\small{AT\-LAB}}}
\newcommand{\computer}{\Matlab Instructions}
\newcommand{\half}{\mbox{$\frac{1}{2}$}}
\newcommand{\compose}{\raisebox{.15ex}{\mbox{{\scriptsize$\circ$}}}}
\newcommand{\AND}{\quad\mbox{and}\quad}
\newcommand{\vect}[2]{\left(\begin{array}{c} #1_1 \\ \vdots \\
 #1_{#2}\end{array}\right)}
\newcommand{\mattwo}[4]{\left(\begin{array}{rr} #1 & #2\\ #3
&#4\end{array}\right)}
\newcommand{\mattwoc}[4]{\left(\begin{array}{cc} #1 & #2\\ #3
&#4\end{array}\right)}
\newcommand{\vectwo}[2]{\left(\begin{array}{r} #1 \\ #2\end{array}\right)}
\newcommand{\vectwoc}[2]{\left(\begin{array}{c} #1 \\ #2\end{array}\right)}



\newcommand{\inv}{^{-1}}
\newcommand{\CC}{{\cal C}}
\newcommand{\CCone}{\CC^1}
\newcommand{\Span}{{\rm span}}
\newcommand{\rank}{{\rm rank}}
\newcommand{\trace}{{\rm tr}}
\newcommand{\RE}{{\rm Re}}
\newcommand{\IM}{{\rm Im}}
\newcommand{\nulls}{{\rm null\;space}}

\newcommand{\dps}{\displaystyle}
\newcommand{\arraystart}{\renewcommand{\arraystretch}{1.8}}
\newcommand{\arrayfinish}{\renewcommand{\arraystretch}{1.2}}
\newcommand{\Start}[1]{\vspace{0.08in}\noindent {\bf Section~\ref{#1}}}
\newcommand{\exer}[1]{\noindent {\bf \ref{#1}}}
\newcommand{\ans}{}
\newcommand{\matthree}[9]{\left(\begin{array}{rrr} #1 & #2 & #3 \\ #4 & #5 & #6
\\ #7 & #8 & #9\end{array}\right)}
\newcommand{\cvectwo}[2]{\left(\begin{array}{c} #1 \\ #2\end{array}\right)}
\newcommand{\cmatthree}[9]{\left(\begin{array}{ccc} #1 & #2 & #3 \\ #4 & #5 &
#6 \\ #7 & #8 & #9\end{array}\right)}
\newcommand{\vecthree}[3]{\left(\begin{array}{r} #1 \\ #2 \\
#3\end{array}\right)}
\newcommand{\cvecthree}[3]{\left(\begin{array}{c} #1 \\ #2 \\
#3\end{array}\right)}
\newcommand{\cmattwo}[4]{\left(\begin{array}{cc} #1 & #2\\ #3
&#4\end{array}\right)}

\newcommand{\Matrix}[1]{\ensuremath{\left(\begin{array}{rrrrrrrrrrrrrrrrrr} #1 \end{array}\right)}}

\newcommand{\Matrixc}[1]{\ensuremath{\left(\begin{array}{cccccccccccc} #1 \end{array}\right)}}



\renewcommand{\labelenumi}{\theenumi)}
\newenvironment{enumeratea}%
{\begingroup
 \renewcommand{\theenumi}{\alph{enumi}}
 \renewcommand{\labelenumi}{(\theenumi)}
 \begin{enumerate}}
 {\end{enumerate}\endgroup}



\newcounter{help}
\renewcommand{\thehelp}{\thesection.\arabic{equation}}

%\newenvironment{equation*}%
%{\renewcommand\endequation{\eqno (\theequation)* $$}%
%   \begin{equation}}%
%   {\end{equation}\renewcommand\endequation{\eqno \@eqnnum
%$$\global\@ignoretrue}}

%\input{psfig.tex}

\author{Martin Golubitsky and Michael Dellnitz}

%\newenvironment{matlabEquation}%
%{\renewcommand\endequation{\eqno (\theequation*) $$}%
%   \begin{equation}}%
%   {\end{equation}\renewcommand\endequation{\eqno \@eqnnum
% $$\global\@ignoretrue}}

\newcommand{\soln}{\textbf{Solution:} }
\newcommand{\exercap}[1]{\centerline{Figure~\ref{#1}}}
\newcommand{\exercaptwo}[1]{\centerline{Figure~\ref{#1}a\hspace{2.1in}
Figure~\ref{#1}b}}
\newcommand{\exercapthree}[1]{\centerline{Figure~\ref{#1}a\hspace{1.2in}
Figure~\ref{#1}b\hspace{1.2in}Figure~\ref{#1}c}}
\newcommand{\para}{\hspace{0.4in}}

\renewenvironment{solution}{\suppress}{\endsuppress}

\ifxake
\newenvironment{matlabEquation}{\begin{equation}}{\end{equation}}
\else
\newenvironment{matlabEquation}%
{\let\oldtheequation\theequation\renewcommand{\theequation}{\oldtheequation*}\begin{equation}}%
  {\end{equation}\let\theequation\oldtheequation}
\fi

\makeatother

\begin{document}

\CEXER

\noindent In Exercises~\ref{c10.2.9a} -- \ref{c10.2.9b}, use \Matlab to 
compute (a) the eigenvalues, traces, and characteristic polynomials of 
the given matrix.  (b) Use the results from part (a) to confirm 
Theorems~\ref{T:inveig} and \ref{T:tracen}.
\begin{exercise} \label{c10.2.9a}
\begin{matlabEquation}\label{find-eigenvalues}
A=\left( \begin{array}{rrrrr}
      -12 & -19 &  -3 &  14 &   0\\
      -12 &  10 &  14 & -19 &   8\\
        4 &  -2 &   1 &   7 &  -3\\
       -9 &  17 & -12 &  -5 &  -8\\
      -12 &  -1 &   7 &  13 & -12
\end{array} \right).
\end{matlabEquation}

\begin{solution}

(a) By calculation in \Matlab using the {\tt eig}, {\tt trace}, and
{\tt poly} commands, the eigenvalues of $A$ are 
\[
\lambda = -0.5861 \pm 20.2517, \quad
\lambda = -12.9416, \quad
\lambda = -9.1033, \AND
\lambda = 5.2171.
\]
The trace of $A$ is $-18$.  The characteristic polynomial of $A$ is
\[
p_A = \lambda^5 + 18\lambda^4 + 433\lambda^3 + 6296\lambda^2 +
429\lambda - 252292.
\]
Note that in order to obtain an accurate value for the characteristic
polynomial, it may be necessary to use the {\tt format} command.

(b) Theorem~\ref{T:inveig} states that the eigenvalues of $A^{-1}$ are
the inverses of the eigenvalues of $A$.  In \Matlab, compute
\begin{verbatim}
eig(inv(A)) =
  -0.1098    
  -0.0773    
  -0.0014 + 0.0493i
  -0.0014 - 0.0493i
   0.1917
\end{verbatim}
Then, compute the inverse of each eigenvalue of $A$ to find that if
$\lambda$ is an eigenvalue of $A$, then $\lambda^{-1}$ is indeed an
eigenvalue of $A^{-1}$. 

\end{solution}
\end{exercise}
\begin{exercise} \label{c10.2.9b}
\begin{matlabEquation}\label{compute-more-eigenvalues}
B=\left( \begin{array}{rrrrrr}
      -12 &  -5 &  13 &  -6 & -5 &  12\\
        7 &  14 &   6 &   1 &  8 &  18\\
       -8 &  14 &  13 &   9 &  2 &   1\\
        2 &   4 &   6 &  -8 & -2 &  15\\
      -14 &   0 &  -6 &  14 &  8 & -13\\
        8 &  16 &  -8 &   3 &  5 &  19
\end{array} \right).
\end{matlabEquation}

\begin{solution}

(a) The eigenvalues of $B$ are
\[
\lambda = 32.6273, \quad
\lambda = -12.1564 \pm 5.8787i, \quad
\lambda = 18.0009, \quad
\lambda = -3.4878, \AND 
\lambda = 11.1723.
\]
The trace of $B$ is $34$.  The characteristic polynomial of $B$ is
\[
p_B = \lambda^6 - 24\lambda^5 - 298\lambda^4 + 9618\lambda^3
+ 86273\lambda^2 - 1019656\lambda - 4172976.
\]

(b) Theorem~\ref{T:inveig} states that the eigenvalues of $A^{-1}$ are
the inverses of the eigenvalues of $A$.  In \Matlab, compute
\begin{verbatim}
eig(inv(B)) =
  -0.2867
  -0.0667 + 0.0322i
  -0.0667 - 0.0322i
   0.0895
   0.0556
   0.0306
\end{verbatim}
Then, compute the inverse of each eigenvalue of $B$ to find that if
$\lambda$ is an eigenvalue of $B$, then $\lambda^{-1}$
is indeed an eigenvalue of $B^{-1}$.

\end{solution}
\end{exercise}
\end{document}
