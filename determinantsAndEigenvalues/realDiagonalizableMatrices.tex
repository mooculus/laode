\documentclass{ximera}

\usepackage{epsfig}

\graphicspath{
  {./}
  {figures/}
}

\usepackage{epstopdf}
%\usepackage{ulem}
\usepackage[normalem]{ulem}

\epstopdfsetup{outdir=./}

\usepackage{morewrites}
\makeatletter
\newcommand\subfile[1]{%
\renewcommand{\input}[1]{}%
\begingroup\skip@preamble\otherinput{#1}\endgroup\par\vspace{\topsep}
\let\input\otherinput}
\makeatother

\newcommand{\EXER}{}
\newcommand{\includeexercises}{\EXER\directlua{dofile(kpse.find_file("exercises","lua"))}}

\newenvironment{computerExercise}{\begin{exercise}}{\end{exercise}}

%\newcounter{ccounter}
%\setcounter{ccounter}{1}
%\newcommand{\Chapter}[1]{\setcounter{chapter}{\arabic{ccounter}}\chapter{#1}\addtocounter{ccounter}{1}}

%\newcommand{\section}[1]{\section{#1}\setcounter{thm}{0}\setcounter{equation}{0}}

%\renewcommand{\theequation}{\arabic{chapter}.\arabic{section}.\arabic{equation}}
%\renewcommand{\thefigure}{\arabic{chapter}.\arabic{figure}}
%\renewcommand{\thetable}{\arabic{chapter}.\arabic{table}}

%\newcommand{\Sec}[2]{\section{#1}\markright{\arabic{ccounter}.\arabic{section}.#2}\setcounter{equation}{0}\setcounter{thm}{0}\setcounter{figure}{0}}
  
\newcommand{\Sec}[2]{\section{#1}}

\setcounter{secnumdepth}{2}
%\setcounter{secnumdepth}{1} 

%\newcounter{THM}
%\renewcommand{\theTHM}{\arabic{chapter}.\arabic{section}}

\newcommand{\trademark}{{R\!\!\!\!\!\bigcirc}}
%\newtheorem{exercise}{}

\newcommand{\dfield}{{\sf SlopeField}}

\newcommand{\pplane}{{\sf PhasePlane}}

\newcommand{\PPLANE}{{\sf PHASEPLANE}}

% BADBAD: \newcommand{\Bbb}{\bf}. % Package amsfonts Warning: Obsolete command \Bbb; \mathbb should be used instead.

\newcommand{\R}{\mbox{$\mathbb{R}$}}
\let\C\relax
\newcommand{\C}{\mbox{$\mathbb{C}$}}
\newcommand{\Z}{\mbox{$\mathbb{Z}$}}
\newcommand{\N}{\mbox{$\mathbb{N}$}}
\newcommand{\D}{\mbox{{\bf D}}}

\newcommand{\WW}{\mathcal{W}}

\usepackage{amssymb}
%\newcommand{\qed}{\hfill\mbox{\raggedright$\square$} \vspace{1ex}}
%\newcommand{\proof}{\noindent {\bf Proof:} \hspace{0.1in}}

\newcommand{\setmin}{\;\mbox{--}\;}
\newcommand{\Matlab}{{M\small{AT\-LAB}} }
\newcommand{\Matlabp}{{M\small{AT\-LAB}}}
\newcommand{\computer}{\Matlab Instructions}
\renewcommand{\computer}{M\small{ATLAB} Instructions}
\newcommand{\half}{\mbox{$\frac{1}{2}$}}
\newcommand{\compose}{\raisebox{.15ex}{\mbox{{\scriptsize$\circ$}}}}
\newcommand{\AND}{\quad\mbox{and}\quad}
\newcommand{\vect}[2]{\left(\begin{array}{c} #1_1 \\ \vdots \\
 #1_{#2}\end{array}\right)}
\newcommand{\mattwo}[4]{\left(\begin{array}{rr} #1 & #2\\ #3
&#4\end{array}\right)}
\newcommand{\mattwoc}[4]{\left(\begin{array}{cc} #1 & #2\\ #3
&#4\end{array}\right)}
\newcommand{\vectwo}[2]{\left(\begin{array}{r} #1 \\ #2\end{array}\right)}
\newcommand{\vectwoc}[2]{\left(\begin{array}{c} #1 \\ #2\end{array}\right)}

\newcommand{\ignore}[1]{}


\newcommand{\inv}{^{-1}}
\newcommand{\CC}{{\cal C}}
\newcommand{\CCone}{\CC^1}
\newcommand{\Span}{{\rm span}}
\newcommand{\rank}{{\rm rank}}
\newcommand{\trace}{{\rm tr}}
\newcommand{\RE}{{\rm Re}}
\newcommand{\IM}{{\rm Im}}
\newcommand{\nulls}{{\rm null\;space}}

\newcommand{\dps}{\displaystyle}
\newcommand{\arraystart}{\renewcommand{\arraystretch}{1.8}}
\newcommand{\arrayfinish}{\renewcommand{\arraystretch}{1.2}}
\newcommand{\Start}[1]{\vspace{0.08in}\noindent {\bf Section~\ref{#1}}}
\newcommand{\exer}[1]{\noindent {\bf \ref{#1}}}
\newcommand{\ans}{\textbf{Answer:} }
\newcommand{\matthree}[9]{\left(\begin{array}{rrr} #1 & #2 & #3 \\ #4 & #5 & #6
\\ #7 & #8 & #9\end{array}\right)}
\newcommand{\cvectwo}[2]{\left(\begin{array}{c} #1 \\ #2\end{array}\right)}
\newcommand{\cmatthree}[9]{\left(\begin{array}{ccc} #1 & #2 & #3 \\ #4 & #5 &
#6 \\ #7 & #8 & #9\end{array}\right)}
\newcommand{\vecthree}[3]{\left(\begin{array}{r} #1 \\ #2 \\
#3\end{array}\right)}
\newcommand{\cvecthree}[3]{\left(\begin{array}{c} #1 \\ #2 \\
#3\end{array}\right)}
\newcommand{\cmattwo}[4]{\left(\begin{array}{cc} #1 & #2\\ #3
&#4\end{array}\right)}

\newcommand{\Matrix}[1]{\ensuremath{\left(\begin{array}{rrrrrrrrrrrrrrrrrr} #1 \end{array}\right)}}

\newcommand{\Matrixc}[1]{\ensuremath{\left(\begin{array}{cccccccccccc} #1 \end{array}\right)}}



\renewcommand{\labelenumi}{\theenumi}
\newenvironment{enumeratea}%
{\begingroup
 \renewcommand{\theenumi}{\alph{enumi}}
 \renewcommand{\labelenumi}{(\theenumi)}
 \begin{enumerate}}
 {\end{enumerate}
 \endgroup}

\newcounter{help}
\renewcommand{\thehelp}{\thesection.\arabic{equation}}

%\newenvironment{equation*}%
%{\renewcommand\endequation{\eqno (\theequation)* $$}%
%   \begin{equation}}%
%   {\end{equation}\renewcommand\endequation{\eqno \@eqnnum
%$$\global\@ignoretrue}}

%\input{psfig.tex}

\author{Martin Golubitsky and Michael Dellnitz}

%\newenvironment{matlabEquation}%
%{\renewcommand\endequation{\eqno (\theequation*) $$}%
%   \begin{equation}}%
%   {\end{equation}\renewcommand\endequation{\eqno \@eqnnum
% $$\global\@ignoretrue}}

\newcommand{\soln}{\textbf{Solution:} }
\newcommand{\exercap}[1]{\centerline{Figure~\ref{#1}}}
\newcommand{\exercaptwo}[1]{\centerline{Figure~\ref{#1}a\hspace{2.1in}
Figure~\ref{#1}b}}
\newcommand{\exercapthree}[1]{\centerline{Figure~\ref{#1}a\hspace{1.2in}
Figure~\ref{#1}b\hspace{1.2in}Figure~\ref{#1}c}}
\newcommand{\para}{\hspace{0.4in}}

\usepackage{ifluatex}
\ifluatex
\ifcsname displaysolutions\endcsname%
\else
\renewenvironment{solution}{\suppress}{\endsuppress}
\fi
\else
\renewenvironment{solution}{}{}
\fi

\ifcsname answer\endcsname
\renewcommand{\answer}{}
\fi

%\ifxake
%\newenvironment{matlabEquation}{\begin{equation}}{\end{equation}}
%\else
\newenvironment{matlabEquation}%
{\let\oldtheequation\theequation\renewcommand{\theequation}{\oldtheequation*}\begin{equation}}%
  {\end{equation}\let\theequation\oldtheequation}
%\fi

\makeatother

\newcommand{\RED}[1]{{\color{red}{#1}}} 


\title{Real Diagonalizable Matrices}

\begin{document}
\begin{abstract}
\end{abstract}
\maketitle

 
\label{S:RDM}

An $n\times n$ matrix is {\em real diagonalizable\/}
\index{real diagonalizable} if it is similar \index{similar} to a
diagonal matrix\index{matrix!diagonal}.  More precisely, 
an $n\times n$ matrix $A$ is
real diagonalizable if there exists an invertible $n\times n$
matrix S such that
\[
D=S\inv AS
\]
is a diagonal matrix.  In this section we investigate when a
matrix is diagonalizable.  In this discussion we assume that all
matrices have real entries.

We begin with the observation that not all matrices are real
diagonalizable.  We saw in Example~\ref{E:triangular} that the
diagonal entries of the diagonal matrix $D$ are the eigenvalues
of $D$. Theorem~\ref{T:similareigen} states that similar
matrices have the same eigenvalues.  Thus if a matrix is real
diagonalizable, then it must have real eigenvalues.  It follows,
for example, that the $2\times 2$ matrix 
\[
\mattwo{0}{-1}{1}{0}
\]
is not real diagonalizable, since its eigenvalues are $\pm i$. 

However, even if a matrix $A$ has real eigenvalues, it need not
be diagonalizable.  For example, the only matrix similar to the
identity matrix $I_n$ is the identity matrix itself.  To verify
this point, calculate
\[
S\inv I_n S = S\inv S = I_n.
\]
Suppose that $A$ is a matrix all of whose eigenvalues are equal
to $1$.  If $A$ is similar to a diagonal matrix $D$, then $D$
must have all of its eigenvalues equal to $1$.  Since the
identity matrix is the only diagonal matrix with all eigenvalues
equal to $1$, $D=I_n$.  So, if $A$ is similar to a diagonal
matrix, it must itself be the identity matrix.  Consider,
however, the $2\times 2$ matrix
\[
A=\mattwo{1}{1}{0}{1}.
\]
Since $A$ is triangular, it follows that both eigenvalues of $A$
are equal to $1$.  Since $A$ is not the identity matrix, it
cannot be diagonalizable. More generally, if $N$ is a nonzero
strictly upper triangular $n\times n$ matrix, then the matrix
$I_n+N$ is not diagonalizable.  \index{matrix!strictly upper
triangular}

These examples show that complex eigenvalues are always
obstructions to real diagonalization and multiple real eigenvalues
are sometimes obstructions to diagonalization.  Indeed, 

\begin{theorem}  \label{T:diagsimple}
Let $A$ be an $n\times n$ matrix with $n$ distinct real
eigenvalues.  
Then $A$ is real diagonalizable\index{real diagonalizable}.
\end{theorem}  \index{eigenvalue}\index{eigenvalue!real!distinct}

There are two ideas in the proof of Theorem~\ref{T:diagsimple}, and 
they are summarized in the following lemmas.

\begin{lemma} \label{L:simpleeigen}
Let $\lambda_1,\ldots,\lambda_k$ be distinct real eigenvalues
for an $n\times n$ matrix $A$.  Let $v_j$ be eigenvectors
associated with the eigenvalue $\lambda_j$.  Then
$\{v_1,\ldots,v_k\}$ is a linearly independent set.
\end {lemma} \index{eigenvector}\index{linearly!independent}

\begin{proof} We prove the lemma by using induction on $k$.  When $k=1$
the proof is simple, since $v_1\neq 0$.  So we can assume that
$\{v_1,\ldots,v_{k-1}\}$ is a linearly independent set. 

Let $\alpha_1,\ldots,\alpha_k$ be scalars such that
\begin{equation}  \label{e:linindep}
\alpha_1 v_1 + \cdots + \alpha_k v_k = 0.
\end{equation}
We must show that all $\alpha_j=0$.

Begin by multiplying both sides of \eqref{e:linindep} by $A$, to
obtain: 
\begin{eqnarray}
0 & = & A(\alpha_1 v_1 + \cdots + \alpha_k v_k) \nonumber \\
& = & \alpha_1 Av_1 + \cdots + \alpha_k Av_k \label{e:linother}\\
& = & \alpha_1 \lambda_1 v_1 + \cdots + \alpha_k \lambda_k v_k.\nonumber
\end{eqnarray}

Now subtract $\lambda_k$ times \eqref{e:linindep} from \eqref{e:linother},
to obtain:
\[
\alpha_1(\lambda_1-\lambda_k)v_1 + \cdots +
\alpha_{k-1}(\lambda_{k-1}-\lambda_k)v_{k-1} = 0.
\]
Since $\{v_1,\ldots,v_{k-1}\}$ is a linearly independent set, it
follows that 
\[
\alpha_j(\lambda_j-\lambda_k)=0,
\]
for $j=1,\ldots,k-1$.  Since all of the eigenvalues are
distinct, $\lambda_j-\lambda_k\neq 0$ and $\alpha_j=0$ for
$j=1,\ldots,k-1$. Substituting this information into
\eqref{e:linindep} yields $\alpha_k v_k=0$.  Since $v_k\neq 0$, 
$\alpha_k$ is also equal to zero.  \end{proof}

\begin{lemma}  \label{L:eigenv-diag}
Let $A$ be an $n\times n$ matrix.  Then $A$ is real diagonalizable if
and only if $A$ has $n$ real linearly independent 
eigenvectors\index{eigenvector!linearly independent}.
\end{lemma}  \index{real diagonalizable}

\begin{proof}  Suppose that $A$ has $n$ linearly independent eigenvectors 
$v_1,\ldots,v_n$.  Let $\lambda_1,\ldots,\lambda_n$ be the 
corresponding eigenvalues of $A$; that is, $Av_j=\lambda_jv_j$.
Let $S=(v_1|\cdots|v_n)$ be the $n\times n$ matrix whose columns are the 
eigenvectors $v_j$.  We claim that $D=S\inv AS$ is a diagonal matrix.
Compute
\begin{align*}
  D&=S\inv AS=S\inv A(v_1|\cdots|v_n)=S\inv(Av_1|\cdots|Av_n)\\
  &=S\inv(\lambda_1v_1|\cdots|\lambda_nv_n).
\end{align*}
It follows that 
\[
D=(\lambda_1S\inv v_1|\cdots|\lambda_nS\inv v_n).
\]
Note that 
\[
S\inv v_j=e_j,
\]
since
\[
Se_j = v_j.
\]
Therefore,
\[
D= (\lambda_1e_1|\cdots|\lambda_ne_n)
\]
is a diagonal matrix.  

Conversely, suppose that $A$ is a real diagonalizable matrix.  Then there
exists an invertible matrix $S$ such that $D=S\inv AS$ is diagonal.  Let
$v_j = Se_j$.  We claim that $\{v_1,\ldots,v_n\}$ is a linearly independent 
set of eigenvectors of $A$.

Since $D$ is diagonal, $De_j=\lambda_je_j$ for some real number $\lambda_j$. 
It follows that
\[
Av_j = SDS\inv v_j = SDS\inv Se_j = SDe_j = \lambda_j Se_j = \lambda_jv_j.
\]  
So $v_j$ is an eigenvector of $A$.  Since the matrix $S$ is invertible, its
columns are linearly independent.  Since the columns of $S$ are $v_j$, the
set $\{v_1,\ldots,v_n\}$ is a linearly independent set of eigenvectors of
$A$, as claimed. \end{proof}



\begin{proof}[Proof of Theorem~\ref{T:diagsimple}] Let
$\lambda_1,\ldots,\lambda_n$ be the distinct eigenvalues of 
$A$ and let $v_1,\ldots,v_n$ be the corresponding eigenvectors.
Lemma~\ref{L:simpleeigen} implies that $\{v_1,\ldots,v_n\}$ is 
a linearly independent set in $\R^n$ and therefore a basis.
Lemma~\ref{L:eigenv-diag} implies that $A$ is diagonalizable.  \end{proof}

\begin{remark} 
Theorem~\ref{T:diagsimple} can be generalized as follows.  Suppose all 
eigenvalues of the $n\times n$ matrix $A$ are real. Then $A$ is diagonalizable 
if and only if the dimension of the eigenspace associated with each eigenvalue 
$\lambda$ is equal to the number of times $\lambda$ is an eigenvalue of $A$.
Issues surrounding this remark are discussed in Chapter~\ref{C:MNF}.
\end{remark}



\subsection*{Diagonalization Using MATLAB}
\index{diagonalization!in \protect\Matlab}

Let
\begin{matlabEquation}\label{diagonalize-example}
A= \left( \begin{array}{rrr} -6 & 12 & 4 \\
 8 & -21 & -8 \\
  -29 & 72 & 27 \end{array} \right).
\end{matlabEquation}
We use \Matlab to answer the questions:  Is $A$ real diagonalizable 
and, if it is, can we find the matrix $S$ such that $S\inv AS$ is diagonal?
We can find the eigenvalues of $A$ by typing {\tt eig(A)}. \Matlabp's
response is:
\begin{verbatim}
ans =
   -2.0000
   -1.0000
    3.0000
\end{verbatim}
Since the eigenvalues of $A$ are real and distinct, 
Theorem~\ref{T:diagsimple} states that $A$ can be diagonalized.  
That is, there is a matrix $S$ such that 
\[
S\inv AS = \left(\begin{array}{rrr} -1 & 0 & 0 \\ 0 & -2 & 0\\
0 & 0 & 3 \end{array}\right)
\]
The proof of Lemma~\ref{L:eigenv-diag} tells us how to find the 
matrix $S$.  We need to find the eigenvectors $v_1,v_2,v_3$ 
associated with the eigenvalues $-1,-2,3$, respectively.  Then 
the matrix $(v_1|v_2|v_3)$ whose columns are the eigenvectors is 
the matrix $S$. To verify this construction we first find the 
eigenvectors of $A$ by typing
\begin{verbatim}
v1 = null(A+eye(3));
v2 = null(A+2*eye(3));
v3 = null(A-3*eye(3));
\end{verbatim} 
Now type {\tt S = [v1 v2 v3]} to obtain
\begin{verbatim}
S =
   -0.8729    0.7071    0.0000
   -0.4364    0.0000   -0.3162
    0.2182    0.7071    0.9487
\end{verbatim}
Finally, check that $S\inv AS$ is the desired diagonal matrix by 
typing {\tt inv(S)*A*S}\index{\computer!inv} to obtain
\begin{verbatim}
ans =
   -1.0000    0.0000   -0.0000
    0.0000   -2.0000    0.0000
   -0.0000    0.0000    3.0000
\end{verbatim}

It is cumbersome to use the 
{\tt null}\index{\computer!null} command to find 
eigenvectors and \Matlab has been preprogrammed to do these
computations automatically.  We can use the {\tt eig} command 
to find the eigenvectors and eigenvalues of a matrix $A$, as 
follows.  Type
\begin{verbatim}
[S,D] = eig(A)
\end{verbatim}\index{\computer!eig}
and \Matlab responds with 
\begin{verbatim}
S =
   -0.7071    0.8729    0.0000
    0.0000    0.4364   -0.3162
   -0.7071   -0.2182    0.9487
 
D = 
   -2.0000         0         0
         0   -1.0000         0
         0         0    3.0000
\end{verbatim}
The matrix $S$ is the transition matrix\index{matrix!transition} 
whose columns are the 
eigenvectors of $A$ and the matrix $D$ is a diagonal matrix whose 
$j^{th}$ diagonal entry is the eigenvalue of $A$ corresponding to 
the eigenvector in the $j^{th}$ column of $S$.




\includeexercises



\end{document}
