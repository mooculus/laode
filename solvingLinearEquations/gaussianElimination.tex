\documentclass{ximera}

\usepackage{epsfig}

\graphicspath{
  {./}
  {figures/}
}

\usepackage{epstopdf}
%\usepackage{ulem}
\usepackage[normalem]{ulem}

\epstopdfsetup{outdir=./}

\usepackage{morewrites}
\makeatletter
\newcommand\subfile[1]{%
\renewcommand{\input}[1]{}%
\begingroup\skip@preamble\otherinput{#1}\endgroup\par\vspace{\topsep}
\let\input\otherinput}
\makeatother

\newcommand{\EXER}{}
\newcommand{\includeexercises}{\EXER\directlua{dofile(kpse.find_file("exercises","lua"))}}

\newenvironment{computerExercise}{\begin{exercise}}{\end{exercise}}

%\newcounter{ccounter}
%\setcounter{ccounter}{1}
%\newcommand{\Chapter}[1]{\setcounter{chapter}{\arabic{ccounter}}\chapter{#1}\addtocounter{ccounter}{1}}

%\newcommand{\section}[1]{\section{#1}\setcounter{thm}{0}\setcounter{equation}{0}}

%\renewcommand{\theequation}{\arabic{chapter}.\arabic{section}.\arabic{equation}}
%\renewcommand{\thefigure}{\arabic{chapter}.\arabic{figure}}
%\renewcommand{\thetable}{\arabic{chapter}.\arabic{table}}

%\newcommand{\Sec}[2]{\section{#1}\markright{\arabic{ccounter}.\arabic{section}.#2}\setcounter{equation}{0}\setcounter{thm}{0}\setcounter{figure}{0}}
  
\newcommand{\Sec}[2]{\section{#1}}

\setcounter{secnumdepth}{2}
%\setcounter{secnumdepth}{1} 

%\newcounter{THM}
%\renewcommand{\theTHM}{\arabic{chapter}.\arabic{section}}

\newcommand{\trademark}{{R\!\!\!\!\!\bigcirc}}
%\newtheorem{exercise}{}

\newcommand{\dfield}{{\sf SlopeField}}

\newcommand{\pplane}{{\sf PhasePlane}}

\newcommand{\PPLANE}{{\sf PHASEPLANE}}

% BADBAD: \newcommand{\Bbb}{\bf}. % Package amsfonts Warning: Obsolete command \Bbb; \mathbb should be used instead.

\newcommand{\R}{\mbox{$\mathbb{R}$}}
\let\C\relax
\newcommand{\C}{\mbox{$\mathbb{C}$}}
\newcommand{\Z}{\mbox{$\mathbb{Z}$}}
\newcommand{\N}{\mbox{$\mathbb{N}$}}
\newcommand{\D}{\mbox{{\bf D}}}

\newcommand{\WW}{\mathcal{W}}

\usepackage{amssymb}
%\newcommand{\qed}{\hfill\mbox{\raggedright$\square$} \vspace{1ex}}
%\newcommand{\proof}{\noindent {\bf Proof:} \hspace{0.1in}}

\newcommand{\setmin}{\;\mbox{--}\;}
\newcommand{\Matlab}{{M\small{AT\-LAB}} }
\newcommand{\Matlabp}{{M\small{AT\-LAB}}}
\newcommand{\computer}{\Matlab Instructions}
\renewcommand{\computer}{M\small{ATLAB} Instructions}
\newcommand{\half}{\mbox{$\frac{1}{2}$}}
\newcommand{\compose}{\raisebox{.15ex}{\mbox{{\scriptsize$\circ$}}}}
\newcommand{\AND}{\quad\mbox{and}\quad}
\newcommand{\vect}[2]{\left(\begin{array}{c} #1_1 \\ \vdots \\
 #1_{#2}\end{array}\right)}
\newcommand{\mattwo}[4]{\left(\begin{array}{rr} #1 & #2\\ #3
&#4\end{array}\right)}
\newcommand{\mattwoc}[4]{\left(\begin{array}{cc} #1 & #2\\ #3
&#4\end{array}\right)}
\newcommand{\vectwo}[2]{\left(\begin{array}{r} #1 \\ #2\end{array}\right)}
\newcommand{\vectwoc}[2]{\left(\begin{array}{c} #1 \\ #2\end{array}\right)}

\newcommand{\ignore}[1]{}


\newcommand{\inv}{^{-1}}
\newcommand{\CC}{{\cal C}}
\newcommand{\CCone}{\CC^1}
\newcommand{\Span}{{\rm span}}
\newcommand{\rank}{{\rm rank}}
\newcommand{\trace}{{\rm tr}}
\newcommand{\RE}{{\rm Re}}
\newcommand{\IM}{{\rm Im}}
\newcommand{\nulls}{{\rm null\;space}}

\newcommand{\dps}{\displaystyle}
\newcommand{\arraystart}{\renewcommand{\arraystretch}{1.8}}
\newcommand{\arrayfinish}{\renewcommand{\arraystretch}{1.2}}
\newcommand{\Start}[1]{\vspace{0.08in}\noindent {\bf Section~\ref{#1}}}
\newcommand{\exer}[1]{\noindent {\bf \ref{#1}}}
\newcommand{\ans}{\textbf{Answer:} }
\newcommand{\matthree}[9]{\left(\begin{array}{rrr} #1 & #2 & #3 \\ #4 & #5 & #6
\\ #7 & #8 & #9\end{array}\right)}
\newcommand{\cvectwo}[2]{\left(\begin{array}{c} #1 \\ #2\end{array}\right)}
\newcommand{\cmatthree}[9]{\left(\begin{array}{ccc} #1 & #2 & #3 \\ #4 & #5 &
#6 \\ #7 & #8 & #9\end{array}\right)}
\newcommand{\vecthree}[3]{\left(\begin{array}{r} #1 \\ #2 \\
#3\end{array}\right)}
\newcommand{\cvecthree}[3]{\left(\begin{array}{c} #1 \\ #2 \\
#3\end{array}\right)}
\newcommand{\cmattwo}[4]{\left(\begin{array}{cc} #1 & #2\\ #3
&#4\end{array}\right)}

\newcommand{\Matrix}[1]{\ensuremath{\left(\begin{array}{rrrrrrrrrrrrrrrrrr} #1 \end{array}\right)}}

\newcommand{\Matrixc}[1]{\ensuremath{\left(\begin{array}{cccccccccccc} #1 \end{array}\right)}}



\renewcommand{\labelenumi}{\theenumi}
\newenvironment{enumeratea}%
{\begingroup
 \renewcommand{\theenumi}{\alph{enumi}}
 \renewcommand{\labelenumi}{(\theenumi)}
 \begin{enumerate}}
 {\end{enumerate}
 \endgroup}

\newcounter{help}
\renewcommand{\thehelp}{\thesection.\arabic{equation}}

%\newenvironment{equation*}%
%{\renewcommand\endequation{\eqno (\theequation)* $$}%
%   \begin{equation}}%
%   {\end{equation}\renewcommand\endequation{\eqno \@eqnnum
%$$\global\@ignoretrue}}

%\input{psfig.tex}

\author{Martin Golubitsky and Michael Dellnitz}

%\newenvironment{matlabEquation}%
%{\renewcommand\endequation{\eqno (\theequation*) $$}%
%   \begin{equation}}%
%   {\end{equation}\renewcommand\endequation{\eqno \@eqnnum
% $$\global\@ignoretrue}}

\newcommand{\soln}{\textbf{Solution:} }
\newcommand{\exercap}[1]{\centerline{Figure~\ref{#1}}}
\newcommand{\exercaptwo}[1]{\centerline{Figure~\ref{#1}a\hspace{2.1in}
Figure~\ref{#1}b}}
\newcommand{\exercapthree}[1]{\centerline{Figure~\ref{#1}a\hspace{1.2in}
Figure~\ref{#1}b\hspace{1.2in}Figure~\ref{#1}c}}
\newcommand{\para}{\hspace{0.4in}}

\usepackage{ifluatex}
\ifluatex
\ifcsname displaysolutions\endcsname%
\else
\renewenvironment{solution}{\suppress}{\endsuppress}
\fi
\else
\renewenvironment{solution}{}{}
\fi

\ifcsname answer\endcsname
\renewcommand{\answer}{}
\fi

%\ifxake
%\newenvironment{matlabEquation}{\begin{equation}}{\end{equation}}
%\else
\newenvironment{matlabEquation}%
{\let\oldtheequation\theequation\renewcommand{\theequation}{\oldtheequation*}\begin{equation}}%
  {\end{equation}\let\theequation\oldtheequation}
%\fi

\makeatother

\newcommand{\RED}[1]{{\color{red}{#1}}} 


\title{Gaussian Elimination}

\begin{document}
\begin{abstract}
\end{abstract}
\maketitle

  \label{S:Gauss}

A general system of $m$ {\em linear\/} equations \index{linear}
in $n$ unknowns has the form
\begin{equation}    \label{general}
\arraycolsep 2pt
\begin{array}{rcrcrcrcr}
 a_{11}x_1 & + & a_{12}x_2 & + & \cdots & + & a_{1n}x_n & = &   b_1  \\
 a_{21}x_1 & + & a_{22}x_2 & + & \cdots & + & a_{2n}x_n & = &   b_2  \\
        & \vdots &      & \vdots &    & \vdots &     & \vdots &     \\
 a_{m1}x_1 & + & a_{m2}x_2 & + & \cdots & + & a_{mn}x_n & = &   b_m  \\
\end{array}
\end{equation}
The entries $a_{ij}$ and $b_i$ are constants.  Our task is to find
a method for solving \eqref{general} for the variables
$x_1,\ldots,x_n$.

\subsection*{Easily Solved Equations}

Some systems are easily solved.  The system of three
equations ($m=3$) in three unknowns ($n=3$)
\begin{equation}
\begin{array}{rcrcrcrl} \label{examp3}
  x_1 & + & 2x_2 & + & 3x_3 & = &  10  & \\
      &   &  x_2 & - & \frac{1}{5}x_3 & = & \frac{7}{5}  & \\
      &   &      &   &  x_3 & = &   3  & \\
\end{array}
\end{equation}
is one example.  The $3^{rd}$ equation states that $x_3=3$.
Substituting this value into the $2^{nd}$ equation allows us to
solve the $2^{nd}$ equation for $x_2=2$.  Finally, substituting
$x_2=2$ and $x_3=3$ into the $1^{st}$ equation allows us to
solve for $x_1=-3$.  The process that we have just described is
called {\em back substitution\/}.\index{back substitution}

Next, consider the system of two equations ($m=2$) in three
unknowns ($n=3$):
\begin{equation}  \label{e23}
\begin{array}{rcrcrcrl}
  x_1 & + & 2x_2 & + & 3x_3 & = &  10  & \\
      &   &      &   &  x_3 & = &   3  & \! . \\
\end{array}
\end{equation}
The $2^{nd}$ equation in \eqref{e23} states that $x_3=3$.
Substituting
this value into the $1^{st}$ equation leads to the equation
\[
x_1 = 1-2x_2.
\]
We have shown that every solution to \eqref{e23} has the form
$(x_1,x_2,x_3)=(1-2x_2,x_2,3)$ and that every vector
$(1-2x_2,x_2,3)$ is a solution of \eqref{e23}.  Thus, there is an
infinite number of solutions to \eqref{e23}, and these solutions
can be parameterized by one number $x_2$.

\subsection*{Equations Having No Solutions}

Note that the system of equations
\begin{eqnarray*}
x_1 - x_2 = 1\\
x_1 - x_2 = 2
\end{eqnarray*}
has no solutions.

\begin{definition}
A linear system of equations is {\em inconsistent\/} if the
system has no solutions and {\em consistent\/} if the system
does have solutions.
\end{definition} \index{consistent} \index{inconsistent}

As discussed in the previous section, \eqref{incon} is an example
of a linear system that \Matlab cannot solve.  In fact, that
system is inconsistent --- inspect the $2^{nd}$ and $4^{th}$
equations in \eqref{incon}.

Gaussian elimination is an algorithm for finding all solutions
to a system of linear equations by reducing the given system to
ones like \eqref{examp3} and \eqref{e23}, that are easily solved by
back substitution.  Consequently, Gaussian elimination can also be
used to determine whether a system is consistent or inconsistent.

\subsection*{Elementary Equation Operations}

There are three ways to change a system of equations without
changing the set of solutions; Gaussian elimination
\index{Gaussian elimination} is based on this observation.  The
three elementary operations are:
\begin{enumerate}
\item   Swap two equations.
\item   Multiply a single equation by a nonzero number.
\item   Add a scalar multiple of one equation to another.
\end{enumerate}

We begin with an example:
\begin{equation}
\begin{array}{rcrcrcrl}
  x_1 & + & 2x_2 & + & 3x_3 & = &  10  & \\
  x_1 & + & 2x_2 & + &  x_3 & = &   4  & \\
 2x_1 & + & 9x_2 & + & 5x_3 & = &  27  & .\\
\end{array}
\end{equation}
Gaussian elimination works by eliminating variables from the
equations in a fashion similar to the substitution method in the
previous section.  To begin, eliminate the variable $x_1$ from
all but the $1^{st}$ equation, as follows.  Subtract the
$1^{st}$ equation from the $2^{nd}$, and subtract twice the
$1^{st}$ equation from the $3^{rd}$, obtaining:
\begin{equation}
\begin{array}{rcrcrcrl}
  x_1 & + & 2x_2 & + & 3x_3 & = &  10  & \\
      &   &      &   &-2x_3 & = &  -6  & \\
      &   & 5x_2 & - &  x_3 & = &   7  & .\\
\end{array}
\end{equation}
Next, swap the $2^{nd}$ and $3^{rd}$ equations, so that
the coefficient of $x_2$ in the new $2^{nd}$ equation is nonzero.
This yields
\begin{equation}
\begin{array}{rcrcrcrl}
  x_1 & + & 2x_2 & + & 3x_3 & = &  10  & \\
      &   & 5x_2 & - &  x_3 & = &   7  & \\
      &   &      &   &-2x_3 & = &  -6  & .\\
\end{array}
\end{equation}
Now, divide the $2^{nd}$ equation by $5$ and the $3^{rd}$
equation by $-2$ to obtain a system of equations identical to
our first example \eqref{examp3}, which we solved by back
substitution.


\subsection*{Augmented Matrices}

The process of performing Gaussian elimination when the number
of equations is greater than two or three is painful.  The
computer, however, can help with the manipulations.  We begin by
introducing the {\em augmented matrix\/}. \index{matrix!augmented}
The augmented matrix associated with \eqref{general} has
$m$ rows and $n+1$ columns and is written as:
\begin{equation}  \label{augmented}
\left(
\begin{array}{rrrr|r}
 a_{11} & a_{12} & \cdots & a_{1n} &  b_1 \\
 a_{21} & a_{22} & \cdots & a_{2n} &  b_2 \\
 \vdots & \vdots &        & \vdots & \vdots \\
 a_{m1} & a_{m2} & \cdots & a_{mn} &  b_m
\end{array}
\right)
\end{equation}
The augmented matrix contains all of the information that is
needed to solve system \eqref{general}.

\subsection*{Elementary Row Operations} \index{elementary row
operations}

The elementary operations used in Gaussian elimination
can be interpreted as {\em row operations\/} on
the augmented matrix, as follows:
\begin{enumerate}
\item   Swap two rows.
\item   Multiply a single row by a nonzero number.
\item   Add a scalar multiple of one row to another.
\end{enumerate}
We claim that by using these elementary row operations
intelligently, we can always solve a consistent linear system
--- indeed, we can determine when a linear system is consistent
or inconsistent.  The idea is to perform elementary row
operations in such a way that the new augmented matrix has zero
entries below the diagonal.

We describe this process inductively.  Begin with the $1^{st}$
column.  We assume for now that some entry in this column is
nonzero.  If $a_{11}=0$, then swap two rows so that the
number $a_{11}$ is nonzero.  Then divide the $1^{st}$ row by
$a_{11}$ so that the leading entry in that row is $1$.  Now
subtract $a_{i1}$ times the $1^{st}$ row from the $i^{th}$ row
for each row $i$ from $2$ to $m$.  The end result is that the
$1^{st}$ column has a $1$ in the $1^{st}$ row and a $0$ in every
row below the $1^{st}$.  The result is
\[
\left(\begin{array}{cccc}  1 & * & \cdots & * \\
0 & * & \cdots & * \\ \vdots & \vdots & \vdots & \vdots \\
0 & * & \cdots & * \end{array} \right).
\]


Next we consider the $2^{nd}$ column.  We assume that some entry
in that column below the $1^{st}$ row is nonzero.  So, if
necessary, we can swap two rows below the $1^{st}$ row so
that the entry $a_{22}$ is nonzero.  Then we divide the $2^{nd}$
row by $a_{22}$ so that its leading nonzero entry is $1$.  Then
we subtract appropriate multiples of the $2^{nd}$ row from each
row below the $2^{nd}$ so that all the entries in the $2^{nd}$
column below the $2^{nd}$ row are $0$.  The result is
\[
\left(\begin{array}{cccc}  1 & * & \cdots & * \\
0 & 1 & \cdots & * \\ \vdots & \vdots & \vdots & \vdots \\
0 & 0 & \cdots & * \end{array} \right).
\]

Then we continue with the $3^{rd}$ column.  That's the idea.  However,
does this process always work and what happens if all of the entries
in a column are zero?  Before answering these questions we do
experimentation with \Matlabp.

\subsection*{Row Operations in \Matlab}\index{elementary row
operations!in \protect\Matlab}

In \Matlab the $i^{th}$ row of a matrix {\tt A} is specified by
{\tt A(i,:)}.  Thus to replace the $5^{th}$ row of a matrix {\tt
A} by twice itself, we need only type:
\begin{verbatim}
A(5,:) = 2*A(5,:)
\end{verbatim}
In general, we can replace the $i^{th}$ row of the matrix
{\tt A} by $c$ times itself by typing
\begin{verbatim}
A(i,:) = c*A(i,:)
\end{verbatim}
Similarly, we can divide the $i^{th}$ row of the matrix {\tt A}
by the nonzero number $c$ by typing
\begin{verbatim}
A(i,:) = A(i,:)/c
\end{verbatim}

The third elementary row operation is performed similarly.
Suppose we want to add $c$ times the $i^{th}$ row to the
$j^{th}$ row, then we type
\begin{verbatim}
A(j,:) = A(j,:) + c*A(i,:)
\end{verbatim}
For example, subtracting $3$ times the $7^{th}$ row from the
$4^{th}$ row of the matrix {\tt A} is accomplished by typing:
\begin{verbatim}
A(4,:) = A(4,:) - 3*A(7,:)
\end{verbatim}

The first elementary row operation, swapping two rows, requires
a different kind of \Matlab command.  In \Matlabp, the $i^{th}$
and $j^{th}$ rows of the matrix {\tt A} are permuted by the
command
\begin{verbatim}
A([i j],:) = A([j i],:)
\end{verbatim}
So, to swap the $1^{st}$ and $3^{rd}$ rows of the matrix
{\tt A}, we type
\begin{verbatim}
A([1 3],:) = A([3 1],:)
\end{verbatim}  \index{\computer!A([1 3],:)}

\subsection*{Examples of Row Reduction in \Matlab}

Let us see how the row operations can be used in \Matlabp.  As
an example, we consider the augmented matrix
\begin{matlabEquation}  \label{examp4}
\left(
\begin{array}{rrrr|r}
 1  &  3  &  0  & -1  &  -8\\
 2  &  6  & -4  &  4  &   4\\
 1  &  0  & -1  & -9  & -35\\
 0  &  1  &  0  &  3  &  10
\end{array}
\right)
\end{matlabEquation}

We enter this information into \Matlab by typing
\begin{verbatim}
e2_3_8
\end{verbatim}
which produces the result
\begin{verbatim}
A =
     1     3     0    -1    -8
     2     6    -4     4     4
     1     0    -1    -9   -35
     0     1     0     3    10
\end{verbatim}

We now perform Gaussian elimination on {\tt A}, and then solve the 
resulting system by back substitution.  Gaussian elimination uses 
elementary row operations to set the entries that are in the lower 
left part of {\tt A} to zero. These entries are indicated by
numbers in the following matrix:
\begin{verbatim}
    *     *     *     *     *
    2     *     *     *     *
    1     0     *     *     *
    0     1     0     *     *
\end{verbatim}

Gaussian elimination \index{Gaussian elimination} works
inductively.  Since the first entry in the matrix $A$ is equal
to $1$, the first step in Gaussian elimination is to set to zero
all entries in the $1^{st}$ column below the $1^{st}$ row.  We
begin by eliminating the {\tt 2} that is the first entry in the
$2^{nd}$ row of {\tt A}.  We replace the $2^{nd}$ row by the
$2^{nd}$ row minus twice the $1^{st}$ row.  To accomplish this
elementary row operation, we type
\begin{verbatim}
A(2,:) = A(2,:) - 2*A(1,:)
\end{verbatim}
and the result is
\begin{verbatim}
A =
     1     3     0    -1    -8
     0     0    -4     6    20
     1     0    -1    -9   -35
     0     1     0     3    10
\end{verbatim}
In the next step, we eliminate the {\tt 1} from the entry in the
$3^{rd}$ row, $1^{st}$ column of {\tt A}.  We do this by
typing
\begin{verbatim}
A(3,:) = A(3,:) - A(1,:)
\end{verbatim}
which yields
\begin{verbatim}
A =
     1     3     0    -1    -8
     0     0    -4     6    20
     0    -3    -1    -8   -27
     0     1     0     3    10
\end{verbatim}

Using elementary row operations, we have now set the entries
in the $1^{st}$ column below the $1^{st}$ row to $0$.  Next,
we alter the $2^{nd}$ column.  We begin by swapping
the $2^{nd}$ and $4^{th}$ rows so that the leading nonzero entry
in the $2^{nd}$ row is $1$.   To accomplish this swap, we type
\begin{verbatim}
A([2 4],:) = A([4 2],:)
\end{verbatim}
and obtain
\begin{verbatim}
A =
     1     3     0    -1    -8
     0     1     0     3    10
     0    -3    -1    -8   -27
     0     0    -4     6    20
\end{verbatim}
The next elementary row operation is the command
\begin{verbatim}
A(3,:) = A(3,:) + 3*A(2,:)
\end{verbatim}
which leads to
\begin{verbatim}
A =
     1     3     0    -1    -8
     0     1     0     3    10
     0     0    -1     1     3
     0     0    -4     6    20
\end{verbatim}
Now we have set all entries in the $2^{nd}$ column below
the $2^{nd}$ row to $0$.

Next, we set the first nonzero entry in the $3^{rd}$ row to {\tt 1}
by multiplying the $3^{rd}$ row by $-1$, obtaining
\begin{verbatim}
A =
     1     3     0    -1    -8
     0     1     0     3    10
     0     0     1    -1    -3
     0     0    -4     6    20
\end{verbatim}

Since the leading nonzero entry in the $3^{rd}$ row is $1$,  we
next eliminate the nonzero entry in the $3^{rd}$ column,
$4^{th}$ row. This is accomplished by the following \Matlab
command:
\begin{verbatim}
A(4,:) = A(4,:) + 4*A(3,:)
\end{verbatim}
Finally, divide the $4^{th}$ row by $2$ to obtain:
\begin{verbatim}
A =
     1     3     0    -1    -8
     0     1     0     3    10
     0     0     1    -1    -3
     0     0     0     1     4
\end{verbatim}

By using elementary row operations, we have arrived at the system
\begin{equation}
\arraycolsep 2pt
\begin{array}{rcrcrcrcrl}
  x_1 & + & 3x_2 &   &     & - &  x_4 & = &  -8 & \\
      &   &  x_2 &   &     & + & 3x_4 & = &  10 & \\
      &   &      &   & x_3 & - &  x_4 & = &  -3 & \\
      &   &      &   &     &   &  x_4 & = &   4 &\! ,
\end{array}
\end{equation}
that can now be solved by back substitution.  We obtain
\begin{equation} \label{ans1}
        x_4 = 4,\quad x_3 = 1,\quad x_2 = -2,\quad x_1 = 2.
\end{equation}
We return to the original set of equations corresponding to
\eqref{examp4}
\begin{matlabEquation}    \label{examp4_con}
\arraycolsep 2pt
\begin{array}{rcrcrcrcrl}
  x_1 & + & 3x_2 &   &      & - &  x_4 & = &  -8  & \\
 2x_1 & + & 6x_2 & - & 4x_3 & + & 4x_4 & = &   4  & \\
  x_1 &   &      & - &  x_3 & - & 9x_4 & = & -35  & \\
      &   &  x_2 &   &      & + & 3x_4 & = &  10  & \! .
\end{array}
\end{matlabEquation}
Load the corresponding linear system into \Matlab by typing
\begin{verbatim}
e2_3_11
\end{verbatim}
The information in \eqref{examp4_con} is contained in the
coefficient matrix {\tt C} and the right hand side {\tt b}.
A direct solution is found by typing
\begin{verbatim}
x = C\b
\end{verbatim}
which yields the same answer as in \eqref{ans1}, namely,
\begin{verbatim}
x =
    2.0000
   -2.0000
    1.0000
    4.0000
\end{verbatim}

\subsection*{Introduction to Echelon Form} \index{echelon form}

Next, we discuss how Gaussian elimination works in an example
in which the number of rows and the number of columns in the
coefficient matrix are unequal.  We consider the augmented
matrix\index{matrix!augmented}
\begin{matlabEquation}  \label{examp5}
\left(
\begin{array}{rrrrrr|r}
 1  &  0  & -2  &  3  &  4  &  0  &  1\\
 0  &  1  &  2  &  4  &  0  & -2  &  0\\
 2  & -1  & -4  &  0  & -2  &  8  & -4\\
-3  &  0  &  6  & -8  &-12  &  2  & -2
\end{array}
\right)
\end{matlabEquation}
This information is entered into \Matlab by typing
\begin{verbatim}
e2_3_12
\end{verbatim}
Again, the augmented matrix is denoted by {\tt A}.

We begin by eliminating the {\tt 2} in the entry in the $3^{rd}$
row, $1^{st}$ column.  To accomplish the corresponding
elementary row operation, we type
\begin{verbatim}
A(3,:) = A(3,:) - 2*A(1,:)
\end{verbatim}
resulting in
\begin{verbatim}
A =
     1     0    -2     3     4     0     1
     0     1     2     4     0    -2     0
     0    -1     0    -6   -10     8    -6
    -3     0     6    -8   -12     2    -2
\end{verbatim}
We proceed with
\begin{verbatim}
A(4,:) = A(4,:) + 3*A(1,:)
\end{verbatim}
to create two more zeros in the $4^{th}$ row.  Finally, we
eliminate the {\tt -1} in the $3^{rd}$ row, $2^{nd}$
column by
\begin{verbatim}
A(3,:) = A(3,:) + A(2,:)
\end{verbatim}
to arrive at
\begin{verbatim}
A =
     1     0    -2     3     4     0     1
     0     1     2     4     0    -2     0
     0     0     2    -2   -10     6    -6
     0     0     0     1     0     2     1
\end{verbatim}
Next we set the leading nonzero entry in the $3^{rd}$ row to $1$
by dividing the $3^{rd}$ row by $2$.   That is, we type
\begin{verbatim}
A(3,:) = A(3,:)/2
\end{verbatim}
to obtain
\begin{verbatim}
A =
     1     0    -2     3     4     0     1
     0     1     2     4     0    -2     0
     0     0     1    -1    -5     3    -3
     0     0     0     1     0     2     1
\end{verbatim}
We say that the matrix {\tt A} is in (row) {\em echelon form\/}
\index{echelon form} since the first nonzero entry in each row
is a {\tt 1}, each entry in a column below a leading {\tt 1} is
{\tt 0}, and the leading {\tt 1} moves to the right as you go
down the matrix.  In row echelon form, the entries where leading
$1$'s occur are called {\em pivots\/}.\index{pivot}

If we compare the structure of this matrix to the ones we have
obtained previously, then we see that here we have two columns
too many.  Indeed, we may solve these equations by back
substitution for any choice of the variables $x_5$ and $x_6$.

The idea behind back substitution \index{back substitution} is
to solve the last equation for the variable corresponding to the
first nonzero coefficient.  In this case, we use the $4^{th}$
equation to solve for $x_4$ in terms of $x_5$ and $x_6$, and
then we substitute for $x_4$ in the first three equations.  This
process can also be accomplished by elementary row operations.
Indeed, eliminating the variable $x_4$ from the first three
equations is the same as using row operations to set the first
three entries in the $4^{th}$ column to {\tt 0}.  We can do this
by typing
\begin{verbatim}
A(3,:) = A(3,:) + A(4,:);
A(2,:) = A(2,:) - 4*A(4,:);
A(1,:) = A(1,:) - 3*A(4,:)
\end{verbatim}
{\bf Remember:} By typing semicolons after the first two rows, we
have told \Matlab not to print the intermediate results.  Since
we have not typed a semicolon after the $3^{rd}$ row, \Matlab
outputs
\begin{verbatim}
A =
     1     0    -2     0     4    -6    -2
     0     1     2     0     0   -10    -4
     0     0     1     0    -5     5    -2
     0     0     0     1     0     2     1
\end{verbatim}
We proceed with back substitution by eliminating the nonzero
entries in the first two rows of the $3^{rd}$ column.  To do
this, type
\begin{verbatim}
A(2,:) = A(2,:) - 2*A(3,:);
A(1,:) = A(1,:) + 2*A(3,:)
\end{verbatim}
which yields
\begin{verbatim}
A =
     1     0     0     0    -6     4    -6
     0     1     0     0    10   -20     0
     0     0     1     0    -5     5    -2
     0     0     0     1     0     2     1
\end{verbatim}
The augmented matrix is now in {\em reduced echelon form}
\index{echelon form!reduced} and the corresponding system of
equations has the form
\begin{equation}   \label{e:refexamp5}
\arraycolsep 2pt
\begin{array}{rcrcrcrcrcrcrl}
  x_1 &   &     &   &     &  &     & - &  6x_5 & + &  4x_6 & = &
-6 & \\
      &   & x_2 &   &     &  &     & + & 10x_5 & - & 20x_6 & = &
0 & \\
      &   &     &   & x_3 &  &     & - &  5x_5 & + &  5x_6 & = &
-2 & \\
      &   &     &   &     &  & x_4 &   &       & + &  2x_6 & = &
1 &\! ,
\end{array}
\end{equation}
A matrix is in reduced echelon form \index{echelon form!reduced}
if it is in echelon form and if {\em every\/} entry in a column
containing a pivot, other than the pivot itself, is {\tt 0}.

Reduced echelon form allows us to solve directly this system of
equations in terms of the variables $x_5$ and $x_6$,
\begin{equation}  \label{e:refexamp6}
\left(\begin{array}{c} x_1\\x_2\\x_3\\x_4\\x_5\\x_6\end{array}\right) =
\left(\begin{array}{c} -6+6x_5-4x_6\\-10x_5+20x_6\\
-2+5x_5-5x_6\\1-2x_6\\x_5\\x_6\end{array}\right).
\end{equation}
It is important to note that every consistent system of linear equations
corresponding to an augmented matrix in reduced echelon form can be
solved as in \eqref{e:refexamp6} --- and this is one reason for emphasizing
reduced echelon form.  We can rewrite the solutions in \eqref{e:refexamp6}
in the form:
\[
\Matrix{x_1,\\x_2\\x_3\\x_4\\x_5\\x_6} = \Matrix{-6 \\ 0 \\ -2 \\1 \\0 \\ 0} +
x_5\Matrix{ 6\\-10 \\ 5\\ 0\\ 1\\ 0} + x_6\Matrix{ -4\\20 \\-5 \\ -2\\0 \\ 1}.
\]

\begin{definition} \rm \label{D:lin_comb1}
A {\em linear combination} of the vectors $v_1,\ldots, v_k$ in 
$\R^n$ is a vector in $\R^n$ of the form
\[
v = \alpha_1 v_1 + \cdots + \alpha_k v_k
\] 
where $\alpha_1,\ldots\alpha_k$ are scalars in $\R$.  
\end{definition} \index{linear!combination}
It follows that the general solution to a linear system of equations is given by 
a single solution ($x_5 = x_6=0$) plus the linear combination of a finite number 
of vectors.  We will discuss reduced echelon form in more detail in the next section.



\includeexercises



\end{document}
