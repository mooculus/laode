\documentclass{ximera}

\usepackage{epsfig}

\graphicspath{
  {./}
  {figures/}
}


\usepackage{morewrites}

%\newcounter{ccounter}
%\setcounter{ccounter}{1}
%\newcommand{\Chapter}[1]{\setcounter{chapter}{\arabic{ccounter}}\chapter{#1}\addtocounter{ccounter}{1}}

%\newcommand{\section}[1]{\section{#1}\setcounter{thm}{0}\setcounter{equation}{0}}

%\renewcommand{\theequation}{\arabic{chapter}.\arabic{section}.\arabic{equation}}
%\renewcommand{\thefigure}{\arabic{chapter}.\arabic{figure}}
%\renewcommand{\thetable}{\arabic{chapter}.\arabic{table}}

%\newcommand{\Sec}[2]{\section{#1}\markright{\arabic{ccounter}.\arabic{section}.#2}\setcounter{equation}{0}\setcounter{thm}{0}\setcounter{figure}{0}}

\newcommand{\Sec}[2]{\section{#1}}

\setcounter{secnumdepth}{2}
%\setcounter{secnumdepth}{1} 

%\newcounter{THM}
%\renewcommand{\theTHM}{\arabic{chapter}.\arabic{section}}

\newcommand{\trademark}{{R\!\!\!\!\!\bigcirc}}
%\newtheorem{exercise}{}

\newcommand{\dfield}{{\sf dfield9}}
\newcommand{\pplane}{{\sf pplane9}}

\newcommand{\EXER}{\section*{Exercises}}%\vspace*{0.2in}\hrule\small\setcounter{exercise}{0}}
\newcommand{\CEXER}{}%\vspace{0.08in}\begin{center}Computer Exercises\end{center}}
\newcommand{\TEXER}{} %\vspace{0.08in}\begin{center}Hand Exercises\end{center}}
\newcommand{\AEXER}{} %\vspace{0.08in}\begin{center}Hand Exercises\end{center}}

% BADBAD: \newcommand{\Bbb}{\bf}

\newcommand{\R}{\mbox{$\Bbb{R}$}}
\newcommand{\C}{\mbox{$\Bbb{C}$}}
\newcommand{\Z}{\mbox{$\Bbb{Z}$}}
\newcommand{\N}{\mbox{$\Bbb{N}$}}
\newcommand{\D}{\mbox{{\bf D}}}
\usepackage{amssymb}
%\newcommand{\qed}{\hfill\mbox{\raggedright$\square$} \vspace{1ex}}
%\newcommand{\proof}{\noindent {\bf Proof:} \hspace{0.1in}}

\newcommand{\setmin}{\;\mbox{--}\;}
\newcommand{\Matlab}{{M\small{AT\-LAB}} }
\newcommand{\Matlabp}{{M\small{AT\-LAB}}}
\newcommand{\computer}{\Matlab Instructions}
\newcommand{\half}{\mbox{$\frac{1}{2}$}}
\newcommand{\compose}{\raisebox{.15ex}{\mbox{{\scriptsize$\circ$}}}}
\newcommand{\AND}{\quad\mbox{and}\quad}
\newcommand{\vect}[2]{\left(\begin{array}{c} #1_1 \\ \vdots \\
 #1_{#2}\end{array}\right)}
\newcommand{\mattwo}[4]{\left(\begin{array}{rr} #1 & #2\\ #3
&#4\end{array}\right)}
\newcommand{\mattwoc}[4]{\left(\begin{array}{cc} #1 & #2\\ #3
&#4\end{array}\right)}
\newcommand{\vectwo}[2]{\left(\begin{array}{r} #1 \\ #2\end{array}\right)}
\newcommand{\vectwoc}[2]{\left(\begin{array}{c} #1 \\ #2\end{array}\right)}



\newcommand{\inv}{^{-1}}
\newcommand{\CC}{{\cal C}}
\newcommand{\CCone}{\CC^1}
\newcommand{\Span}{{\rm span}}
\newcommand{\rank}{{\rm rank}}
\newcommand{\trace}{{\rm tr}}
\newcommand{\RE}{{\rm Re}}
\newcommand{\IM}{{\rm Im}}
\newcommand{\nulls}{{\rm null\;space}}

\newcommand{\dps}{\displaystyle}
\newcommand{\arraystart}{\renewcommand{\arraystretch}{1.8}}
\newcommand{\arrayfinish}{\renewcommand{\arraystretch}{1.2}}
\newcommand{\Start}[1]{\vspace{0.08in}\noindent {\bf Section~\ref{#1}}}
\newcommand{\exer}[1]{\noindent {\bf \ref{#1}}}
\newcommand{\ans}{}
\newcommand{\matthree}[9]{\left(\begin{array}{rrr} #1 & #2 & #3 \\ #4 & #5 & #6
\\ #7 & #8 & #9\end{array}\right)}
\newcommand{\cvectwo}[2]{\left(\begin{array}{c} #1 \\ #2\end{array}\right)}
\newcommand{\cmatthree}[9]{\left(\begin{array}{ccc} #1 & #2 & #3 \\ #4 & #5 &
#6 \\ #7 & #8 & #9\end{array}\right)}
\newcommand{\vecthree}[3]{\left(\begin{array}{r} #1 \\ #2 \\
#3\end{array}\right)}
\newcommand{\cvecthree}[3]{\left(\begin{array}{c} #1 \\ #2 \\
#3\end{array}\right)}
\newcommand{\cmattwo}[4]{\left(\begin{array}{cc} #1 & #2\\ #3
&#4\end{array}\right)}

\newcommand{\Matrix}[1]{\ensuremath{\left(\begin{array}{rrrrrrrrrrrrrrrrrr} #1 \end{array}\right)}}

\newcommand{\Matrixc}[1]{\ensuremath{\left(\begin{array}{cccccccccccc} #1 \end{array}\right)}}



\renewcommand{\labelenumi}{\theenumi)}
\newenvironment{enumeratea}%
{\begingroup
 \renewcommand{\theenumi}{\alph{enumi}}
 \renewcommand{\labelenumi}{(\theenumi)}
 \begin{enumerate}}
 {\end{enumerate}\endgroup}



\newcounter{help}
\renewcommand{\thehelp}{\thesection.\arabic{equation}}

%\newenvironment{equation*}%
%{\renewcommand\endequation{\eqno (\theequation)* $$}%
%   \begin{equation}}%
%   {\end{equation}\renewcommand\endequation{\eqno \@eqnnum
%$$\global\@ignoretrue}}

%\input{psfig.tex}

\author{Martin Golubitsky and Michael Dellnitz}

%\newenvironment{matlabEquation}%
%{\renewcommand\endequation{\eqno (\theequation*) $$}%
%   \begin{equation}}%
%   {\end{equation}\renewcommand\endequation{\eqno \@eqnnum
% $$\global\@ignoretrue}}

\newcommand{\soln}{\textbf{Solution:} }
\newcommand{\exercap}[1]{\centerline{Figure~\ref{#1}}}
\newcommand{\exercaptwo}[1]{\centerline{Figure~\ref{#1}a\hspace{2.1in}
Figure~\ref{#1}b}}
\newcommand{\exercapthree}[1]{\centerline{Figure~\ref{#1}a\hspace{1.2in}
Figure~\ref{#1}b\hspace{1.2in}Figure~\ref{#1}c}}
\newcommand{\para}{\hspace{0.4in}}

\renewenvironment{solution}{\suppress}{\endsuppress}

\ifxake
\newenvironment{matlabEquation}{\begin{equation}}{\end{equation}}
\else
\newenvironment{matlabEquation}%
{\let\oldtheequation\theequation\renewcommand{\theequation}{\oldtheequation*}\begin{equation}}%
  {\end{equation}\let\theequation\oldtheequation}
\fi

\makeatother


\title{Systems of Linear Equations and Matrices}

\begin{document}
\begin{abstract}
\end{abstract}
\maketitle


\label{S:2.1}

It is a simple exercise to solve the system of two equations
\begin{equation} \label{small}
\arraycolsep 2pt
\begin{array}{rcrcr}
 x & + & y & = & 7 \\
-x & + & 3y & = & 1
\end{array}
\end{equation}
to find that $x=5$ and $y=2$.  One way to solve
system \eqref{small} is to add the two equations, obtaining
\[
4y=8;
\]
hence $y=2$.  Substituting $y=2$ into the $1^{st}$ equation in
\eqref{small} yields $x=5$.

This system of equations can be solved in a more algorithmic
fashion by solving the $1^{st}$ equation in \eqref{small} for $x$
as
\[
x = 7 - y,
\]
and substituting this answer into the $2^{nd}$ equation in
\eqref{small}, to obtain
\[
-(7-y) +3y = 1.
\]
This equation simplifies to:
\[
4y = 8.
\]
Now proceed as before.

\subsection*{Solving Larger Systems by Substitution}

In contrast to solving the simple system of two equations,
it is less clear how to solve a complicated system of five
equations such as:
\begin{equation}    \label{big}
\arraycolsep 2pt
\begin{array}{rcrcrcrcrcrl}
 5x_1 & - & 4x_2 & + & 3x_3 & - & 6x_4 & + & 2x_5 & = &   4  & \\
 2x_1 & + &  x_2 & - &  x_3 & - &  x_4 & + &  x_5 & = &   6  & \\
  x_1 & + & 2x_2 & + &  x_3 & + &  x_4 & + & 3x_5 & = &  19  & \\
-2x_1 & - &  x_2 & - &  x_3 & + &  x_4 & - &  x_5 & = & -12  & \\
  x_1 & - & 6x_2 & + &  x_3 & + &  x_4 & + & 4x_5 & = &   4  & \!
.
\end{array}
\end{equation}
The algorithmic method used to solve \eqref{small} can be expanded
to produce a method, called {\em substitution\/},
\index{substitution} for solving larger systems. We describe the
substitution method as it applies to \eqref{big}.  Solve the
$1^{st}$ equation in \eqref{big} for $x_1$, obtaining
\begin{equation} \label{x1}
x_1 = \frac{4}{5}  + \frac{4}{5}x_2 - \frac{3}{5}x_3
   + \frac{6}{5}x_4 - \frac{2}{5}x_5.
\end{equation}
Then substitute the right hand side of \eqref{x1} for $x_1$ in the
remaining four equations in \eqref{big} to obtain a new system of
four equations in the four variables $x_2$,$x_3$,$x_4$,$x_5$.
This procedure eliminates the variable $x_1$.  Now proceed
inductively --- solve the $1^{st}$ equation in the new system
for $x_2$ and substitute this expression into the remaining
three equations to obtain a system of three equations in three
unknowns.  This step eliminates the variable $x_2$.  Continue by
substitution to eliminate the variables $x_3$ and $x_4$, and
arrive at a simple equation in $x_5$ --- which can be solved.
Once $x_5$ is known, then $x_4$, $x_3$, $x_2$, and $x_1$ can be
found in turn.

\subsection*{Two Questions}

\begin{itemize}
\item Is it realistic to expect to complete the substitution
procedure without making a mistake in arithmetic?
\item Will this procedure work --- or will some unforeseen
difficulty arise?
\end{itemize}

Almost surely, attempts to solve \eqref{big} by hand,
using the substitution procedure, will lead to arithmetic
errors.  However, computers and software have developed to the
point where solving a system such as \eqref{big} is routine.  In
this text, we use the software package \Matlab to illustrate
just how easy it has become to solve equations such as \eqref{big}.

The answer to the second question requires knowledge of the
{\em theory\/} of linear algebra.  In fact, no difficulties will
develop when trying to solve the particular system \eqref{big}
using the substitution algorithm.  We discuss why later.

\subsection*{Solving Equations by \Matlab}

We begin by discussing the information that is needed by \Matlab
to solve \eqref{big}.  The computer needs to know that there are
five equations in five unknowns --- but it does not need to keep
track of the unknowns $(x_1,x_2,x_3,x_4,x_5)$ by name.  Indeed,
the computer just needs to know the {\em matrix of
coefficients\/} \index{matrix!coefficient} in \eqref{big}
\begin{matlabEquation}  \label{bigmatrix}
\left(
\begin{array}{rrrrr}
 5 & -4 &  3 & -6 &  2 \\
 2 &  1 & -1 & -1 &  1 \\
 1 &  2 &  1 &  1 &  3 \\
-2 & -1 & -1 &  1 & -1 \\
 1 & -6 &  1 &  1 &  4
\end{array}
\right)
\end{matlabEquation}
and the {\em vector\/} on the right hand side of \eqref{big}
\begin{matlabEquation} \label{bigRHS}
\left(
\begin{array}{r}
  4 \\
  6 \\
 19 \\
-12 \\
  4
\end{array}
\right).
\end{matlabEquation}

We now describe how we enter this information into \Matlabp.  To
reduce the drudgery and to allow us to focus on ideas, the entries
in equations having a $*$ after their label,
such as \eqref{bigmatrix}, have been entered in the {\tt laode}
toolbox. This information can be accessed as follows.  After
starting your \Matlab session, type
\begin{verbatim}
e2_1_4
\end{verbatim}
followed by a carriage return.  This instruction tells \Matlab to
load equation \eqref{bigmatrix} of Chapter~\ref{lineq}.  The matrix of
coefficients is now available in \Matlabp; note that this matrix is
stored in the $5\times 5$ array {\tt A}.  What should appear is:
\begin{verbatim}
A =
     5    -4     3    -6     2
     2     1    -1    -1     1
     1     2     1     1     3
    -2    -1    -1     1    -1
     1    -6     1     1     4
\end{verbatim}
Indeed, comparing this result with \eqref{bigmatrix}, we see that
{\tt A} contains precisely the same information.

Since the label \eqref{bigRHS} is followed by a `$*$', we can enter
the vector in \eqref{bigRHS} into \Matlab by typing
\begin{verbatim}
e2_1_5
\end{verbatim}
Note that the right hand side of \eqref{big} is stored in the vector {\tt b}.
\Matlab should have responded with
\begin{verbatim}
b =
     4
     6
    19
   -12
     4
\end{verbatim}
Now \Matlab has all the information it needs to solve the system
of equations given in \eqref{big}.  To have \Matlab solve this
system, type
\begin{verbatim}
x = A\b
\end{verbatim}
\index{\computer!$\backslash$}to obtain
\begin{verbatim}
x =
    5.0000
    2.0000
    3.0000
    4.0000
    1.0000
\end{verbatim}
This answer is interpreted as follows: the five values of the
unknowns $x_1$,$x_2$,$x_3$,$x_4$,$x_5$ are stored in the vector
$x$; that is,
\begin{equation} \label{answer1}
 x_1 = 5,\quad x_2 = 2,\quad x_3 = 3,\quad x_4 = 4,\quad x_5 = 1.
\end{equation}
The reader may verify that \eqref{answer1} is indeed a solution of
\eqref{big} by substituting the values in \eqref{answer1} into the
equations in \eqref{big}.

\subsection*{Changing Entries in \Matlab}

\Matlab also permits access to single components of $x$.  For
instance, type
\begin{verbatim}
x(5)
\end{verbatim}
and the $5^{th}$ entry of $x$ is displayed,
\begin{verbatim}
ans =
    1.0000
\end{verbatim}
We see that the component {\tt x(i)} of {\tt x} corresponds to
the component $x_i$ of the vector $x$ where $i=1,2,3,4,5$.
Similarly, we can access the entries of the coefficient matrix
\index{matrix!coefficient} {\tt A}.
For instance, by typing
\begin{verbatim}
A(3,4)
\end{verbatim}
\Matlab responds with
\begin{verbatim}
ans =
    1
\end{verbatim}

It is also possible to change an individual entry in either a vector
or a matrix.  For example, if we enter
\begin{verbatim}
A(3,4) = -2
\end{verbatim}  \index{\computer!A(3,4)}
we obtain a new matrix {\tt A} which when displayed is:
\begin{verbatim}
A =
     5    -4     3    -6     2
     2     1    -1    -1     1
     1     2     1    -2     3
    -2    -1    -1     1    -1
     1    -6     1     1     4
\end{verbatim}
Thus the command {\tt A(3,4) = -2} changes the entry in the
$3^{rd}$ row, $4^{th}$ column of {\tt A} from $1$ to $-2$.
In other words, we have now entered into \Matlab the
information that is needed to solve the system of equations
\[
\arraycolsep 2pt
\begin{array}{rcrcrcrcrcrl}
 5x_1 & - & 4x_2 & + & 3x_3 & - &  6x_4 & + & 2x_5 & = &   4  & \\
 2x_1 & + &  x_2 & - &  x_3 & - &   x_4 & + &  x_5 & = &   6  & \\
  x_1 & + & 2x_2 & + &  x_3 & - &  2x_4 & + & 3x_5 & = &  19  & \\
-2x_1 & - &  x_2 & - &  x_3 & + &   x_4 & - &  x_5 & = & -12  & \\
x_1 & - & 6x_2 & + & x_3 & + & x_4 & + & 4x_5 & = & 4 & \! .
\end{array}
\]
As expected, this change in the coefficient matrix results in a
change in the solution of system \eqref{big}, as well.  Typing
\begin{verbatim}
x = A\b
\end{verbatim}
now leads to the solution
\begin{verbatim}
x =
    1.9455
    3.0036
    3.0000
    1.7309
    3.8364
\end{verbatim}
that is displayed to an accuracy of four decimal places.

In the next step, change {\tt A} as follows:
\begin{verbatim}
A(2,3) = 1
\end{verbatim}
The new system of equations is:
\begin{equation}  \label{incon}
\arraycolsep 2pt
\begin{array}{rcrcrcrcrcrl}
 5x_1 & - & 4x_2 & + & 3x_3 & - &  6x_4 & + & 2x_5 & = &   4  & \\
 2x_1 & + &  x_2 & + &  x_3 & - &   x_4 & + &  x_5 & = &   6  & \\
  x_1 & + & 2x_2 & + &  x_3 & - &  2x_4 & + & 3x_5 & = &  19  & \\
-2x_1 & - &  x_2 & - &  x_3 & + &   x_4 & - &  x_5 & = & -12  & \\
  x_1 & - & 6x_2 & + &  x_3 & + &   x_4 & + & 4x_5 & = &   4  & \!
.
\end{array}
\end{equation}
The command
\begin{verbatim}
x = A\b
\end{verbatim}  \index{\computer!$\backslash$}
now leads to the message
\begin{verbatim}
Warning: Matrix is singular to working precision.

x =
   Inf
   Inf
   Inf
   Inf
   Inf
\end{verbatim}  \index{\computer!inf}
Obviously, something is {\em wrong\/}; \Matlab cannot find a
solution to this system of equations!  Assuming that \Matlab is
working correctly, we have shed light on one of our previous
questions: the method of substitution described by \eqref{x1} need
{\em not\/} always lead to a solution, even though the method
does work for system \eqref{big}.  Why?  As we will see, this is
one of the questions that is answered by the theory of linear
algebra.  In the case of \eqref{incon}, it is fairly easy to see
what the difficulty is: the second and fourth equations
have the form $y=6$ and $-y=-12$, respectively.

\vspace{0.1in}

\noindent {\bf Warning:}  The \Matlab command
\begin{verbatim}
x = A\b
\end{verbatim}
may give an error message similar to the previous one.  When
this happens, one must approach the answer with caution.

\EXER


\TEXER

\noindent In Exercises~\ref{c2.1.8a} -- \ref{c2.1.8c} find solutions
to the given system of linear equations.
\begin{exercise} \label{c2.1.8a}
\[
\begin{array}{rcrcr}
 2x & - & y & = & 0 \\
 3x &   &   & = & 6 \end{array}
\]
\begin{prompt}
  Then $(x,y) = (\answer{2},\answer{4})$.
\end{prompt}
\end{exercise}
\begin{exercise} \label{c2.1.8b}
\[
\begin{array}{rcrcrcr}
 3x & - & 4y &   &    & = & 2\\
    &   & 2y & + & z  & = & 1\\
    &   &    &   & 3z & = & 9 \end{array}
\]
\begin{prompt}
  Then  $(x,y,z) = (\answer{-\frac{2}{3}}, \answer{-1}, \answer{3})$.
\end{prompt}
\end{exercise}
\begin{exercise} \label{c2.1.8c}
\[
\begin{array}{rcrcr}
 -2x & + &  y & = &  9 \\
  3x & + & 3y & = & -9 \end{array}
\]
\begin{prompt}
  Then $(x,y) = (\answer{-4},\answer{1})$.
\end{prompt}
\end{exercise}

\begin{exercise} \label{c2.1.8A}
Write the coefficient matrices for each of the systems of linear equations 
given in Exercises~\ref{c2.1.8a} -- \ref{c2.1.8c}.
\begin{prompt}
  For Exercise~\ref{c2.1.8a}, the coefficient matrix is
  \[
    \mattwo{\answer{2}}{\answer{-1}}{3}{\answer{0}}
  \]
  For Exercise~\ref{c2.1.8b}, the coefficient matrix is
  \[
    \matthree{\answer{3}}{\answer{-4}}{0}{0}{\answer{2}}{1}{0}{0}{3}
  \]
  For Exercise~\ref{c2.1.8c}, the coefficient matrix is
  \[
    \mattwo{\answer{-2}}{\answer{1}}{\answer{3}}{\answer{3}}.
  \]
\end{prompt}

\end{exercise}

\begin{exercise} \label{c2.1.9}
Neither of the following systems of three equations in three
unknowns has a unique solution --- but for different
reasons.  Solve these systems and explain why these systems
cannot be solved uniquely.
\[
\mbox{(a) }\; \begin{array}{rcrcrcr}
  x & - &  y &   &    & = &  4\\
  x & + & 3y & - & 2z & = & -6\\
 4x & + & 2y & - & 3z & = &  1
              \end{array}
            \]
            and
            \[\mbox{(b) }\; \begin{array}{rcrcrcr}
 2x & - & 4y & + & 3z & = &  4\\
 3x & - & 5y & + & 3z & = &  5\\
    &   & 2y & - & 3z & = & -4
\end{array}
\]
\begin{prompt}
  The matrix in part~(a) has
  \begin{multipleChoice}
    \choice{No solutions.}    
    \choice{Exactly one solution.}
    \choice[correct]{Infinitely many solutions.}
  \end{multipleChoice}

  The matrix in part~(b) has
  \begin{multipleChoice}
    \choice[correct]{No solutions.}    
    \choice{Exactly one solution.}
    \choice{Infinitely many solutions.}
  \end{multipleChoice}
\end{prompt}

\begin{hint}
For part~(a), replace $x$ in the second and third equations with
$4 + y$ to obtain $4y - 2z = -10$ and $6y - 3z = -15$.  Since these
equations have identical solutions, the system can be restated as
\[
\begin{array}{rrrrrrr}
x & = & y & + & 4 \\ 
z & = & -2y & + & 5\end{array}
\]
\end{hint}
\begin{hint}
So, for each choice of $y$, there exists a single solution.  For
example, if $y = 1$, then $x = 5$ and $z = 3$; or if $y = 0$, then
$x = 4$ and $z = 5$.  
\end{hint}
\begin{hint}
For part~(b), if we again substitute for $x$ using the first equation, the
second and third equations become
\[
2y - 3z = -2 \AND 2y - 3z = -4
\]
\end{hint}
\begin{hint}
These two expressions contradict each other, so there is no solution
for this system.  
\end{hint}
\end{exercise}

\begin{exercise} \label{c2.1.10}
Last year Dick was twice as old as Jane.  Four years ago the
sum of Dick's age and Jane's age was twice Jane's age now.  How
old are Dick and Jane?
\begin{prompt}
  Dick is $\answer{17}$ and Jane is $\answer{9}$.
\end{prompt}
\begin{hint}
  Rewrite the two statements
  as linear equations in $D$ --- Dick's age now --- and $J$ ---
  Jane's age now.  Then solve the system of linear equations.
\end{hint}
\end{exercise}

\begin{exercise} \label{c2.1.11}
\begin{itemize}
\item[(a)] Find a quadratic polynomial $p(x) = ax^2 + bx + c$
  satisfying $p(0) = 1$, $p(1) = 5$, and $p(-1) = -5$.
  \begin{prompt}
    The quadratic $p(x) = \answer{-x^2 + 5x + 1}$ satisfies these conditions.
  \end{prompt}
  \begin{hint}
    Since $p(x) = ax^2 + bx + c$ for any quadratic equation, we find
this solution by evaluating $p(0) = 1$, $p(1) = 5$, and $p(-1) = -5$,
which yields the system of equations
\[
\begin{array}{lrrrrrrrr}
p(0) & = & & & & & c & = & 1 \\
p(1) & = & a & + & b & + & c & = & 5 \\
p(-1) & = & a & - & b & + & c & = & -5\end{array}
\]
We solve this system to obtain $(a,b,c) = (-1,5,1)$, then substitute
these coefficients into the general quadratic.
  \end{hint}
\item[(b)] Prove that for every triple of real numbers $L$, $M$,
and $N$, there is a quadratic polynomial satisfying $p(0) = L$,
$p(1) = M$, and $p(-1) = N$.
\begin{hint}
   Let $p(x) = ax^2 + bx + c$ be a quadratic equation.  Then, the
assumptions $p(0) = L$, $p(1) = M$, and $p(-1) = N$ imply:
\[
\begin{array}{lrrrrrrrr}
p(0) & = & & & & & c & = & L \\
p(1) & = & a & + & b & + & c & = & M \\
p(-1) & = & a & - & b & + & c & = & N\end{array}
\]
The unique solution to this system is $(a,b,c) =
(\frac{M + N - 2L}{2},\frac{M - N}{2},L)$.
\end{hint}
\item[(c)] Let $x_1,x_2,x_3$ be three unequal real
numbers and let $A_1,A_2,A_3$ be three real numbers.  Show
that finding a quadratic polynomial $q(x)$ that satisfies
$q(x_i) = A_i$ is equivalent to solving a system of three
linear equations.
\begin{hint}
   Substituting $q(x_i) = A_i$, for $i = 1,2,3$, into the standard
quadratic equation $q(x) = ax^2 + bx + c$ yields
\[
\begin{array}{ccccccc}
ax_1^2 & + & bx_1 & + & c & = & A_1 \\
ax_2^2 & + & bx_2 & + & c & = & A_2 \\
ax_3^2 & + & bx_3 & + & c & = & A_3\end{array}
\]
Finding the appropriate quadratic polynomial would be equivalent to
solving this system of linear equations for $a$, $b$, and $c$ in
terms of $A_1$, $A_2$, and $A_3$.
\end{hint}
\end{itemize}
\end{exercise}

\CEXER

\begin{exercise} \label{c2.1.1}
Using \Matlab type the commands {\tt e2\_1\_8} and {\tt e2\_1\_9}
to load the matrices:
\begin{matlabEquation}\label{MATLAB:15}
A = \left(
\begin{array}{rrrrrr}
   -5.6 &  0.4 & -9.8 &  8.6 &  4.0 & -3.4\\
   -9.1 &  6.6 & -2.3 &  6.9 &  8.2 &  2.7\\
    3.6 & -9.3 & -8.7 &  0.5 &  5.2 &  5.1\\
    3.6 & -8.9 & -1.7 & -8.2 & -4.8 &  9.8\\
    8.7 &  0.6 &  3.7 &  3.1 & -9.1 & -2.7\\
   -2.3 &  3.4 &  1.8 & -1.7 &  4.7 & -5.1
\end{array}
\right)
\end{matlabEquation}
and the {\em vector\/}
\begin{matlabEquation}\label{MATLAB:16}
b = \left(
\begin{array}{r}
    9.7\\
    4.5\\
    5.1\\
    3.0\\
   -8.5\\
    2.6
\end{array}
\right)
\end{matlabEquation}
Solve the corresponding system of linear equations.
\end{exercise}

\begin{exercise} \label{c2.1.2}
Matrices are entered in \Matlab as follows. To enter
the $2\times 3$ matrix $A$, type {\tt A = [ -1 1 2; 4 1 2]}.
Enter this matrix into \Matlabp; the displayed matrix should be
\begin{verbatim}
A =
    -1     1     2
     4     1     2
\end{verbatim}
Now change the entry in the $2^{nd}$ row, $1^{st}$ column to
$-5$.
\end{exercise}

\begin{exercise} \label{c2.1.3}
Column vectors with $n$ entries are viewed by \Matlab as
$n\times 1$ matrices.  Enter the vector {\tt b = [1; 2; -4]}.
Then change the $3^{rd}$ entry in {\tt b} to $13$.
\end{exercise}


\begin{exercise} \label{c2.1.4}
This problem illustrates some of the different ways that \Matlab
displays numbers using the {\tt format long}, the {\tt format short} and
the {\tt format rational} commands.

Use \Matlab to solve the following system of equations
\[
\arraycolsep 2pt
\begin{array}{rcrcrcrl}
 2x_1 & - & 4.5x_2 & + & 3.1x_3 & = &   4.2  & \\
  x_1 & + &  x_2 & + &  x_3 & = &  -5.1  & \\
  x_1 & - & 6.2x_2 & + &  x_3 & = &  1.3  & \! .
\end{array}
\]
You may change the format of your answer in \Matlabp.  For
example, to print your result with an accuracy of $15$ digits
type {\tt format long} \index{\computer!format!long} and redisplay the
answer.  Similarly, to print your result as fractions type {\tt
format rational} \index{\computer!format!rational} and redisplay your
answer.
\end{exercise}

\begin{exercise} \label{c2.1.5}
Enter the following matrix and vector into \Matlab
\begin{verbatim}
 A = [ 1 0 -1 ; 2 5 3 ; 5 -1 0];
 b = [ 1; 1; -2];
\end{verbatim}
and solve the corresponding system of linear equations by typing
\begin{verbatim}
x = A\b
\end{verbatim}
Your answer should be
\begin{verbatim}
x =
   -0.2000
    1.0000
   -1.2000
\end{verbatim}
Find an integer for the entry in the $2^{nd}$ row, $2^{nd}$ column
of $A$ so that the solution
\begin{verbatim}
x = A\b
\end{verbatim}
is not defined.  {\bf Hint:} The answer is an integer between
$-4$ and $4$.
\end{exercise}

\begin{exercise} \label{c2.1.6}
The \Matlab command {\tt rand(m,n)}\index{\computer!rand} defines
matrices with random
entries between $0$ and $1$.  For example, the command {\tt A =
rand(5,5)} generates a random $5\times 5$ matrix, whereas the
command {\tt b = rand(5,1)} generates a column vector with $5$
random entries.  Use these commands to construct several systems
of linear equations and then solve them.
\end{exercise}

\begin{exercise} \label{c2.1.7}
Suppose that the four substances $S_1$, $S_2$, $S_3$, $S_4$
contain the following percentages of vitamins A, B, C and F by
weight
\begin{center}
\begin{tabular}{|c||r|r|r|r|}
\hline
Vitamin   & $S_1$ & $S_2$ & $S_3$ & $S_4$\\
\hline
 A & 25\% &    19\% &    20\% &    3\% \\
 B &  2\% &    14\% &     2\% &   14\% \\
 C &  8\% &     4\% &     1\% &     0\% \\
 F & 25\% &    31\% &    25\% &    16\% \\
\hline
\end{tabular}
\end{center}
Mix the substances $S_1$, $S_2$, $S_3$ and $S_4$ so that the
resulting mixture contains precisely $3.85$ grams of vitamin A,
$2.30$ grams of vitamin B, $0.80$ grams of vitamin C, and $5.95$
grams of vitamin F.  How many grams of each substance have to be
contained in the mixture?

Discuss what happens if we require that the resulting mixture contains
$2.00$ grams of vitamin B instead of $2.30$ grams.
\end{exercise}



\end{document}

%%% Local Variables:
%%% mode: latex
%%% TeX-master: "../linearAlgebra"
%%% End:
