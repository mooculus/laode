\documentclass{ximera}

\usepackage{epsfig}

\graphicspath{
  {./}
  {figures/}
}


\usepackage{morewrites}

%\newcounter{ccounter}
%\setcounter{ccounter}{1}
%\newcommand{\Chapter}[1]{\setcounter{chapter}{\arabic{ccounter}}\chapter{#1}\addtocounter{ccounter}{1}}

%\newcommand{\section}[1]{\section{#1}\setcounter{thm}{0}\setcounter{equation}{0}}

%\renewcommand{\theequation}{\arabic{chapter}.\arabic{section}.\arabic{equation}}
%\renewcommand{\thefigure}{\arabic{chapter}.\arabic{figure}}
%\renewcommand{\thetable}{\arabic{chapter}.\arabic{table}}

%\newcommand{\Sec}[2]{\section{#1}\markright{\arabic{ccounter}.\arabic{section}.#2}\setcounter{equation}{0}\setcounter{thm}{0}\setcounter{figure}{0}}

\newcommand{\Sec}[2]{\section{#1}}

\setcounter{secnumdepth}{2}
%\setcounter{secnumdepth}{1} 

%\newcounter{THM}
%\renewcommand{\theTHM}{\arabic{chapter}.\arabic{section}}

\newcommand{\trademark}{{R\!\!\!\!\!\bigcirc}}
%\newtheorem{exercise}{}

\newcommand{\dfield}{{\sf dfield9}}
\newcommand{\pplane}{{\sf pplane9}}

\newcommand{\EXER}{\section*{Exercises}}%\vspace*{0.2in}\hrule\small\setcounter{exercise}{0}}
\newcommand{\CEXER}{}%\vspace{0.08in}\begin{center}Computer Exercises\end{center}}
\newcommand{\TEXER}{} %\vspace{0.08in}\begin{center}Hand Exercises\end{center}}
\newcommand{\AEXER}{} %\vspace{0.08in}\begin{center}Hand Exercises\end{center}}

% BADBAD: \newcommand{\Bbb}{\bf}

\newcommand{\R}{\mbox{$\Bbb{R}$}}
\newcommand{\C}{\mbox{$\Bbb{C}$}}
\newcommand{\Z}{\mbox{$\Bbb{Z}$}}
\newcommand{\N}{\mbox{$\Bbb{N}$}}
\newcommand{\D}{\mbox{{\bf D}}}
\usepackage{amssymb}
%\newcommand{\qed}{\hfill\mbox{\raggedright$\square$} \vspace{1ex}}
%\newcommand{\proof}{\noindent {\bf Proof:} \hspace{0.1in}}

\newcommand{\setmin}{\;\mbox{--}\;}
\newcommand{\Matlab}{{M\small{AT\-LAB}} }
\newcommand{\Matlabp}{{M\small{AT\-LAB}}}
\newcommand{\computer}{\Matlab Instructions}
\newcommand{\half}{\mbox{$\frac{1}{2}$}}
\newcommand{\compose}{\raisebox{.15ex}{\mbox{{\scriptsize$\circ$}}}}
\newcommand{\AND}{\quad\mbox{and}\quad}
\newcommand{\vect}[2]{\left(\begin{array}{c} #1_1 \\ \vdots \\
 #1_{#2}\end{array}\right)}
\newcommand{\mattwo}[4]{\left(\begin{array}{rr} #1 & #2\\ #3
&#4\end{array}\right)}
\newcommand{\mattwoc}[4]{\left(\begin{array}{cc} #1 & #2\\ #3
&#4\end{array}\right)}
\newcommand{\vectwo}[2]{\left(\begin{array}{r} #1 \\ #2\end{array}\right)}
\newcommand{\vectwoc}[2]{\left(\begin{array}{c} #1 \\ #2\end{array}\right)}



\newcommand{\inv}{^{-1}}
\newcommand{\CC}{{\cal C}}
\newcommand{\CCone}{\CC^1}
\newcommand{\Span}{{\rm span}}
\newcommand{\rank}{{\rm rank}}
\newcommand{\trace}{{\rm tr}}
\newcommand{\RE}{{\rm Re}}
\newcommand{\IM}{{\rm Im}}
\newcommand{\nulls}{{\rm null\;space}}

\newcommand{\dps}{\displaystyle}
\newcommand{\arraystart}{\renewcommand{\arraystretch}{1.8}}
\newcommand{\arrayfinish}{\renewcommand{\arraystretch}{1.2}}
\newcommand{\Start}[1]{\vspace{0.08in}\noindent {\bf Section~\ref{#1}}}
\newcommand{\exer}[1]{\noindent {\bf \ref{#1}}}
\newcommand{\ans}{}
\newcommand{\matthree}[9]{\left(\begin{array}{rrr} #1 & #2 & #3 \\ #4 & #5 & #6
\\ #7 & #8 & #9\end{array}\right)}
\newcommand{\cvectwo}[2]{\left(\begin{array}{c} #1 \\ #2\end{array}\right)}
\newcommand{\cmatthree}[9]{\left(\begin{array}{ccc} #1 & #2 & #3 \\ #4 & #5 &
#6 \\ #7 & #8 & #9\end{array}\right)}
\newcommand{\vecthree}[3]{\left(\begin{array}{r} #1 \\ #2 \\
#3\end{array}\right)}
\newcommand{\cvecthree}[3]{\left(\begin{array}{c} #1 \\ #2 \\
#3\end{array}\right)}
\newcommand{\cmattwo}[4]{\left(\begin{array}{cc} #1 & #2\\ #3
&#4\end{array}\right)}

\newcommand{\Matrix}[1]{\ensuremath{\left(\begin{array}{rrrrrrrrrrrrrrrrrr} #1 \end{array}\right)}}

\newcommand{\Matrixc}[1]{\ensuremath{\left(\begin{array}{cccccccccccc} #1 \end{array}\right)}}



\renewcommand{\labelenumi}{\theenumi)}
\newenvironment{enumeratea}%
{\begingroup
 \renewcommand{\theenumi}{\alph{enumi}}
 \renewcommand{\labelenumi}{(\theenumi)}
 \begin{enumerate}}
 {\end{enumerate}\endgroup}



\newcounter{help}
\renewcommand{\thehelp}{\thesection.\arabic{equation}}

%\newenvironment{equation*}%
%{\renewcommand\endequation{\eqno (\theequation)* $$}%
%   \begin{equation}}%
%   {\end{equation}\renewcommand\endequation{\eqno \@eqnnum
%$$\global\@ignoretrue}}

%\input{psfig.tex}

\author{Martin Golubitsky and Michael Dellnitz}

%\newenvironment{matlabEquation}%
%{\renewcommand\endequation{\eqno (\theequation*) $$}%
%   \begin{equation}}%
%   {\end{equation}\renewcommand\endequation{\eqno \@eqnnum
% $$\global\@ignoretrue}}

\newcommand{\soln}{\textbf{Solution:} }
\newcommand{\exercap}[1]{\centerline{Figure~\ref{#1}}}
\newcommand{\exercaptwo}[1]{\centerline{Figure~\ref{#1}a\hspace{2.1in}
Figure~\ref{#1}b}}
\newcommand{\exercapthree}[1]{\centerline{Figure~\ref{#1}a\hspace{1.2in}
Figure~\ref{#1}b\hspace{1.2in}Figure~\ref{#1}c}}
\newcommand{\para}{\hspace{0.4in}}

\renewenvironment{solution}{\suppress}{\endsuppress}

\ifxake
\newenvironment{matlabEquation}{\begin{equation}}{\end{equation}}
\else
\newenvironment{matlabEquation}%
{\let\oldtheequation\theequation\renewcommand{\theequation}{\oldtheequation*}\begin{equation}}%
  {\end{equation}\let\theequation\oldtheequation}
\fi

\makeatother


\title{Linear Equations with Special Coefficients}

\begin{document}
\begin{abstract}
\end{abstract}
\maketitle


\label{S:specialcoeff}

In this chapter we have shown how to use elementary row
operations to solve systems of linear equations.  We have
assumed that each linear equation in the system has the form
\[
a_{j1}x_1 + \cdots + a_{jn}x_n = b_j,
\]
where the $a_{ji}$s and the $b_j$s are real numbers.  For
simplicity, in our examples we have only chosen equations with
integer coefficients --- such as:
\[
2x_1 - 3x_2 +15x_3 = -1.
\]

\subsection*{Systems with Nonrational Coefficients}

In fact, a more general choice of coefficients for a system of
two equations might have been
\begin{eqnarray}
\sqrt{2}x_1 + 2\pi x_2 & = & 22.4  \nonumber \\
3x_1+36.2 x_2 & = & e. \label{e:irrat}
\end{eqnarray} \index{\computer!sqrt}  \index{\computer!pi}
\index{\computer!exp(1)}

Suppose that we solve \Ref{e:irrat} by elementary row
operations.  In  matrix form we have the augmented matrix
\[
\left(\begin{array}{cc|c} \sqrt{2} & 2\pi & 22.4\\
3 & 36.2 & e\end{array}\right).
\]
Proceed with the following elementary row operations. Divide the
$1^{st}$ row by $\sqrt{2}$ to obtain
\[
\left(\begin{array}{cc|c} 1 & \pi\sqrt{2} & 11.2\sqrt{2}\\
3 & 36.2 & e\end{array}\right).
\]
Next, subtract $3$ times the $1^{st}$ row from the $2^{nd}$ row
to obtain:
\[
\left(\begin{array}{cc|c} 1 & \pi\sqrt{2} & 11.2\sqrt{2}\\
0 & 36.2- 3\pi\sqrt{2} & e- 33.6\sqrt{2}\end{array}\right).
\]
Then divide the $2^{nd}$ row by $36.2-3\pi\sqrt{2}$,
obtaining:
\[
\left(\begin{array}{cc|c} 1 & \pi\sqrt{2} & 11.2\sqrt{2}\\
0 & 1 & \frac{e-33.6\sqrt{2}}{36.2-3\pi\sqrt{2}}\end{array}\right).
\]
Finally, multiply the $2^{nd}$ row by $\pi\sqrt{2}$ and
subtract it from the $1^{st}$ row to obtain:
\[
\left(\begin{array}{cc|c} 1 & 0 &
11.2\sqrt{2}-\pi\sqrt{2}\frac{e-33.6\sqrt{2}}{36.2-3\pi\sqrt{2}} \\
0 & 1 & \frac{e-33.6\sqrt{2}}{36.2-3\pi\sqrt{2}}\end{array}\right).
\]
So
\begin{eqnarray}
x_1 & = &  11.2\sqrt{2}-\pi\sqrt{2}\frac{e-33.6\sqrt{2}}{36.2-3\pi\sqrt{2}}
\nonumber
\\
  &  &   \label{e:irratans} \\
x_2 & = & \frac{e-33.6\sqrt{2}}{36.2-3\pi\sqrt{2}} \nonumber
\end{eqnarray}
which is both hideous to look at and quite uninformative.  It is,
however, correct.

Both $x_1$ and $x_2$ are real numbers --- they had to be because
all of the manipulations involved addition, subtraction,
multiplication, and division of real numbers --- which yield
real numbers.

If we wanted to use \Matlab to perform these calculations, we
have to convert $\sqrt{2}$, $\pi$, and $e$ to their
decimal equivalents --- at least up to a certain decimal place
accuracy. This introduces errors --- which for the moment we
assume are small.

To enter $A$ and $b$ in \Matlab, type
\begin{verbatim}
A = [sqrt(2) 2*pi; 3 36.2];
b = [22.4; exp(1)];
\end{verbatim}
Now type {\tt A} to obtain:
\begin{verbatim}
A =
    1.4142    6.2832
    3.0000   36.2000
\end{verbatim}
As its default display, \Matlab displays real numbers to four
decimal place accuracy.  Similarly, type {\tt b} to obtain
\begin{verbatim}
b =
   22.4000
    2.7183
\end{verbatim}


Next use \Matlab to solve this system by typing:
\begin{verbatim}
A\b
\end{verbatim}
to obtain
\begin{verbatim}
ans =
   24.5417
   -1.9588
\end{verbatim}

The reader may check that this answer agrees with the answer in
\Ref{e:irratans} to \Matlab output accuracy by typing
\begin{verbatim}
x1 = 11.2*sqrt(2)-pi*sqrt(2)*(exp(1)-33.6*sqrt(2))/(36.2-3*pi*sqrt(2))
x2 = (exp(1)-33.6*sqrt(2))/(36.2-3*pi*sqrt(2))
\end{verbatim}
to obtain
\begin{verbatim}
x1 =
   24.5417
\end{verbatim}
and
\begin{verbatim}
x2 =
   -1.9588
\end{verbatim}


\subsubsection*{More Accuracy}

\Matlab can display numbers in machine precision (15 digits) rather
than the standard four decimal place accuracy. To change to this
display, type
\begin{verbatim}
format long
\end{verbatim} \index{\computer!format!long}
Now solve the system of equations \Ref{e:irrat} again by typing
\begin{verbatim}
A\b
\end{verbatim}  \index{\computer!$\backslash$}
and obtaining
\begin{verbatim}
ans =
  24.54169560069650
  -1.95875151860858
\end{verbatim}


\subsection*{Integers and Rational Numbers}

Now suppose that all of the coefficients in a system of linear
equations are integers.  When we add, subtract or multiply
integers --- we get integers.  In general, however, when we
divide an integer by an integer we get a rational number rather
than an integer.  Indeed, since elementary row operations
involve only the operations of addition, subtraction,
multiplication and division, we see that if we perform
elementary row operations on a matrix with integer entries, we
will end up with a matrix with rational numbers as entries.

\Matlab can display calculations using rational numbers rather
than decimal numbers.  To display calculations using only
rational numbers, type
\begin{verbatim}
format rational
\end{verbatim} \index{\computer!format!rational}
For example, let
\begin{equation*} \label{e:A4}
A=\left(\begin{array}{ccrc} 2 & 2 & 1 & 0\\ 1 & 3 & -5 & 1\\
4 & 2 & 1 & 3 \\ 2 & 1 & -1 & 4\end{array}\right)
\end{equation*}
and let
\begin{equation*} \label{e:b4}
b=\left(\begin{array}{r} 1 \\ 1 \\ -5 \\2 \end{array} \right).
\end{equation*}
Enter $A$ and $b$ into \Matlab by typing
\begin{verbatim}
e2_5_3
e2_5_4
\end{verbatim}
Solve the system by typing
\begin{verbatim}
A\b
\end{verbatim}
to obtain
\begin{verbatim}
ans =
  -357/41
   309/41
   137/41
   156/41
\end{verbatim}
To display the answer in standard decimal form, type
\begin{verbatim}
format
A\b
\end{verbatim}
obtaining
\begin{verbatim}
ans =
   -8.7073
    7.5366
    3.3415
    3.8049
\end{verbatim}


The same logic shows that if we begin with a system of equations
whose coefficients are rational numbers, we will obtain an
answer consisting of rational numbers --- since adding,
subtracting, multiplying and dividing rational numbers yields
rational numbers. More precisely:
\begin{theorem}
Let $A$ be an $n\times n$ matrix that is row equivalent to
$I_n$, and let $b$ be an $n$ vector.  Suppose that all entries of
$A$ and $b$ are rational numbers.  Then there is a unique
solution to the system corresponding to the augmented matrix
$(A|b)$ and this solution has rational numbers as entries.
\end{theorem}

\begin{proof} Since $A$ is row equivalent to $I_n$,
Corollary~\ref{consistent} states that this linear system
has a unique solution $x$.  As we have just discussed, solutions
are found using elementary row operations --- hence the entries
of $x$ are rational numbers.  \end{proof}


\subsection*{Complex Numbers} \index{complex numbers}

In the previous parts of this section, we have discussed why
solutions to linear systems whose coefficients are rational
numbers must themselves have entries that are rational numbers.
We now discuss solving linear equations whose coefficients are
more general than real numbers; that is, whose coefficients are
complex numbers.

First recall that addition, subtraction, multiplication and
division of complex numbers yields complex numbers.  Suppose
that
\begin{eqnarray*}
a & = & \alpha + i\beta \\
b & = & \gamma + i\delta
\end{eqnarray*}
where $\alpha,\beta,\gamma,\delta$ are real numbers and
$i=\sqrt{-1}$. Then
\begin{eqnarray*}
a+b & = & (\alpha+\gamma) + i(\beta+\delta) \\
a-b & = & (\alpha-\gamma) + i(\beta-\delta) \\
ab & = &
(\alpha\gamma-\beta\delta)+i(\alpha\delta+\beta\gamma)\\
\frac{a}{b} & = & \frac{\alpha\gamma+\beta\delta}{\gamma^2+\delta^2}+
i\frac{\beta\gamma-\alpha\delta}{\gamma^2+\delta^2}
\end{eqnarray*}

\Matlab has been programmed to do arithmetic with complex numbers
using exactly the same instructions as it uses to do arithmetic
with real and rational numbers.  For example, we can solve the
system of linear equations
\begin{eqnarray*}
(4-i)x_1+2x_2 & = & 3-i \\
2x_1 +(4-3i)x_2 & = & 2+i
\end{eqnarray*}
in \Matlab by typing
\begin{verbatim}
A = [4-i 2; 2 4-3i];
b = [3-i; 2+i];
A\b
\end{verbatim} \index{\computer!$\backslash$}
The solution to this system of equations is:
\begin{verbatim}
ans =
   0.8457 - 0.1632i
  -0.1098 + 0.2493i
\end{verbatim}  \index{\computer!i}

\footnotesize
\noindent {\bf Note:} Care must be given when entering complex
numbers into arrays in \Matlabp.  For example, if you type
\begin{verbatim}
b = [3 -i; 2 +i]
\end{verbatim}
then \Matlab will respond with the $2\times 2$ matrix
\begin{verbatim}
b =
   3.0000                  0 - 1.0000i
   2.0000                  0 + 1.0000i
\end{verbatim}
Typing either {\tt b = [3-i; 2+i]} or {\tt b = [3 - i; 2 + i]} will
yield the desired $2\times 1$ column vector.

\normalsize

All of the theorems concerning the existence and uniqueness
of row echelon form --- and for solving systems of linear
equations --- work when the coefficients of the linear system
are complex numbers as opposed to real numbers.  In particular:

\begin{theorem}  \label{T:complexcoeff}
If the coefficients of a system of $n$ linear equations in $n$
unknowns are complex numbers and if the coefficient matrix is
row equivalent to $I_n$, then there is a unique solution to
this system whose entries are complex numbers.
\end{theorem} \index{row!equivalent} \index{matrix!identity}

\subsubsection*{Complex Conjugation}

Let $a=\alpha+i\beta$ be a complex number.  Then the {\em complex
conjugate\/} \index{complex conjugation} of $a$ is defined to be
\[
\overline{a} = \alpha - i\beta.
\]
Let $a=\alpha+i\beta$ and $c=\gamma+i\delta$ be complex numbers.  Then we
claim that
\begin{equation}
\begin{array}{rcl}
\overline{a+c} & = & \overline{a} + \overline{c}\\
\overline{ac}  & = & \overline{a}\;\overline{c}
\end{array}
\end{equation}
To verify these statements, calculate
\[
\overline{a+c} = \overline{(\alpha+\gamma)+i(\beta+\delta)} =
(\alpha+\gamma)-i(\beta+\delta)= (\alpha-i\beta) + (\gamma-i\delta)
= \overline{a} + \overline{c}
\]
and
\[
\overline{ac} =
\overline{(\alpha\gamma-\beta\delta)+i(\alpha\delta+\beta\gamma)}
= (\alpha\gamma-\beta\delta)-i(\alpha\delta+\beta\gamma)
= (\alpha-i\beta)(\gamma-i\delta)=\overline{a}\;\overline{c}.
\]

\EXER

\TEXER

\begin{exercise} \label{c2.5.1}
Solve the system of equations
\[
\begin{array}{rcrcr}
 x_1 & - & ix_2 & = &  1\\
ix_1 & + & 3x_2 & = & -1
\end{array}
\]
Check your answer using \Matlabp.
\end{exercise}

\noindent Solve the systems of linear equations given in Exercises~\ref{c2.5.1A}
-- \ref{c2.5.1B} and verify that the answers are rational numbers. 
\begin{exercise} \label{c2.5.1A}
\qquad $\begin{array}{rcrcrcr}
         x_1 & + &   x_2 & - &   2x_3 & = & 1 \\
         x_1 & + &   x_2 & + &    x_3 & = & 2 \\
         x_1 & - &  7x_2 & + &    x_3 & = & 3
\end{array}$
\end{exercise}
\begin{exercise} \label{c2.5.1B}
\qquad $\begin{array}{rcrcr}
 x_1 & - &  x_2 & = &  1\\
 x_1 & + & 3x_2 & = & -1
\end{array}$
\end{exercise}

\CEXER

\noindent In Exercises~\ref{c2.5.2a} -- \ref{c2.5.2c} use \Matlab to
solve the given system of linear equations to four significant decimal places.
\begin{exercise} \label{c2.5.2a}
\[
\begin{array}{rcrcrcr}
     0.1 x_1 & + & \sqrt{5}x_2 & - &   2 x_3 & = & 1 \\
-\sqrt{3}x_1 & + &     \pi x_2 & - & 2.6 x_3 & = & 14.3 \\
         x_1 & - &       7 x_2 & + & \frac{\pi}{2}x_3 & = & \sqrt{2}
\end{array}.
\]
\end{exercise}
\begin{exercise} \label{c2.5.2b}
\[
\begin{array}{rcrcr}
(4-i)x_1 & + & (2+3i)x_2 & = &  -i \\
   i x_1 & - &     4 x_2 & = & 2.2
\end{array}.
\]
\end{exercise}
\begin{exercise} \label{c2.5.2c}
\[
\begin{array}{rcrcrcr}
 (2+i) x_1 & + & (\sqrt{2}-3i)x_2 & - &    10.66 x_3 & = &     4.23 \\
    14 x_1 & - &    \sqrt{5}i x_2 & + & (10.2-i) x_3 & = &    3-1.6i \\
-4.276 x_1 & + &            2 x_2 & - &   (4-2i) x_3 & = & \sqrt{2}i
\end{array}.
\]

\noindent {\bf Hint:}  When entering $\sqrt{2}i$ in \Matlab you must type
{\tt sqrt(2)*i}, even though when you enter $2i$, you can just type
{\tt 2i}.
\end{exercise}


\end{document}
