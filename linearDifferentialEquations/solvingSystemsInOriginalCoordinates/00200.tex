\documentclass{ximera}
\usepackage{epsfig}

\graphicspath{
  {./}
  {figures/}
}


\usepackage{morewrites}

%\newcounter{ccounter}
%\setcounter{ccounter}{1}
%\newcommand{\Chapter}[1]{\setcounter{chapter}{\arabic{ccounter}}\chapter{#1}\addtocounter{ccounter}{1}}

%\newcommand{\section}[1]{\section{#1}\setcounter{thm}{0}\setcounter{equation}{0}}

%\renewcommand{\theequation}{\arabic{chapter}.\arabic{section}.\arabic{equation}}
%\renewcommand{\thefigure}{\arabic{chapter}.\arabic{figure}}
%\renewcommand{\thetable}{\arabic{chapter}.\arabic{table}}

%\newcommand{\Sec}[2]{\section{#1}\markright{\arabic{ccounter}.\arabic{section}.#2}\setcounter{equation}{0}\setcounter{thm}{0}\setcounter{figure}{0}}

\newcommand{\Sec}[2]{\section{#1}}

\setcounter{secnumdepth}{2}
%\setcounter{secnumdepth}{1} 

%\newcounter{THM}
%\renewcommand{\theTHM}{\arabic{chapter}.\arabic{section}}

\newcommand{\trademark}{{R\!\!\!\!\!\bigcirc}}
%\newtheorem{exercise}{}

\newcommand{\dfield}{{\sf dfield9}}
\newcommand{\pplane}{{\sf pplane9}}

\newcommand{\EXER}{\section*{Exercises}}%\vspace*{0.2in}\hrule\small\setcounter{exercise}{0}}
\newcommand{\CEXER}{}%\vspace{0.08in}\begin{center}Computer Exercises\end{center}}
\newcommand{\TEXER}{} %\vspace{0.08in}\begin{center}Hand Exercises\end{center}}
\newcommand{\AEXER}{} %\vspace{0.08in}\begin{center}Hand Exercises\end{center}}

% BADBAD: \newcommand{\Bbb}{\bf}

\newcommand{\R}{\mbox{$\Bbb{R}$}}
\newcommand{\C}{\mbox{$\Bbb{C}$}}
\newcommand{\Z}{\mbox{$\Bbb{Z}$}}
\newcommand{\N}{\mbox{$\Bbb{N}$}}
\newcommand{\D}{\mbox{{\bf D}}}
\usepackage{amssymb}
%\newcommand{\qed}{\hfill\mbox{\raggedright$\square$} \vspace{1ex}}
%\newcommand{\proof}{\noindent {\bf Proof:} \hspace{0.1in}}

\newcommand{\setmin}{\;\mbox{--}\;}
\newcommand{\Matlab}{{M\small{AT\-LAB}} }
\newcommand{\Matlabp}{{M\small{AT\-LAB}}}
\newcommand{\computer}{\Matlab Instructions}
\newcommand{\half}{\mbox{$\frac{1}{2}$}}
\newcommand{\compose}{\raisebox{.15ex}{\mbox{{\scriptsize$\circ$}}}}
\newcommand{\AND}{\quad\mbox{and}\quad}
\newcommand{\vect}[2]{\left(\begin{array}{c} #1_1 \\ \vdots \\
 #1_{#2}\end{array}\right)}
\newcommand{\mattwo}[4]{\left(\begin{array}{rr} #1 & #2\\ #3
&#4\end{array}\right)}
\newcommand{\mattwoc}[4]{\left(\begin{array}{cc} #1 & #2\\ #3
&#4\end{array}\right)}
\newcommand{\vectwo}[2]{\left(\begin{array}{r} #1 \\ #2\end{array}\right)}
\newcommand{\vectwoc}[2]{\left(\begin{array}{c} #1 \\ #2\end{array}\right)}



\newcommand{\inv}{^{-1}}
\newcommand{\CC}{{\cal C}}
\newcommand{\CCone}{\CC^1}
\newcommand{\Span}{{\rm span}}
\newcommand{\rank}{{\rm rank}}
\newcommand{\trace}{{\rm tr}}
\newcommand{\RE}{{\rm Re}}
\newcommand{\IM}{{\rm Im}}
\newcommand{\nulls}{{\rm null\;space}}

\newcommand{\dps}{\displaystyle}
\newcommand{\arraystart}{\renewcommand{\arraystretch}{1.8}}
\newcommand{\arrayfinish}{\renewcommand{\arraystretch}{1.2}}
\newcommand{\Start}[1]{\vspace{0.08in}\noindent {\bf Section~\ref{#1}}}
\newcommand{\exer}[1]{\noindent {\bf \ref{#1}}}
\newcommand{\ans}{}
\newcommand{\matthree}[9]{\left(\begin{array}{rrr} #1 & #2 & #3 \\ #4 & #5 & #6
\\ #7 & #8 & #9\end{array}\right)}
\newcommand{\cvectwo}[2]{\left(\begin{array}{c} #1 \\ #2\end{array}\right)}
\newcommand{\cmatthree}[9]{\left(\begin{array}{ccc} #1 & #2 & #3 \\ #4 & #5 &
#6 \\ #7 & #8 & #9\end{array}\right)}
\newcommand{\vecthree}[3]{\left(\begin{array}{r} #1 \\ #2 \\
#3\end{array}\right)}
\newcommand{\cvecthree}[3]{\left(\begin{array}{c} #1 \\ #2 \\
#3\end{array}\right)}
\newcommand{\cmattwo}[4]{\left(\begin{array}{cc} #1 & #2\\ #3
&#4\end{array}\right)}

\newcommand{\Matrix}[1]{\ensuremath{\left(\begin{array}{rrrrrrrrrrrrrrrrrr} #1 \end{array}\right)}}

\newcommand{\Matrixc}[1]{\ensuremath{\left(\begin{array}{cccccccccccc} #1 \end{array}\right)}}



\renewcommand{\labelenumi}{\theenumi)}
\newenvironment{enumeratea}%
{\begingroup
 \renewcommand{\theenumi}{\alph{enumi}}
 \renewcommand{\labelenumi}{(\theenumi)}
 \begin{enumerate}}
 {\end{enumerate}\endgroup}



\newcounter{help}
\renewcommand{\thehelp}{\thesection.\arabic{equation}}

%\newenvironment{equation*}%
%{\renewcommand\endequation{\eqno (\theequation)* $$}%
%   \begin{equation}}%
%   {\end{equation}\renewcommand\endequation{\eqno \@eqnnum
%$$\global\@ignoretrue}}

%\input{psfig.tex}

\author{Martin Golubitsky and Michael Dellnitz}

%\newenvironment{matlabEquation}%
%{\renewcommand\endequation{\eqno (\theequation*) $$}%
%   \begin{equation}}%
%   {\end{equation}\renewcommand\endequation{\eqno \@eqnnum
% $$\global\@ignoretrue}}

\newcommand{\soln}{\textbf{Solution:} }
\newcommand{\exercap}[1]{\centerline{Figure~\ref{#1}}}
\newcommand{\exercaptwo}[1]{\centerline{Figure~\ref{#1}a\hspace{2.1in}
Figure~\ref{#1}b}}
\newcommand{\exercapthree}[1]{\centerline{Figure~\ref{#1}a\hspace{1.2in}
Figure~\ref{#1}b\hspace{1.2in}Figure~\ref{#1}c}}
\newcommand{\para}{\hspace{0.4in}}

\renewenvironment{solution}{\suppress}{\endsuppress}

\ifxake
\newenvironment{matlabEquation}{\begin{equation}}{\end{equation}}
\else
\newenvironment{matlabEquation}%
{\let\oldtheequation\theequation\renewcommand{\theequation}{\oldtheequation*}\begin{equation}}%
  {\end{equation}\let\theequation\oldtheequation}
\fi

\makeatother

\begin{document}

\noindent  In Exercises~\ref{c12.1.6a} -- \ref{c12.1.6d} use Corollary~\ref{T:etA0} to compute $e^{tA}$ for the given matrix.
\begin{exercise} \label{c12.1.6a}
$A = \mattwo{0}{1}{1}{0}$.

\begin{solution}
\ans
\[
e^{tA} = -e^t\frac{1}{2}\mattwo{1}{1}{1}{1} +
e^{-t}\frac{1}{2}\mattwo{-1}{1}{1}{-1}
= -\frac{1}{2}\cmattwo{e^{t} + e^{-t}}{e^t - e^{-t}}{e^t - e^{-t}}
{e^t + e^{-t}}.
\]

\soln First find the eigenvalues of $A$, which are $\lambda_1 = 1$ and
$\lambda_2 = -1$.  By Corollary~\ref{T:etA0},
\[
e^{tA} = \sum_{\ell = 1}^2 e^{\lambda_\ell t}a_\ell(A)P_\ell(A).
\]
The characteristic polynomial of $A$ is $p_A(\lambda) = (\lambda -
1)(\lambda + 1)$, so $P_1(\lambda) = \lambda + 1$ and $P_2(\lambda) =
\lambda - 1$.  Use partial fractions to find that
\[
\frac{1}{p_A(\lambda)} = \frac{-\frac{1}{2}}{1 - \lambda} + \frac{\frac{1}{2}}
{-1 - \lambda}.
\]
Thus, $a_1(\lambda) = -\frac{1}{2}$ and $a_2(\lambda) = \frac{1}{2}$.  So
\[
\begin{array}{rcl}
e^{tA} & = & e^ta_1(A)P_1(A) + e^{-t}a_2(A)P_2(A) \\
& = & \dps e^t\left(-\frac{1}{2}\right)\left(\mattwo{0}{1}{1}{0} +
\mattwo{1}{0}{0}{1}\right) + e^{-t}\left(\frac{1}{2}\right)
\left(\mattwo{0}{1}{1}{0} - \mattwo{1}{0}{0}{1}\right).
\end{array}
\]

\end{solution}
\end{exercise}
\begin{exercise} \label{c12.1.6b}
$A = \mattwo{2}{1}{1}{2}$.

\begin{solution}
\ans
\[
e^{tA} = -e^{3t}\frac{1}{2}\mattwo{1}{1}{1}{1} +
e^t\frac{1}{2}\mattwo{-1}{1}{1}{-1}
= -\frac{1}{2}\cmattwo{e^{3t} + e^{t}}{e^{3t} - e^{t}}{e^{3t} - e^{t}}
{e^{3t} + e^{t}}.
\]

\soln First find the eigenvalues of $A$, which are $\lambda_1 = 3$ and
$\lambda_2 = 1$.  By Corollary~\ref{T:etA0},
\[
e^{tA} = \sum_{\ell = 1}^2 e^{\lambda_\ell t}a_\ell(A)P_\ell(A).
\]
The characteristic polynomial of $A$ is $p_A(\lambda) = (\lambda -
1)(\lambda - 3)$, so $P_1(\lambda) = \lambda - 1$ and $P_2(\lambda) =
\lambda - 3$.  Use partial fractions to find that
\[
\frac{1}{p_A(\lambda)} = \frac{-\frac{1}{2}}{3 - \lambda} + \frac{\frac{1}{2}}
{1 - \lambda}.
\]
Thus, $a_1(\lambda) = -\frac{1}{2}$ and $a_2(\lambda) = \frac{1}{2}$.  So
\[
\begin{array}{rcl}
e^{tA} & = & e^ta_1(A)P_1(A) + e^{-t}a_2(A)P_2(A) \\
& = & \dps e^{3t}\left(-\frac{1}{2}\right)\left(\mattwo{2}{1}{1}{2} -
\mattwo{1}{0}{0}{1}\right) + e^{t}\left(\frac{1}{2}\right)
\left(\mattwo{2}{1}{1}{2} - \mattwo{3}{0}{0}{3}\right).
\end{array}
\]

\end{solution}
\end{exercise}
\begin{exercise} \label{c12.1.6c}
$A = \mattwo{3}{2}{0}{1}$.

\begin{solution}
\ans
\[
e^{tA} = -e^{3t}\frac{1}{2}\mattwo{2}{2}{0}{0} +
e^t\frac{1}{2}\mattwo{0}{2}{0}{-2}
= -\cmattwo{e^{3t}}{e^{3t} - e^{t}}{0}{e^{t}}.
\]

\soln First find the eigenvalues of $A$, which are $\lambda_1 = 3$ and
$\lambda_2 = 1$.  By Corollary~\ref{T:etA0},
\[
e^{tA} = \sum_{\ell = 1}^2 e^{\lambda_\ell t}a_\ell(A)P_\ell(A).
\]
The characteristic polynomial of $A$ is $p_A(\lambda) = (\lambda -
1)(\lambda - 3)$, so $P_1(\lambda) = \lambda - 1$ and $P_2(\lambda) =
\lambda - 3$.  Use partial fractions to find that
\[
\frac{1}{p_A(\lambda)} = \frac{-\frac{1}{2}}{3 - \lambda} + \frac{\frac{1}{2}}
{1 - \lambda}.
\]
Thus, $a_1(\lambda) = -\frac{1}{2}$ and $a_2(\lambda) = \frac{1}{2}$.  So
\[
\begin{array}{rcl}
e^{tA} & = & e^ta_1(A)P_1(A) + e^{-t}a_2(A)P_2(A) \\
& = & \dps e^{3t}\left(-\frac{1}{2}\right)\left(\mattwo{3}{2}{0}{1} -
\mattwo{1}{0}{0}{1}\right) + e^{t}\left(\frac{1}{2}\right)
\left(\mattwo{3}{2}{0}{1} - \mattwo{3}{0}{0}{3}\right).
\end{array}
\]


\end{solution}
\end{exercise}
\begin{exercise} \label{c12.1.6d}
$A = \left(\begin{array}{rrr} 1 & -3 & 1\\ 0 & -2 & -5 \\ 0 & 0 & 3
\end{array}\right)$.

\begin{solution}
\ans
\[
e^{tA} =
\cmatthree{e^t}{-e^t + e^{-2t}}{2e^{3t} - e^t + e^{-2t}}
{0}{e^{-2t}}{-e^{3t} + e^{-2t}}
{0}{0}{e^{3t}}.
\]

\soln The characteristic polynomial of $A$ is
$p_A(\lambda) = (\lambda - 3)(\lambda - 1)(\lambda + 2)$, so the
eigenvalues are $\lambda_1 = 3$, $\lambda_2 = 1$, and $\lambda_3 = -2$.
By Corollary~\ref{T:etA0},
\[
e^{tA} = \sum_{\ell = 3}^2 e^{\lambda_\ell t}a_\ell(A)P_\ell(A).
\]
From the characteristic polynomial $p_A(\lambda)$, compute
\[
P_1(\lambda) = \lambda^2 + \lambda - 2, \quad
P_2(\lambda) = \lambda^2 - \lambda - 6, \AND
P_3(\lambda) = \lambda^2 - 4\lambda + 3.
\]
Use partial fractions to find that
\[
\frac{1}{p_A(\lambda)} =
\frac{\frac{1}{5}\lambda - \frac{1}{2}}{P_1(\lambda)}
+ \frac{-\frac{1}{6}\lambda}{P_2(\lambda)}
+ \frac{-\frac{1}{30}\lambda}{P_3(\lambda)}.
\]
Thus
\[
a_1(\lambda) = \frac{1}{5}\lambda - \frac{1}{2}, \quad
a_2(\lambda) = -\frac{1}{6}\lambda, \AND
a_3(\lambda) = -\frac{1}{30}\lambda.
\]
So,
\[
\begin{array}{rcl}
e^{tA} & = & e^{\lambda_1 t}a_1(A)P_1(A) + e^{\lambda_2 t}a_2(A)P_2(A)
+ e^{\lambda_3 t}a_3(A)P_3(A) \\
& = &
\begin{array}{l} e^{3t}(\frac{1}{5}A - \frac{1}{2})(A^2 + A - 2I_3)
+ e^t(-\frac{1}{6}A)(A^2 - A - 6I_3) \\
+\; e^{-2t}(-\frac{1}{30}A)(A^2 - 4A + 3I_2) \end{array} \\
& = & e^{3t}\matthree{0}{0}{2}{0}{0}{-1}{0}{0}{1}
+ e^t\matthree{1}{-1}{-3}{0}{0}{0}{0}{0}{0}
+ e^{-2t}\matthree{0}{1}{1}{0}{1}{1}{0}{0}{0}.
\end{array}
\]

\end{solution}
\end{exercise}
\end{document}
