\documentclass{ximera}
\usepackage{epsfig}

\graphicspath{
  {./}
  {figures/}
}


\usepackage{morewrites}

%\newcounter{ccounter}
%\setcounter{ccounter}{1}
%\newcommand{\Chapter}[1]{\setcounter{chapter}{\arabic{ccounter}}\chapter{#1}\addtocounter{ccounter}{1}}

%\newcommand{\section}[1]{\section{#1}\setcounter{thm}{0}\setcounter{equation}{0}}

%\renewcommand{\theequation}{\arabic{chapter}.\arabic{section}.\arabic{equation}}
%\renewcommand{\thefigure}{\arabic{chapter}.\arabic{figure}}
%\renewcommand{\thetable}{\arabic{chapter}.\arabic{table}}

%\newcommand{\Sec}[2]{\section{#1}\markright{\arabic{ccounter}.\arabic{section}.#2}\setcounter{equation}{0}\setcounter{thm}{0}\setcounter{figure}{0}}

\newcommand{\Sec}[2]{\section{#1}}

\setcounter{secnumdepth}{2}
%\setcounter{secnumdepth}{1} 

%\newcounter{THM}
%\renewcommand{\theTHM}{\arabic{chapter}.\arabic{section}}

\newcommand{\trademark}{{R\!\!\!\!\!\bigcirc}}
%\newtheorem{exercise}{}

\newcommand{\dfield}{{\sf dfield9}}
\newcommand{\pplane}{{\sf pplane9}}

\newcommand{\EXER}{\section*{Exercises}}%\vspace*{0.2in}\hrule\small\setcounter{exercise}{0}}
\newcommand{\CEXER}{}%\vspace{0.08in}\begin{center}Computer Exercises\end{center}}
\newcommand{\TEXER}{} %\vspace{0.08in}\begin{center}Hand Exercises\end{center}}
\newcommand{\AEXER}{} %\vspace{0.08in}\begin{center}Hand Exercises\end{center}}

% BADBAD: \newcommand{\Bbb}{\bf}

\newcommand{\R}{\mbox{$\Bbb{R}$}}
\newcommand{\C}{\mbox{$\Bbb{C}$}}
\newcommand{\Z}{\mbox{$\Bbb{Z}$}}
\newcommand{\N}{\mbox{$\Bbb{N}$}}
\newcommand{\D}{\mbox{{\bf D}}}
\usepackage{amssymb}
%\newcommand{\qed}{\hfill\mbox{\raggedright$\square$} \vspace{1ex}}
%\newcommand{\proof}{\noindent {\bf Proof:} \hspace{0.1in}}

\newcommand{\setmin}{\;\mbox{--}\;}
\newcommand{\Matlab}{{M\small{AT\-LAB}} }
\newcommand{\Matlabp}{{M\small{AT\-LAB}}}
\newcommand{\computer}{\Matlab Instructions}
\newcommand{\half}{\mbox{$\frac{1}{2}$}}
\newcommand{\compose}{\raisebox{.15ex}{\mbox{{\scriptsize$\circ$}}}}
\newcommand{\AND}{\quad\mbox{and}\quad}
\newcommand{\vect}[2]{\left(\begin{array}{c} #1_1 \\ \vdots \\
 #1_{#2}\end{array}\right)}
\newcommand{\mattwo}[4]{\left(\begin{array}{rr} #1 & #2\\ #3
&#4\end{array}\right)}
\newcommand{\mattwoc}[4]{\left(\begin{array}{cc} #1 & #2\\ #3
&#4\end{array}\right)}
\newcommand{\vectwo}[2]{\left(\begin{array}{r} #1 \\ #2\end{array}\right)}
\newcommand{\vectwoc}[2]{\left(\begin{array}{c} #1 \\ #2\end{array}\right)}



\newcommand{\inv}{^{-1}}
\newcommand{\CC}{{\cal C}}
\newcommand{\CCone}{\CC^1}
\newcommand{\Span}{{\rm span}}
\newcommand{\rank}{{\rm rank}}
\newcommand{\trace}{{\rm tr}}
\newcommand{\RE}{{\rm Re}}
\newcommand{\IM}{{\rm Im}}
\newcommand{\nulls}{{\rm null\;space}}

\newcommand{\dps}{\displaystyle}
\newcommand{\arraystart}{\renewcommand{\arraystretch}{1.8}}
\newcommand{\arrayfinish}{\renewcommand{\arraystretch}{1.2}}
\newcommand{\Start}[1]{\vspace{0.08in}\noindent {\bf Section~\ref{#1}}}
\newcommand{\exer}[1]{\noindent {\bf \ref{#1}}}
\newcommand{\ans}{}
\newcommand{\matthree}[9]{\left(\begin{array}{rrr} #1 & #2 & #3 \\ #4 & #5 & #6
\\ #7 & #8 & #9\end{array}\right)}
\newcommand{\cvectwo}[2]{\left(\begin{array}{c} #1 \\ #2\end{array}\right)}
\newcommand{\cmatthree}[9]{\left(\begin{array}{ccc} #1 & #2 & #3 \\ #4 & #5 &
#6 \\ #7 & #8 & #9\end{array}\right)}
\newcommand{\vecthree}[3]{\left(\begin{array}{r} #1 \\ #2 \\
#3\end{array}\right)}
\newcommand{\cvecthree}[3]{\left(\begin{array}{c} #1 \\ #2 \\
#3\end{array}\right)}
\newcommand{\cmattwo}[4]{\left(\begin{array}{cc} #1 & #2\\ #3
&#4\end{array}\right)}

\newcommand{\Matrix}[1]{\ensuremath{\left(\begin{array}{rrrrrrrrrrrrrrrrrr} #1 \end{array}\right)}}

\newcommand{\Matrixc}[1]{\ensuremath{\left(\begin{array}{cccccccccccc} #1 \end{array}\right)}}



\renewcommand{\labelenumi}{\theenumi)}
\newenvironment{enumeratea}%
{\begingroup
 \renewcommand{\theenumi}{\alph{enumi}}
 \renewcommand{\labelenumi}{(\theenumi)}
 \begin{enumerate}}
 {\end{enumerate}\endgroup}



\newcounter{help}
\renewcommand{\thehelp}{\thesection.\arabic{equation}}

%\newenvironment{equation*}%
%{\renewcommand\endequation{\eqno (\theequation)* $$}%
%   \begin{equation}}%
%   {\end{equation}\renewcommand\endequation{\eqno \@eqnnum
%$$\global\@ignoretrue}}

%\input{psfig.tex}

\author{Martin Golubitsky and Michael Dellnitz}

%\newenvironment{matlabEquation}%
%{\renewcommand\endequation{\eqno (\theequation*) $$}%
%   \begin{equation}}%
%   {\end{equation}\renewcommand\endequation{\eqno \@eqnnum
% $$\global\@ignoretrue}}

\newcommand{\soln}{\textbf{Solution:} }
\newcommand{\exercap}[1]{\centerline{Figure~\ref{#1}}}
\newcommand{\exercaptwo}[1]{\centerline{Figure~\ref{#1}a\hspace{2.1in}
Figure~\ref{#1}b}}
\newcommand{\exercapthree}[1]{\centerline{Figure~\ref{#1}a\hspace{1.2in}
Figure~\ref{#1}b\hspace{1.2in}Figure~\ref{#1}c}}
\newcommand{\para}{\hspace{0.4in}}

\renewenvironment{solution}{\suppress}{\endsuppress}

\ifxake
\newenvironment{matlabEquation}{\begin{equation}}{\end{equation}}
\else
\newenvironment{matlabEquation}%
{\let\oldtheequation\theequation\renewcommand{\theequation}{\oldtheequation*}\begin{equation}}%
  {\end{equation}\let\theequation\oldtheequation}
\fi

\makeatother

\begin{document}

\noindent In Exercises~\ref{c12.4.1} -- \ref{c12.4.7} use the method of 
undetermined coefficients to find particular solutions to the given 
differential equations.
\begin{exercise}  \label{c12.4.1}
$(D^2-3D+2)x = \sin(t)$.

\begin{solution}
\ans One solution to the differential equation is
\[
x(t) = \frac{1}{3}\cos t + \frac{1}{6}\sin t.
\]

\soln
\paragraph{Step 1.} First, find an annihilator for $g(t) = \sin t$.  Since
$\sin t$ is a solution to any homogeneous equation with eigenvalue
$\lambda = \pm i$, the differential equation $q(D) = D^2 + 1$ is an
annihilator.

\paragraph{Step 2.} Next, find the trial space of solutions for the
differential equation.  The homogeneous equation $\ddot{x} - 3\dot{x}
+ 2 = 0$ has eigenvalues $2$ and $3$.  These eigenvalues are distinct
from $\lambda$, so the trial space is the space of solutions to $q(D)x
= 0$, which is
\[
y(t) = c_1\cos t + c_2\sin t.
\]
\paragraph{Step 3.} Now substitute $y(t)$ into the differential equation:
\[
(D^2 - 3D + 2)(y(t)) = -(c_1\sin t + c_2\cos t) + 3(c_1\sin t - c_2\cos t)
+ 2(c_1\cos t + c_2\sin t) = \sin t.
\]
Solve this system for $c_1$ and $c_2$ to find that $y(t)$ is a solution to
the differential equation when $c_1 = \frac{1}{3}$ and $c_2 = \frac{1}{6}$.

\end{solution}
\end{exercise}
\begin{exercise}  \label{c12.4.2}
$\ddot{x}+2\dot{x}+x = t + e^t$.

\begin{solution}
\ans One solution to the differential equation is
\[
x(t) = \frac{1}{4}e^t + t - 2.
\]

\soln Let $g(t) = t$ and $h(t) = e^t$.  If $p(D) = D^2 + 2D + 1$, then the
sum of solutions to $p(D)x = g(t)$ and $p(D)x = h(t)$ is a solution to
$p(D)x = g(t) + h(t)$.  First find a solution to $p(D)x = g(t)$:

\paragraph{Step 1.} Since $t$ is a solution to any homogeneous equation
with a double eigenvalue at $\lambda = 0$, the differential equation
$q_1(D) = D^2$ is an annihilator.

\paragraph{Step 2.} The homogeneous equation $p(D)x = 0$ has a double
eigenvalue at $-1$.  This eigenvalue is distinct from $\lambda$, so
the trial space is the space of solutions to $q_1(D)x = 0$, which is
\[
y(t) = c_1 + c_2t.
\]
\paragraph{Step 3.} Now substitute $y(t)$ into $p(D)x = g(t)$:
\[
(D^2 + D2 + 1)(y(t)) = 2c_2 + (c_1 + c_2)t = g(t) = t.
\]
Solve this system for $c_1$ and $c_2$ to find that $y(t)$ is a solution to
the differential equation when $c_1 = -2$ and $c_2 = 1$.

\para Now find a solution for $p(D)x = h(t)$:

\paragraph{Step 1.} In this case, since $e^t$ is a solution to any
homogeneous equation with an eigenvalue at $\mu = 1$, the differential
equation $q_2(D) = D - 1$ is an annihilator.

\paragraph{Step 2.} The eigenvalue $\mu$ is distinct from the eigenvalues
of $p(D)$, so the trial space is the space of solutions to $q_2(D)x =
0$, which is
\[
z(t) = d_1e^t.
\]
\paragraph{Step 3.} Now substitute $z(t)$ into $p(D)x = h(t)$:
\[
(D^2 + D2 + 1)(z(t)) = d_1e^t + 2d_1e^t + d_1e^t = h(t) = e^t.
\]
Solve this system for $d_1$ to find that $z(t)$ is a solution to
the differential equation when $d_1 = \frac{1}{4}$.  Now, $x(t) = y(t) + z(t)$
is a solution to the differential equation $p(D)x = t + e^t$.


\end{solution}
\end{exercise}
\begin{exercise}  \label{c12.4.3}
$(D^3+6D^2+9D+4)x = e^{-t}$.

\begin{solution}
\ans One solution to the differential equation is
\[
x(t) = \frac{1}{6}t^2e^{-t}.
\]

\soln Let $p(D) = D^3 + 6D^2 + 9D + 4$, and let $g(t) = e^{-t}$.
\paragraph{Step 1.} An annihilator for $g(t)$ is $q(D) = D + 1$, since
$g(t)$ is a solution to any homogeneous equation with an eigenvalue at
$\mu = -1$.

\paragraph{Step 2.} The equation $p(D)x = 0$ has an eigenvalue at
$\lambda_1 = -4$, and a double eigenvalue at $\lambda_2 = -1$.  Since
$\mu = \lambda_2$, we find the trial space by applying $q(D)$ to both
sides of $p(D)x = g(t)$:
\[
q(D)(D^3 + 6D^2 + 9D + 4) = D^4 + 7D^3 + 15D^2 + 13D + 4 = 0.
\]
The general solution to this differential equation is
\[
x(t) = c_1e^{-4t} + c_2e^{-t} + c_3te^{-t} + c_4t^2e^{-t}.
\]
Since any solution to the homogeneous equation is equal to zero, we can set
$c_1 = c_2 = c_3 = 0$.  So the trial space is
\[
y(t) = c_4t^2e^{-t}.
\]
\paragraph{Step 3.} Substitute $y(t)$ into $p(D)x = g(t)$:
\[
p(D)(y(t)) = \frac{d^3x}{dt^3}(c_4t^2e^{-t}) +
6\frac{d^2x}{dt^2}(c_4t^2e^{-t}) + 9\frac{dx}{dt}(c_4t^2e^{-t}) +
4(c_4t^2e^{-t}) = 6c_4e^{-t} =  e^{-t}.
\]
Solve this system for $c_4$ to find that $y(t)$ is a solution to the
differential equation when $c_4 = \frac{1}{6}$.

\end{solution}
\end{exercise}
\begin{exercise}  \label{c12.4.4}
$(D^2+D-2)x = 3te^t$.

\begin{solution}
\ans One solution to the differential equation is
\[
x(t) = \frac{3}{4}t^2e^t - \frac{1}{2}te^t.
\]
\soln Let $p(D) = D^2 + D - 2$ and let $g(t) = 3te^t$.

\paragraph{Step 1.} An annihilator for $g(t)$ is
\[
q(D) = (D - 1)^2 = D^2 - 2D + 1,
\]
since $g(t)$ is a solution to any homogeneous differential equation
with a double eigenvalue at $\mu = 1$.

\paragraph{Step 2.} The equation $p(D)x = 0$ has eigenvalues at
$\lambda_1 = -2$ and $\lambda_2 = 1$.  Since $\mu = \lambda_2$, we
find the trial space by applying $q(D)$ to both sides of $p(D)x =
g(t)$:
\[
q(D)(D^2 + D - 2) = D^4 - D^3 - 3D^2 + 5D - 2 = 0.
\]
The general solution to this differential equation is
\[
x(t) = c_1e^{-2t} + c_2e^t + c_3te^t + c_4t^2e^t.
\]
Since any solution to the homogeneous equation is equal to zero, we can set
$c_1 = c_2 = 0$.  So the trial space is
\[
y(t) = c_3te^t + c_4t^2e^t.
\]
\paragraph{Step 3.} Substitute $y(t)$ into $p(D)x = g(t)$, obtaining
\[
c_3(3e^t) + c_4(2e^t + 4te^t) = 3te^t.
\]
Solve this system for $c_3$ and $c_4$ to find that $y(t)$ is a solution
to the differential equation when $c_3 = -\frac{1}{2}$ and
$c_4 = \frac{3}{4}$.

\end{solution}
\end{exercise}
\begin{exercise}  \label{c12.4.5}
$\ddot{x}+x = t\sin t$.

\begin{solution}
\ans One solution to the differential equation is
\[
x(t) = \frac{1}{4}(t\sin t- t^2\cos t).
\]

\soln Let $p(D) = D^2 + 1$ and let $g(t) = t\sin t$.
\paragraph{Step 1.} An annihilator for $g(t)$ is
\[
q(D) = (D^2 + 1)^2 = D^4 + 2D + 1,
\]
since $g(t)$ is a solution to any homogeneous differential equation
with a double eigenvalue at $\mu = \pm i$.

\paragraph{Step 2.} The equation $p(D)x = 0$ has eigenvalue $\lambda =
\pm i$.  Since $\mu = \lambda$, we find the trial space by applying
$q(D)$ to both sides of $p(D)x = g(t)$:
\[
q(D)(D^2 + 1) = D^6 + 2D^4 + 3D^2 + 1 = 0.
\]
The general solution to this differential equation is
\[
x(t) = c_1\cos t + c_2\sin t + c_3t\cos t + c_4t\sin t + c_5t^2\cos t
+ c_6t^2\sin t.
\]
Since any solution to the homogeneous equation is equal to zero, we can
set $c_1 = c_2 = 0$.  So the trial space is
\[
y(t) = c_3t\cos t + c_4t\sin t + c_5t^2\cos t + c_6t^2\sin t.
\]
\paragraph{Step 3.} Substitute $y(t)$ into $p(D)x = g(t)$, obtaining
\[
c_3(-2\sin t) + c_4(2\cos t) + c_5(2\cos t - 4t\sin t) +
c_6(2\sin t + 4t\cos t) = t\sin t.
\]
Solve this system for the scalars $c_j$ to find that $y(t)$ is a solution
to the differential equation when $c_3 = 0$, $c_4 = \frac{1}{4}$,
$c_5 = -\frac{1}{4}$, and $c_6 = 0$.

\end{solution}
\end{exercise}
\begin{exercise}  \label{c12.4.6}
$(D^3+D)x = 6t^2+\sin t$.

\begin{solution}
\ans One solution to the differential equation is
\[
x(t) = 2t^3 - 12t - \frac{1}{2}t\sin t.
\]

\soln Let $p(D) = D^3 + D$ and let $g(t) = 6t^2$ and $h(t) = \sin t$.  Then
the sum of solutions to $p(D)x = g(t)$ and $p(D)x = h(t)$ is a solution to
$p(D)x = g(t) + h(t)$.  So, first find a solution for $p(D)x = g(t)$:
\paragraph{Step 1.} An annihilator for $g(t)$ is
\[
q_1(D) = D^3,
\]
since $g(t)$ is a solution to any homogeneous differential equation
with a triple eigenvalue at $\mu_1 = 0$.

\paragraph{Step 2.} The equation $p(D)x = 0$ has eigenvalues
$\lambda_1 = 0$ and $\lambda_2 = \pm i$.  Since $\lambda_1 = \mu_1$,
we find the trial space by applying $q_1(D)$ to both sides of $p(D)x =
g(t)$:
\[
q_1(D)(D^3 + D) = D^6 + D^4 = 0.
\]
The general solution to this differential equation is
\[
x(t) = c_1 + c_2t + c_3t^2 + c_4t^3 + c_5\cos t + c_6\sin t.
\]
Since any solution to the homogeneous equation is equal to zero, we can
set $c_1 = c_5 = c_6 = 0$.  So the trial space is
\[
y(t) = c_2t + c_3t^2 + c_4t^3.
\]
\paragraph{Step 3.} Substitute $y(t)$ into $p(D)x = g(t)$, obtaining
\[
c_2 + c_3(3t^2) + c_4(t^2 + 6) = 6t^2.
\]
Solve this system for the scalars $c_j$ to find that $y(t)$ is a solution
to $p(D)x = g(t)$ when $c_2 = -12$, $c_3 = 0$, and $c_4 = 2$.

\para Now find a solution for $p(D)x = h(t)$:

\paragraph{Step 1.} In this case, $q_2(D) = D^2 + 1$ is an annihilator,
since $h(t)$ is a solution for any homogeneous differential equation
with an eigenvalue at $\mu_2 = \pm i$.

\paragraph{Step 2.} Since $\lambda_2 = \mu_2$, we find the trial space
by applying $q_2(D)$ to both sides of $p(D)x = h(t)$:
\[
q_2(D)(D^3 + D) = D^5 + 2D^3 + D = 0.
\]
The general solution to this differential equation is
\[
x(t) = d_1 + d_2\cos t + d_3\sin t + d_4t\cos t + d_5t\sin t.
\]
Since any solution to the homogeneous equation is equal to zero, we can
set $d_1 = d_2 = d_3 = 0$.  So the trial space is
\[
z(t) = d_4t\cos t + d_5t\sin t.
\]
\paragraph{Step 3.} Substitute $z(t)$ into $p(D)x = h(t)$, obtaining
\[
d_4(-4\cos t) + d_5(-2\sin t) = \sin t.
\]
Solve this system for $d_4$ and $d_5$ to find that $z(t)$ is a solution
to $p(D)x = h(t)$ when $d_4 = 0$ and $d_5 = -\frac{1}{2}$.  So $x(t) =
y(t) + z(t)$ is a solution to $(D^3 + D)x = 6t^2 + \sin t$.

\end{solution}
\end{exercise}
\begin{exercise}  \label{c12.4.7}
$(D^2+2D+2)x = 8e^{-t}\sin t$.

\begin{solution}
\ans One solution to the differential equation is
\[
x(t) = -4te^{-t}\cos t.
\]

\soln Let $p(D) = D^2 + 2D + 2$ and let $g(t) = 8e^{-t}\sin t$.
\paragraph{Step 1.} An annihilator for $g(t)$ is
\[
q(D) = D^2 + 2D + 2,
\]
since $g(t)$ is a solution to any homogeneous differential equation
with an eigenvalue at $\mu = -1 \pm i$.

\paragraph{Step 2.} The equation $p(D)x = 0$ has eigenvalue
$\lambda = -1 \pm i$.  Since $\mu = \lambda$, we find the trial space
by applying $q(D)$ to both sides of $p(D)x = g(t)$:
\[
q(D)(D^2 + 2D + 1) = D^4 + 4D^3 + 8D^2 + 8D + 4	= 0.
\]
The general solution to this differential equation is
\[
x(t) = c_1e^{-t}\cos t + c_2e^{-t}\sin t + c_3te^{-t}\cos t +
c_4te^{-t}\sin t.
\]
Since any solution to the homogeneous equation is equal to zero, we can
set $c_1 = c_2 = 0$.  So the trial space is
\[
y(t) = c_3te^{-t}\cos t + c_4te^{-t}\sin t.
\]
\paragraph{Step 3.} Substitute $y(t)$ into $p(D)x = g(t)$, obtaining
\[
c_3(-2e^{-t}\sin t) + c_4(2e^{-t}\cos t) = 8e^{-t}\sin t.
\]
Solve this system for $c_3$ and $c_4$ to find that $y(t)$ is a solution
to the differential equation when $c_3 = -4$ and $c_4 = 0$.

\end{solution}
\end{exercise}
\end{document}
