\documentclass{ximera}

\usepackage{epsfig}

\graphicspath{
  {./}
  {figures/}
}


\usepackage{morewrites}

%\newcounter{ccounter}
%\setcounter{ccounter}{1}
%\newcommand{\Chapter}[1]{\setcounter{chapter}{\arabic{ccounter}}\chapter{#1}\addtocounter{ccounter}{1}}

%\newcommand{\section}[1]{\section{#1}\setcounter{thm}{0}\setcounter{equation}{0}}

%\renewcommand{\theequation}{\arabic{chapter}.\arabic{section}.\arabic{equation}}
%\renewcommand{\thefigure}{\arabic{chapter}.\arabic{figure}}
%\renewcommand{\thetable}{\arabic{chapter}.\arabic{table}}

%\newcommand{\Sec}[2]{\section{#1}\markright{\arabic{ccounter}.\arabic{section}.#2}\setcounter{equation}{0}\setcounter{thm}{0}\setcounter{figure}{0}}

\newcommand{\Sec}[2]{\section{#1}}

\setcounter{secnumdepth}{2}
%\setcounter{secnumdepth}{1} 

%\newcounter{THM}
%\renewcommand{\theTHM}{\arabic{chapter}.\arabic{section}}

\newcommand{\trademark}{{R\!\!\!\!\!\bigcirc}}
%\newtheorem{exercise}{}

\newcommand{\dfield}{{\sf dfield9}}
\newcommand{\pplane}{{\sf pplane9}}

\newcommand{\EXER}{\section*{Exercises}}%\vspace*{0.2in}\hrule\small\setcounter{exercise}{0}}
\newcommand{\CEXER}{}%\vspace{0.08in}\begin{center}Computer Exercises\end{center}}
\newcommand{\TEXER}{} %\vspace{0.08in}\begin{center}Hand Exercises\end{center}}
\newcommand{\AEXER}{} %\vspace{0.08in}\begin{center}Hand Exercises\end{center}}

% BADBAD: \newcommand{\Bbb}{\bf}

\newcommand{\R}{\mbox{$\Bbb{R}$}}
\newcommand{\C}{\mbox{$\Bbb{C}$}}
\newcommand{\Z}{\mbox{$\Bbb{Z}$}}
\newcommand{\N}{\mbox{$\Bbb{N}$}}
\newcommand{\D}{\mbox{{\bf D}}}
\usepackage{amssymb}
%\newcommand{\qed}{\hfill\mbox{\raggedright$\square$} \vspace{1ex}}
%\newcommand{\proof}{\noindent {\bf Proof:} \hspace{0.1in}}

\newcommand{\setmin}{\;\mbox{--}\;}
\newcommand{\Matlab}{{M\small{AT\-LAB}} }
\newcommand{\Matlabp}{{M\small{AT\-LAB}}}
\newcommand{\computer}{\Matlab Instructions}
\newcommand{\half}{\mbox{$\frac{1}{2}$}}
\newcommand{\compose}{\raisebox{.15ex}{\mbox{{\scriptsize$\circ$}}}}
\newcommand{\AND}{\quad\mbox{and}\quad}
\newcommand{\vect}[2]{\left(\begin{array}{c} #1_1 \\ \vdots \\
 #1_{#2}\end{array}\right)}
\newcommand{\mattwo}[4]{\left(\begin{array}{rr} #1 & #2\\ #3
&#4\end{array}\right)}
\newcommand{\mattwoc}[4]{\left(\begin{array}{cc} #1 & #2\\ #3
&#4\end{array}\right)}
\newcommand{\vectwo}[2]{\left(\begin{array}{r} #1 \\ #2\end{array}\right)}
\newcommand{\vectwoc}[2]{\left(\begin{array}{c} #1 \\ #2\end{array}\right)}



\newcommand{\inv}{^{-1}}
\newcommand{\CC}{{\cal C}}
\newcommand{\CCone}{\CC^1}
\newcommand{\Span}{{\rm span}}
\newcommand{\rank}{{\rm rank}}
\newcommand{\trace}{{\rm tr}}
\newcommand{\RE}{{\rm Re}}
\newcommand{\IM}{{\rm Im}}
\newcommand{\nulls}{{\rm null\;space}}

\newcommand{\dps}{\displaystyle}
\newcommand{\arraystart}{\renewcommand{\arraystretch}{1.8}}
\newcommand{\arrayfinish}{\renewcommand{\arraystretch}{1.2}}
\newcommand{\Start}[1]{\vspace{0.08in}\noindent {\bf Section~\ref{#1}}}
\newcommand{\exer}[1]{\noindent {\bf \ref{#1}}}
\newcommand{\ans}{}
\newcommand{\matthree}[9]{\left(\begin{array}{rrr} #1 & #2 & #3 \\ #4 & #5 & #6
\\ #7 & #8 & #9\end{array}\right)}
\newcommand{\cvectwo}[2]{\left(\begin{array}{c} #1 \\ #2\end{array}\right)}
\newcommand{\cmatthree}[9]{\left(\begin{array}{ccc} #1 & #2 & #3 \\ #4 & #5 &
#6 \\ #7 & #8 & #9\end{array}\right)}
\newcommand{\vecthree}[3]{\left(\begin{array}{r} #1 \\ #2 \\
#3\end{array}\right)}
\newcommand{\cvecthree}[3]{\left(\begin{array}{c} #1 \\ #2 \\
#3\end{array}\right)}
\newcommand{\cmattwo}[4]{\left(\begin{array}{cc} #1 & #2\\ #3
&#4\end{array}\right)}

\newcommand{\Matrix}[1]{\ensuremath{\left(\begin{array}{rrrrrrrrrrrrrrrrrr} #1 \end{array}\right)}}

\newcommand{\Matrixc}[1]{\ensuremath{\left(\begin{array}{cccccccccccc} #1 \end{array}\right)}}



\renewcommand{\labelenumi}{\theenumi)}
\newenvironment{enumeratea}%
{\begingroup
 \renewcommand{\theenumi}{\alph{enumi}}
 \renewcommand{\labelenumi}{(\theenumi)}
 \begin{enumerate}}
 {\end{enumerate}\endgroup}



\newcounter{help}
\renewcommand{\thehelp}{\thesection.\arabic{equation}}

%\newenvironment{equation*}%
%{\renewcommand\endequation{\eqno (\theequation)* $$}%
%   \begin{equation}}%
%   {\end{equation}\renewcommand\endequation{\eqno \@eqnnum
%$$\global\@ignoretrue}}

%\input{psfig.tex}

\author{Martin Golubitsky and Michael Dellnitz}

%\newenvironment{matlabEquation}%
%{\renewcommand\endequation{\eqno (\theequation*) $$}%
%   \begin{equation}}%
%   {\end{equation}\renewcommand\endequation{\eqno \@eqnnum
% $$\global\@ignoretrue}}

\newcommand{\soln}{\textbf{Solution:} }
\newcommand{\exercap}[1]{\centerline{Figure~\ref{#1}}}
\newcommand{\exercaptwo}[1]{\centerline{Figure~\ref{#1}a\hspace{2.1in}
Figure~\ref{#1}b}}
\newcommand{\exercapthree}[1]{\centerline{Figure~\ref{#1}a\hspace{1.2in}
Figure~\ref{#1}b\hspace{1.2in}Figure~\ref{#1}c}}
\newcommand{\para}{\hspace{0.4in}}

\renewenvironment{solution}{\suppress}{\endsuppress}

\ifxake
\newenvironment{matlabEquation}{\begin{equation}}{\end{equation}}
\else
\newenvironment{matlabEquation}%
{\let\oldtheequation\theequation\renewcommand{\theequation}{\oldtheequation*}\begin{equation}}%
  {\end{equation}\let\theequation\oldtheequation}
\fi

\makeatother


\title{Variation of Parameters for Systems}

\begin{document}
\begin{abstract}
\end{abstract}
\maketitle


\label{sec:LinInhomSys}
\index{variation of parameters}

In this section we consider solving inhomogeneous systems of linear
differential equations  of the form
\arraystart
\begin{equation}  \label{eq:linihsys}
\frac{dX}{dt}  =  A(t)X + G(t) 
\end{equation}
\arrayfinish
where $A(t)=(a_{ij}(t))$ is an $n\times n$ matrix and
$G(t)=(g_1(t),\ldots,g_n(t))^t$ is a vector of continuous functions.  
The system is {\em homogeneous\/} when $G(t)=0$.

We divide this discussion into two parts: finding the 
general solution\index{variation of parameters!general solution} 
of the homogeneous\index{homogeneous} equation and finding 
a particular 
solution\index{variation of parameters!particular solution} to the 
inhomogeneous equation using variation of parameters.

\subsection*{Homogeneous Nonconstant Coefficient Systems}

From the theory of constant coefficient systems of linear differential 
equations, we know abstractly how to write  a  basis of solutions to
the homogeneous system  $\dot{X}=AX$ where $A$ is an $n\times n$
constant matrix.  That basis is 
\[
X_j(t) = e^{tA}v_j,
\]
where $\{v_1,\ldots,v_n\}$ is a basis of $\R^n$.  For the 
homogeneous\index{homogeneous} 
system of $n$ nonconstant coefficient linear differential equations
\begin{equation}  \label{E:NCCH}
\dot{X} = A(t) X
\end{equation}
a similar statement is true.   It should be noted, however, that only in 
special cases can \eqref{E:NCCH} be solved in closed form.

\begin{proposition}  \label{P:NCCH}
Let $v_1,\ldots,v_n$ be a basis\index{basis} 
for $\R^n$.  Then there exist solutions 
\[
X_1(t),\ldots,X_n(t)
\]
of \eqref{E:NCCH} satisfying $X_j(0)=v_j$ and these
solutions form a basis of solutions\index{basis!of solutions} 
to \eqref{E:NCCH}.
\end{proposition}

\begin{proof} The theory of differential equations implies that there exists a 
unique solution to \eqref{E:NCCH}  satisfying the initial condition $X(0)=X_0$ 
for any $X_0\in\R^n$.   Therefore, there exist solutions $X_j(t)$ such that 
$X_1(0)=v_j$.  We claim that these solutions form a basis for the vector 
space of all solutions to \eqref{E:NCCH}.   

Let $X(t)$ be any solution to \eqref{E:NCCH} with initial condition $X(0)=X_0$. 
Since $\{X_1(0),\ldots,X_n(0)\}$ is a basis for $\R^n$, we can find scalars 
$\alpha_j$ such that 
\[
X_0 = \alpha_1X_1(0) + \cdots + \alpha_nX_n(0).
\]
Linearity of \eqref{E:NCCH} implies that 
\[
\alpha_1X_1(t) + \cdots + \alpha_nX_n(t)
\]
is a solution.  Therefore, uniqueness of solutions to \eqref{E:NCCH} implies 
that 
\[
X(t) = \alpha_1X_1(t) + \cdots + \alpha_nX_n(t)
\]
for all time $t$, which proves the proposition. \end{proof}

Existence of solutions to the initial value problem $X(t_0)=X_0$ also
guarantees that the vectors $X_1(t_0),\ldots,X_n(t_0)$ must be linearly 
independent\index{linearly!independent}.  
Indeed, since $X_1(t),\ldots,X_n(t)$ is a 
basis of solutions\index{basis!of solutions} to
\eqref{E:NCCH}, we must be able to write 
\[
X_0 = \alpha_1X_1(t_0) + \cdots + \alpha_nX_n(t_0)
\]
for scalars $\alpha_1,\ldots,\alpha_n$ in order for a solution to this 
initial value problem to exist.  That is, the vectors 
$X_1(t_0),\ldots,X_n(t_0)$  must span $\R^n$.   Hence they are linearly 
independent.  We have proved the following:

\begin{lemma}  \label{L:DEspan}
Let $X_1,\ldots,X_n$ be a linearly independent\index{linearly!independent} 
set of solutions to \eqref{E:NCCH}.
Then, for every $t$, the matrix 
\begin{equation}   \label{E:Y(t)}
Y(t) = \left(X_1(t)|\cdots |X_n(t)\right) 
\end{equation}
is invertible\index{invertible}.
\end{lemma}

\begin{proof}  Proposition~\ref{P:NCCH} implies that the columns of $Y(t)$ are 
linearly independent.  Hence $Y(t)$ is invertible.  \end{proof}

Rather than relying on the general existence and uniqueness theory for
systems of differential equations, there is a computationally direct way of 
proving Lemma~\ref{L:DEspan}.   It is possible to show that the determinant 
of the matrix $Y(t)$ in \eqref{E:Y(t)} is nonzero and hence $Y(t)$ is invertible.
This approach is carried out in Section~\ref{S:wronskian}


\subsection*{The Theory of Variation of Parameters}
\index{variation of parameters!theory}

Let $X_1(t),\ldots,X_n(t)$ be a 
basis of solutions\index{basis!of solutions} to the homogeneous 
system $\dot{X}=A(t)X$.   In the method of variation of parameters we look 
for solutions to  the inhomogeneous system \eqref{eq:linihsys} of the form
\[
X(t) = c_1(t)X_1(t) + \cdots + c_n(t)X_n(t),
\]
where $c_1(t),\ldots,c_n(t)$ are differentiable functions.  In order to 
find a method for determining the functions $c_j$, we assume that $X$ is 
a solution to \eqref{eq:linihsys}.  

Use the product rule\index{product rule} to compute
\begin{eqnarray*}
\dps\frac{dX}{dt} & = &\sum_{j=1}^n\left(c_j\frac{dX_j}{dt}
+\frac{dc_j}{dt} X_j\right)\\
& = & \sum_{j=1}^n\left( c_jAX_j+\frac{dc_j}{dt}X_j\right)\\
& = & AX+\sum_{j=1}^n \frac{dc_j}{dt}X_j.
\end{eqnarray*}
It follows that $X$ is a solution of \eqref{eq:linihsys} if and only if
\begin{equation}  \label{eq:dcjXj}
G(t) = \frac{dc_1}{dt}(t) X_1(t) + \cdots + \frac{dc_n}{dt}(t) X_n(t).
\end{equation}
So variation of parameters works if we can find functions $c_j$ that 
satisfy \eqref{eq:dcjXj}.

We claim that it is always possible to find functions $d_j(t)$ so that 
\begin{equation}  \label{E:djXj}
G(t) = d_1(t)X_1(t) + \cdots + d_n(t)X_n(t).
\end{equation} 
To verify \eqref{E:djXj}, note that  \eqref{E:djXj} can be rewritten in matrix 
form as
\[
Y(t) D(t) = G(t)
\]
where $Y(t)$ is defined in \eqref{E:Y(t)} and
\[
D(t)=(d_1(t),\ldots,d_n(t))^t.
\]
Using the invertibility\index{invertible} of 
$Y(t)$, as proved in Lemma~\ref{L:DEspan}, we find that
\[
D(t) = Y(t)\inv G(t).
\] 
It now follows that \eqref{eq:dcjXj} is satisfied by integrating the 
differential equations $\dot{c}_j=d_j$, where the functions $d_j(t)$ 
are defined in \eqref{E:djXj}, to find the functions $c_j$.  

We summarize this discussion as follows. 
\begin{theorem}[Variation of Parameters: Systems] \label{thm:varparsys}
\index{variation of parameters!for systems}
Consider 
\begin{equation}  \label{eq:VPS}
\begin{array}{rcl}
\dps \frac{dX}{dt} & = & A(t)X+G(t)\\
X(t_0) & = & X_0.
\end{array}
\end{equation}
\begin{itemize}
\item[(a)]  Let $X_1(t),\ldots,X_n(t)$ be a 
basis of solutions\index{basis!of solutions} to 
the homogeneous differential equation $\dot{X}=A(t)X$.  
\item[(b)]  Let $Y(t)$ be defined as in \eqref{E:Y(t)} and define 
$D(t)=(d_1(t),\ldots,d_n(t))^t$ by 
\begin{equation}  \label{e:D(t)}
D(t) = Y(t)\inv G(t).
\end{equation}
\item[(c)]  Let the functions $c_1(t),\ldots,c_n(t)$ satisfy the initial 
value problem\index{variation of parameters!initial value problem} 
\begin{equation}  \label{eq:cj0}
\frac{dc_j}{dt}=d_j \AND  c_1(t_0)X_1(t_0) + \cdots +c_n(t_0)X_n(t_0) = X_0.
\end{equation}
\end{itemize}
Then
\[
X(t) =  c_1(t)X_1(t) + \cdots + c_n(t)X_n(t)
\]
is the unique solution\index{uniqueness of solutions} of \eqref{eq:VPS}.
\end{theorem}

Note that when $n\ge 3$, the hand computation of $Y(t)\inv$ can be  
painful.

\subsection*{Examples with Constant Coefficient Matrix $A$ when $n=2$}

We consider two examples of variation of parameters applied to 
forced systems of differential equations of the form 
\[
\dot{X} = AX + G(t).
\]

\subsubsection*{An Example with Two Real Simple Eigenvalues}

Consider the initial value problem \eqref{eq:VPS} where
\begin{equation}  \label{eq:exinhom2}
A=\mattwo{-1}{-8}{-16}{7}, \quad G(t)= \vectwo{1-t}{-2-t}, \quad 
X_0 =  \vectwo{-1}{-1}.
\end{equation}
Variation of parameters proceeds in four steps:

\paragraph{Step 1: Solutions to the homogeneous equation.} The 
eigenvalues\index{eigenvalue!real!distinct} and 
eigenvectors\index{eigenvector} of $A$ can be 
found either by using \Matlab or by hand.  The eigenvalues are $\lambda_1=-9$ 
and $\lambda_2=15$, and the associated eigenvectors are $v_1=(1,1)^t$ 
and $v_2=(1,-2)^t$.  Therefore, solutions of the homogeneous equation are:
\[
X_1(t) = e^{-9t}\vectwo{1}{1} \AND X_2(t) = e^{15t}\vectwo{1}{-2}.
\]

\paragraph{Step 2: Computation of the vector $D(t)$.}   We compute the inverse 
\[
Y(t)\inv = \mattwo{e^{-9t}}{e^{15t}}{e^{-9t}}{-2e^{15t}}\inv = 
-\frac{1}{3e^{6t}}\mattwo{-2e^{15t}}{-e^{15t}}{-e^{-9t}}{e^{-9t}} =
\frac{1}{3}\mattwo{2e^{9t}}{e^{9t}}{e^{-15t}}{-e^{-15t}}.
\]
The vector $D(t)$ is:
\[
D(t) = Y(t)\inv G(t) = 
\frac{1}{3}\mattwo{2e^{9t}}{e^{9t}}{e^{-15t}}{-e^{-15t}}\vectwo{1-t}{-2-t}
= \vectwo{-te^{9t}}{e^{-15t}}
\]

\paragraph{Step 3: Computation of the functions $c_j(t)$.}   By assumption 
\[
X_0=c_1(0)X_1(0)+c_2(0)X_2(0).
\]
Therefore
\[
\vectwo{-1}{-1} = c_1(0)\vectwo{1}{1} + c_2(0)\vectwo{1}{-2},
\]
from which it follows that $c_1(0)=-1$ and $c_2(0)=0$.  Since
\[
\dot{c}_1 = d_1 = -te^{9t}  \AND \dot{c}_2 = d_2 = e^{-15t},
\]
it follows that 
\[
\begin{array}{rclcl}
c_1(t) & = & \int_0^t (-\tau) e^{9\tau}d\tau - 1
& = &  -\frac{1}{9}\left(t-\frac{1}{9}\right)e^{9t}-\frac{82}{81} \\
c_2(t) & = & \int_0^t e^{-15t}d\tau & = & \frac{1}{15}\left(1-e^{-15t}\right).
\end{array}
\]

\paragraph{Step 4: Write out the solution.}  The solution $X(t)$ of the 
initial value problem \eqref{eq:exinhom2} is now given by
\begin{eqnarray*}
X(t) & = & c_1(t)e^{-9t}\vectwo{1}{1} + c_2(t)e^{15t}\vectwo{1}{-2} \\
& = &  \left(\begin{array}{c}
-\frac{1}{9}\left(t-\frac{1}{9}\right)-
\frac{82}{81}e^{-9t}+\frac{1}{15}\left( e^{15t}-1\right)\\
-\frac{1}{9}\left(t-\frac{1}{9}\right)-
\frac{82}{81}e^{-9t}-\frac{2}{15}\left( e^{15t}-1\right)
\end{array}\right).
\end{eqnarray*}

\subsubsection*{An Example with a Nontrivial Jordan Block}
\index{Jordan block}

Solve the system
\begin{equation}  \label{e:2djb}
\begin{array}{rcl}
\dot{X} & = & A X + (t,1)^t \\
X(0) & = & (1,-1)^t,
\end{array}
\end{equation}
where 
\[
A = \mattwo{2}{1}{0}{2}.
\]

\paragraph{Step 1:}  A basis for solutions to the homogeneous equation 
$\dot{X}=AX$ is
\[
X_1(t) = e^{2t}\vectwo{1}{0} \AND X_2(t) = e^{2t}\vectwo{t}{1}.
\]

\paragraph{Step 2:}  We compute
\[
D(t) =  e^{-2t}\mattwo{1}{t}{0}{1}\inv G(t) = 
e^{-2t}\mattwo{1}{-t}{0}{1}\vectwo{t}{1} = e^{-2t}\vectwo{0}{1}.
\]

\paragraph{Step 3:}  Note that 
\[
X(0) = \vectwo{1}{-1} = c_1(0)\vectwo{1}{0} + c_2(0)\vectwo{0}{1}.
\]
So $c_1(0)=1$ and $c_2(0)=-1$.  Now solve the differential equations
\[
\begin{array}{rclcl}
\dot{c}_1 & = & d_1(t) & = & 0 \\
\dot{c}_2 & = & d_2(t) & = & e^{-2t},
\end{array}
\]
obtaining
\begin{eqnarray*}
c_1(t) & = & 1 \\
c_2(t) & = & -\frac{1}{2}(e^{-2t}+1).
\end{eqnarray*}

\paragraph{Step 4:}  It follows from Theorem~\ref{thm:varparsys} that the 
solution to \eqref{e:2djb} is
\[
X(t) = c_1(t)X_1(t) + c_2(t)X_2(t) = -\frac{1}{2}\left(\begin{array}{c}
t(1+e^{2t})-2e^{2t} \\ e^{2t}+1 \end{array}\right).
\]
 


\EXER

\includeexercises






\end{document}
